\PassOptionsToPackage{unicode=true}{hyperref} % options for packages loaded elsewhere
\PassOptionsToPackage{hyphens}{url}
%
\documentclass[]{article}
\usepackage{lmodern}
\usepackage{amssymb,amsmath}
\usepackage{ifxetex,ifluatex}
\usepackage{fixltx2e} % provides \textsubscript
\ifnum 0\ifxetex 1\fi\ifluatex 1\fi=0 % if pdftex
  \usepackage[T1]{fontenc}
  \usepackage[utf8]{inputenc}
  \usepackage{textcomp} % provides euro and other symbols
\else % if luatex or xelatex
  \usepackage{unicode-math}
  \defaultfontfeatures{Ligatures=TeX,Scale=MatchLowercase}
\fi
% use upquote if available, for straight quotes in verbatim environments
\IfFileExists{upquote.sty}{\usepackage{upquote}}{}
% use microtype if available
\IfFileExists{microtype.sty}{%
\usepackage[]{microtype}
\UseMicrotypeSet[protrusion]{basicmath} % disable protrusion for tt fonts
}{}
\IfFileExists{parskip.sty}{%
\usepackage{parskip}
}{% else
\setlength{\parindent}{0pt}
\setlength{\parskip}{6pt plus 2pt minus 1pt}
}
\usepackage{hyperref}
\hypersetup{
            pdfborder={0 0 0},
            breaklinks=true}
\urlstyle{same}  % don't use monospace font for urls
\usepackage{longtable,booktabs}
% Fix footnotes in tables (requires footnote package)
\IfFileExists{footnote.sty}{\usepackage{footnote}\makesavenoteenv{longtable}}{}
\usepackage{graphicx,grffile}
\makeatletter
\def\maxwidth{\ifdim\Gin@nat@width>\linewidth\linewidth\else\Gin@nat@width\fi}
\def\maxheight{\ifdim\Gin@nat@height>\textheight\textheight\else\Gin@nat@height\fi}
\makeatother
% Scale images if necessary, so that they will not overflow the page
% margins by default, and it is still possible to overwrite the defaults
% using explicit options in \includegraphics[width, height, ...]{}
\setkeys{Gin}{width=\maxwidth,height=\maxheight,keepaspectratio}
\setlength{\emergencystretch}{3em}  % prevent overfull lines
\providecommand{\tightlist}{%
  \setlength{\itemsep}{0pt}\setlength{\parskip}{0pt}}
\setcounter{secnumdepth}{0}
% Redefines (sub)paragraphs to behave more like sections
\ifx\paragraph\undefined\else
\let\oldparagraph\paragraph
\renewcommand{\paragraph}[1]{\oldparagraph{#1}\mbox{}}
\fi
\ifx\subparagraph\undefined\else
\let\oldsubparagraph\subparagraph
\renewcommand{\subparagraph}[1]{\oldsubparagraph{#1}\mbox{}}
\fi

% set default figure placement to htbp
\makeatletter
\def\fps@figure{htbp}
\makeatother


\date{}

\begin{document}

\includegraphics[width=1.58268in,height=1.5748in]{./media/image1.png}\textbf{UNIVERSITÀ
DEGLI STUDI DI PERUGIA}

DIPARTIMENTO DI MEDICINA

\emph{Corso di Laurea Magistrale in} \emph{MEDICINA E CHIRURGIA}

\textbf{Tesi di Laurea}

\emph{Analgesia Epidurale per il travaglio di parto e allattamento al
seno: dati preliminari}

\textbf{LAUREANDO}

\emph{Tommaso Valigi}

\textbf{RELATORE}

\emph{Prof.ssa Simonetta Tesoro}

Anno Accademico 2017/2018

\hypertarget{indice}{%
\section{INDICE}\label{indice}}

\begin{enumerate}
\def\labelenumi{\arabic{enumi}.}
\item
  INTRODUZIONE1
\item
  ALLATTAMENTO AL SENO4

  \begin{enumerate}
  \def\labelenumii{\arabic{enumii}.}
  \item
    Le raccomandazioni5
  \item
    Dati epidemiologici5
  \item
    Benefici derivanti dall'allattamento al seno per il bambino5
  \item
    Durata ottimale dell'allattamento5
  \end{enumerate}
\item
  FISIOLOGIA DELLA LATTAZIONE4

  \begin{enumerate}
  \def\labelenumii{\arabic{enumii}.}
  \setcounter{enumii}{4}
  \item
    Lo sviluppo della mammella5
  \item
    Pubertà e periodo fetale: i cambiamenti associati al ciclo
    mestruale5
  \item
    La mammella durante la gravidanza5
  \item
    La mammella durante l'allattamento5
  \item
    La lattogenesi5
  \item
    Regolazione dello stadio II della lattogenesi5
  \item
    Cause di fallimento della lattogenesi II5
  \item
    Galattopoiesi5
  \item
    Eiezione del latte5
  \end{enumerate}
\item
  SCALE DI VALUTAZIONE DELL'ALLATTAMENTO4
\item
  PARTOANALGESIA4

  \begin{enumerate}
  \def\labelenumii{\arabic{enumii}.}
  \setcounter{enumii}{13}
  \item
    Fasi del travaglio 5
  \item
    Le vie del dolore 5
  \item
    Effetti sistemici del dolore del parto 5
  \item
    Obiettivi della partoanalgesia 5
  \item
    Timing dell'analgesia epidurale in travaglio 5
  \item
    Tecniche neuroassiali di analgesia in travaglio 5
  \item
    Farmaci5
  \end{enumerate}
\item
  OBIETTIVI DELLO STUDIO, MATERIALI E METODI4
\item
  RISULTATI4
\item
  DISCUSSIONE E CONCLUSIONE4

  \begin{enumerate}
  \def\labelenumii{\arabic{enumii}.}
  \setcounter{enumii}{20}
  \item
    Limiti dello studio5
  \item
    Conclusioni5
  \end{enumerate}
\end{enumerate}

\hypertarget{introduzione}{%
\section{INTRODUZIONE}\label{introduzione}}

L'allattamento al seno comporta numerosi vantaggi per il neonato poiché
riduce il rischio di morte improvvisa infantile, infezioni delle basse
vie aeree, gastroenteriti aspecifiche, diabete mellito di tipo 1 e 2 e
di leucemia acuta sia mieloide che linfoide in età pediatrica
\textsuperscript{(1,2,3)}. L'allattamento esclusivo al seno inoltre
potrebbe diminuire anche l'incidenza di malattie allergiche contrastando
l'esposizione del bambino ad antigeni esogeni e promuovendo la
maturazione della mucosa gastrointestinale \textsuperscript{(4)}. Dal
punto di vista materno, nel breve termine l'allattamento riduce le
perdite ematiche postpartum e promuove una rapida involuzione uterina
\textsuperscript{(5)}. A lungo termine le donne che hanno allattato al
seno hanno una minore incidenza di cancro della mammella, dell'ovaio e
di diabete mellito di tipo II \textsuperscript{(1)}.

In considerazione dei benefici sia materni che neonatali,
l'Organizzazione Mondiale della Sanità e l'Unicef hanno lanciato nel
1992 una campagna per aiutare le madri a sostenere l'allattamento al
seno anche attraverso un'adeguata organizzazione e formazione degli
operatori sanitari (Baby-Friendly Hospital Initiative - BFHI)
\textsuperscript{(6)}. Per lo stesso motivo l'American Academy of
Pediatrics e l'American College of Obstetrics and Gynecologist
raccomandano l'allattamento al seno per i primi sei mesi di vita del
bambino e lo consigliano fino a 1 anno di età qualora sia possibile
\textsuperscript{(7,8)}. Contemporaneamente la Società Italiana di
Pediatria e Neonatologia (SIP/SIN) ha pubblicato le medesime
raccomandazioni nel 2015 \textsuperscript{(9)}.

I fattori che possono influenzare l'allattamento al seno sono
molteplici, sia materni che neonatali.

In alcuni studi il successo dell'allattamento si associa all'attitudine
stessa che le madri mostrano nell'intraprendere questa pratica; tale
intenzionalità sarebbe anche in grado di mitigare l'influenza che hanno
altri fattori demografici e sociali sull'allattamento
\textsuperscript{(10)}. Le possibilità di intraprendere e continuare
l'allattamento esclusivo al seno sono ridotte in caso di diabete e/o
obesità materna; il prolungamento del secondo stadio del travaglio, il
parto strumentale o quello cesareo si associano al medesimo rischio.
Altri fattori che possono compromettere l'allattamento sono il rialzo
della temperatura materna (che può associarsi a ipertermia fetale), la
somministrazione di ossitocina durante il travaglio, un ritardato
contatto madre-figlio subito dopo il parto \emph{(skin-to-skin)} e
l'alterazione dello stato neuro-comportamentale del neonato.

Infine, un ruolo decisivo per il successo dell'allattamento al seno è
attribuito alla capacità dell'ambiente sanitario di supportare
l'interazione madre-figlio nei giorni successivi alla nascita del
bambino.

Le tecniche di analgesia per il travaglio sono capaci di ridurre lo
stress materno connesso al parto (consumo di ossigeno, secrezione di
catecolamine, aumento del cortisolo plasmatico); allo stesso tempo però
l'analgesia epidurale non modifica la risposta del sistema
adreno-simpatico fetale che svolge un importante ruolo negli eventi
fisiologici che permettono l'adattamento del feto allo stress
\textsuperscript{(11,12,13)}. In questo modo il controllo del dolore
durante il travaglio potrebbe migliorare l'interazione della diade
madre-neonato senza impedire l'instaurarsi dei comportamenti e dei
fenomeni neuroendocrini finalizzati all'allattamento. Tuttavia in alcuni
studi è stato dimostrato che la partoanalgesia ostacola l'allattamento
nelle prime 24 ore di vita \textsuperscript{(14,15)} e ne determina una
sospensione prematura nei mesi successivi alla nascita
\textsuperscript{(16)}. Tali fenomeni sarebbero connessi al
prolungamento del travaglio e/o alla somministrazione di oppioidi per
via neuroassiale \textsuperscript{(17,18)}.

Fisiologicamente l'integrità del sistema nervoso centrale del neonato è
molto importante per permettere che l'attacco al seno materno sia
adeguato \textsuperscript{(19)}. I farmaci somministrati per via
epidurale, una volta trasferiti nel torrente circolatorio materno,
possono attraversare la barriera placentare e potenzialmente
compromettere i riflessi di ricerca (rooting), di suzione (sucking) e di
deglutizione (swallowing) che sono alla base del comportamento
nutrizionale del neonato \textsuperscript{(20)}. Il mancato istaurarsi
di un comportamento adeguato potrebbe indurre le madri a sospendere
l'allattamento al seno per prediligere la somministrazione di latte
artificiale mediante biberon, introducendo tecniche di suzione non
appropriate per la mammella \textsuperscript{(21)}. Inoltre, sebbene
nello studio di \emph{Kulski et al.} l'efficace rimozione meccanica del
primo latte dai dotti mammari non sembrerebbe necessaria per la sua
produzione prolattina-dipendente nelle 36-96 ore dopo il parto
\textsuperscript{(22)}, altri lavori mostrano che il precoce svuotamento
mammario incrementa la lattogenesi di seconda fase e la mantiene nel
lungo periodo \textsuperscript{(23,24)}, forse permettendo la rimozione
di fattori locali inibitori \textsuperscript{(25)}. In altri lavori non
è stata dimostrata un'associazione negativa tra partoanalgesia e
allattamento \textsuperscript{(26,27)} né che la somministrazione di
oppiodi per via epidurale possa compromettere gli aspetti
neuro-comportamentali del neonato \textsuperscript{(28,29)}.

Gli effetti della partoanalgesia in termini di durata ed esito del
travaglio sono stati oggetto di numerosi studi, tuttavia non è stato
chiaramente definito se e in che modo essa possa influenzare l'inizio
dell'allattamento al seno e la sua prosecuzione nel periodo post-partum.
Gli effetti sull'allattamento potrebbero essere correlati all'influenza
che la partoanalgesia può avere sulla durata del travaglio o sul tipo di
parto (strumentale o no) oppure correlati alla dosi totali dei farmaci
somministrati durante tutto il travaglio e al successivo innalzamento
della concentrazione di anestetico locale/oppioide nel sangue venoso
cordonale con ipotetica compromissione dei comportamenti neonatali
necessari all'assunzione del latte materno.

In alcuni studi la partoanalgesia riduce la probabilità di intraprendere
e mantenere nel lungo periodo l'allattamento al seno, in altri non è
dimostrata tale associazione. La presenza di difetti metodologici
potrebbe spiegare i risultati contrastanti raccolti fino ad oggi; infine
in alcuni casi l'analisi dei dati sull'allattamento è ricavata da studi
clinici che non hanno come outcome primario l'allattamento stesso
\textsuperscript{(30,31)}.

Il presente studio ha lo scopo di indagare se la partoanalgesia condotta
per via epidurale è in grado di influenzare l'inizio dell'allattamento e
la sua prosecuzione in maniera esclusiva a distanza di 3 e 5 mesi dal
parto.

\emph{\textbf{Bibliografia:}}

Ip S. at al. Breastfeeding and maternal and infant health outcomes in
developed countries. \emph{Evid Rep Technol Assess 2007; 153: 1-186.}

Sullivan S. et al. An exclusively human milk-based diet is associated
with a lower rate of necrotizing enterocolitis than a diet of human milk
and bovine milk-based products. \emph{J Pediatr 2010; 156 (4): 562-567.}

Ip S. et al. A summary of the Agency for Healthcare Research and
Quality's evidence report on breastfeeding in developed countries.
\emph{Breastfeed Med 2009; 4 (supp1.1): S17-S30.}

Szajewska H. Early nutritional strategies for preventing allergie
disease. \emph{Isr Med Assoc J 2012; 14:58-62.}

Eidelman Al. Breastfeeding and the use of human milk: an analysis of the
American Academy of Pediatrics 2012 Breastfeeding Policy Statement.
\emph{Breastfeed Med 2012; 7(5): 323-324.}

World Health Organization WHO - Baby Friendly Hospital Initiative;
revised, updated and expanded for integrated care guidelines, 2009.\\
http:ilwww.who.intinutrition/publicationsfinfantfeedinglbfhi
trainingcourse/en

Eidelman Al. Breastfeeding and the use of human milk: an analysis of the
American Academy of Pediatrics 2012 Breastfeeding Policy Statement.
\emph{Breastfeed Med 2012; 7(5): 323-324.}

Committee on Health Care for Underserved Women, American College of
Obstetricians and Gynecologists. ACOG Committee Opinion No. 361:
Breastfeeding: maternal and infant aspects. \emph{Obstet Gynecol 2007;
109: (2, pt 1): 479-480.}

Davanzo R. et al. Position Statement on Breastfeeding from the Italian
Pediatric Societíes. \emph{Ital J Pediatr 2015; 41:80.}

Newton NR. Newton M. Relationship of ability to breast feed and maternal
attitudes toward breast feeding. \emph{Pediatrics 5: 869-875, 1950.}

Hàgerdal M. et al. Minute ventilation and oxygen consumption during
labor with epidural analgesia. \emph{Anesthesiology 1983; 59 425-7.}

Motoyama EK et al. Adverse effect of maternal hyperventilation on the
foetus. \emph{Lancet 1966; 1:286-8}

Westgren M. et al. Maternal and fetal endocrine stress response at
vaginal delivery with and without an epidural block. \emph{J Perinat
Alea. 1986, 14(4).235-41.}

Baumgarder DJ et al. Effect of labor epidural anesthesia on
breast-feeding of hcahhy full-term newhorns delivered vaginally. \emph{J
Am Board Fam Prua 2003 Jan-Feb; 16(1):7- 13.}

Wiklund I et al. Epidural analgesia: breast-feeding success and related
factors. \emph{Midwifery. 2009 Apr, 25(2) e31-8.}

Henderson JJ et al. Impact of intrapartum epidural analgesia on
breast-feeding duration. Aust N Z.1 \emph{Obstet Gynaecol. 2003
Oct;43(5). 372-7.}

Loftus JR et al. Placenta transfer and neonatal effects of epidural
sufentanil and fentanyl administcred with bupivacaine during labor.
\emph{Anesihesiology 1995 Aug; 83(2):300-8.}

Beifin Y et al. Eftect of labor epidural analgesia with and without
fentanyl on infant breast-feeding: a prospective, randomized,
double-blind study. \emph{Anesthesiology 2005 Dec:103(6): 1211- 7}

Radzyminsky S. Neurobehavioural functioning and breastfeeding behaviour
in the newborn \emph{{[}J Obstet Gynecol Neonatal Nurs 2005;
34:335-41{]}}

Chang ZM, Heaman MI. Epidural analgesia during labor and delivery:
effects on the initiation and continuation of effective breastfeeding.
\emph{J Human Lact 2005; 21(3): 305-314.}

Davis HV, Sears RR. Effects of cup, bottle and breast feeding on oral
activities of newborn infants. \emph{Pediatrics 1948 Nov; 2(5): 549-58.}

Kutski JK. Effects of bromocriptine mesylate on the composition of the
marnmary secretion in non-breast-feeding women. \emph{Ofrsiet Gynecol.
1978 Jul; 52(1) 38-42.}

Chapman Di. Identification of risk factors for delayed onset of
lactation. \emph{J Am Met Associ 1999: 99: 450-454.}

Chen DC et al. Stress during labor and delivery and early lactation
performance. \emph{Am J Clin Nutr. 1998 Aug ; 68(2):335-44. }

Neville MC, Morton J. Physiology and endocrine changes underlying human
lactogenesis II. \emph{J Nutr. 2001 Nov; 131(11): 3005S-8S. }

Radzyminski S. The effect of ultra low dose epidural analgesia on
newborn breastfeeding behaviors. \emph{J Obstet Gynecol Neonatal Nurs.
2003 May-Jun; 32(3):322-31. }

Wieczorek PM et al. Breastfeeding success rate after vaginal delivery
can be high despite the use of epidural fentanyl: an observational
cohort study. \emph{Int J Obstet Anesth 2010; 19:273-7. }

Wilson MJA et al. Epidural analgesia and breastfeeding: a randomised
controlled trial of epidural techniques with and without fentanyl and a
non-epidural comparison group. \emph{Anaesthesia 2010; 65: 145-153;}

Porter J. et al. Effect of epidural fentanyl on neonatal respiration.
\emph{Anesthesiologv 1998; 89: 79-85. }

Ashley L., Szabo MD. Intrapartum neuraxial analgeia and breastfeeding
outcomes: limitations of current knowledge. \emph{Anesthesia Analgesia
2013;116:399-405. }

Cynthia A. French et al. Labor epidural analgesia and breastfeeding: a
systematic review. \emph{Journal of Human Lactation 2016:32(3):507-520}.

\hypertarget{allattamento-al-seno}{%
\section{ALLATTAMENTO AL SENO}\label{allattamento-al-seno}}

\hypertarget{le-raccomandazioni}{%
\subsection{Le raccomandazioni}\label{le-raccomandazioni}}

È stato dimostrato che l'allattamento al seno comporta numerosi vantaggi
in termini di salute sia per il neonato che per la madre. Il latte
materno rappresenta il miglior alimento per i neonati, perché fornisce
tutti i nutrienti di cui hanno bisogno nella prima fase della loro vita,
come alcuni acidi grassi polinsaturi, proteine e ferro assimilabile.

Non è un semplice nutrimento, ma un tessuto vivo che si modifica sulla
base delle esigenze del singolo bambino. La promozione dell'allattamento
al seno è considerata da tempo una priorità di salute pubblica tale da
essere espressamente indicata dall'United Nations International
Children's Emergency Fund (UNICEF) come un diritto nell'articolo 24
della Convenzione per i Diritti per l'Infanzia del
1989\textsuperscript{(1)}. Nel corso degli anni diversi interventi sono
stati messi a punto per sostenere le donne nell'allattamento. OMS e
UNICEF hanno pubblicato nel 1989 un documento conosciuto come ``I 10
passi per il successo dell'allattamento al seno''; questi sono poi
diventati parte dell'iniziativa Baby-friendly Hospital nel 1991,
aggiornata nel 2009 \textsuperscript{(2)}.

Nel 2017 sempre l'OMS pubblica il documento ``GUIDELINE Protecting,
promoting and supporting BREASTFEEDING IN FACILITIES providing matemity
and newborn services'', una linea guida sull'allattamento rivolta a un
ampio pubblico: politici, staff di istituzioni nazionali e
organizzazioni che lavorano nell'ambito dei programmi di nutrizione
nell'infanzia, professionisti sanitari, clinici e universitari.

Il documento esamina tre raccomandazioni evidence-informed sulla
protezione, la promozione e il supporto all'allattamento nelle strutture
materno-infantili:

\begin{itemize}
\item
  supporto immediato per l'inizio e l'avvio dell'allattamento con
  raccomandazioni relative al contatto pelle a pelle, all'inizio precoce
  dell'allattamento al seno, al rooming-in etc.
\item
  pratiche di alimentazione e bisogni aggiunti del bambino con
  raccomandazioni relative all'alimentazione supplementare, all'uso di
  ciucci e biberon etc.
\item
  creazione di un ambiente di supporto con raccomandazioni relative alle
  politiche sanitarie in tema di promozione dell'allattamento al seno
  nelle strutture materno-infantili, alla formazione degli operatori
  sanitari, alla promozione prenatale dell'allattamento al seno per le
  madri etc. \textsuperscript{(3) }
\end{itemize}

L'Organizzazione Mondiale della Sanità raccomanda:

\begin{itemize}
\item
  allattamento esclusivo al seno per i primi sei mesi di vita
\item
  introduzione di cibi complementari adeguati e sani (solidi) a 6 mesi
  continuando ad allattare sino a 2 anni di età o oltre.
\end{itemize}

Da realizzarsi attraverso azioni di supporto immediato per l'inizio e
l'avvio dell'allattamento:

\begin{itemize}
\item
  il precoce e continuo skín to skin tra madre e neonato dovrebbe essere
  facilitato e incoraggiato prima possibile dopo il parto (qualità
  dell'evidenza moderata)
\item
  tutte le madri dovrebbero essere supportate ad iniziare l'allattamento
  prima possibile, entro la prima ora dopo il parto (qualità
  dell'evidenza alta)
\item
  le mamme dovrebbero ricevere supporto pratico per consentire loro
  l'inizio e l'avvio dell'allattamento e gestire le più comuni
  difficoltà legate ad esso (qualità dell'evidenza moderata)
\item
  le madri dovrebbero essere istruite su come spremere il latte materno
  per mantenere la lattazione nei casi in cui sono separati dal loro
  neonato (qualità dell'evidenza molto bassa)
\item
  le strutture materno-infantili dovrebbero permettere a madre e neonato
  di restare insieme praticando il rooming in giorno e notte. Questo
  potrebbe essere non `applicabile' nelle circostanze in cui il neonato
  necessita di cure mediche specializzate (qualità dell'evidenza
  moderata)
\item
  le madri dovrebbero essere supportate a praticare un'alimentazione
  'reattiva' come parte di assistenza alla nutrizione (qualità
  dell'evidenza molto bassa) \textsuperscript{(2)}.
\end{itemize}

Nel 2012 l'American Academy of Pediatrics pubblica il documento
``Breastfeeding and the Use of Human Milk: An Analysis of the American
Academy of Pediatrics 2012 Breastfeeding Policy
Statement"\textsuperscript{(4)}, mentre 1'American College of Obstetrics
and Gynecologist pubblica nell'ottobre 2016 il documento "Primary Care
Interventions to Support Breastfeeding US Preventive Services Task Force
Recommendation Statement" all'interno dei quali entrambi raccomandano
l'allattamento al seno per i primi 6 mesi di vita del bambino e lo
consigliano fino all'anno di età qualora sia possibile
\textsuperscript{(5)}.

In Italia viene pubblicato il documento ``Allattamento al seno e uso del
latte materno/umano Position Statement 2015'' da parte della Società
Italiana di Pediatria (SIP), la Società Italiana di Neonatologia (SIN),
la Società Italiana delle Cure Primarie Pediatriche (SICuPP), la Società
Italiana di Gastroenterologia Epatologia e Nutrizione Pediatrica
(SIGENP) e la Società Italiana di Medicina Perinatale (SIMP) con le
medesime raccomandazioni. \textsuperscript{(6) }

\hypertarget{dati-epidemiologici}{%
\subsection{Dati epidemiologici}\label{dati-epidemiologici}}

Al 1 agosto 2017, secondo l'ultimo rapporto redatto da UNICEF e OMS in
collaborazione con il Global Breastfeeding Collective, la nuova
iniziativa internazionale che mira ad ampliare i tassi globali di
allattamento, nessuno Stato al mondo ha raggiunto pienamente gli
standard raccomandati sull'allattamento al seno. La scheda "Global
Breastfeeding Scorecard", che analizza i dati di 194 Stati, mostra come
solo il 40\% dei bambini tra 0 e 6 mesi venga allattato esclusivamente
con latte materno - come prescrivono OMS e UNICEF - e come in appena 23
Stati il tasso di allattamento al seno superi il 60\%. Neì Paesi
industrializzati, i tassi di allattamento al seno sono in generale bassi
sia per quanto riguarda l'allattamento al seno esclusivo nei primi 6
mesi di vita, sia per il suo proseguimento tino al secondo anno di età,
con l'eccezione dei Paesi scandinavi. \textsuperscript{(7)}

Manca attualmente in Italia un sistema di monitoraggio validato e
periodico sull'alimentazione infantile, che consenta di avere dati
accreditabili sulla diffusione dell'avvio e della durata
dell'allattamento al seno, in particolar modo di quello esclusivo.
Questa situazione è confermata dai risultati della Survey 2014
sull'Allattamento al Seno in Italia promossa dal Tavolo Tecnico
Operativo Interdisciplinare sulla Promozione dell'Allattamento al Seno
(TAS) del Ministero della Salute. Il sistema di monitoraggio dovrebbe
avere base regionale e usare definizioni condivise, riproducibili,
precise, quali quelle proposte dall'OMS. Un sistema di monitoraggio è
opportuno per confronto trasversale e longitudinale, anche per la
valutazione di eventuali appropriati interventi di promozione
dell'allattamento al seno. Il TAS ha pertanto avanzato la proposta di
una raccolta dati, che riguardi tutti i Punti Nascita italiani ed i
centri di prevenzione, utilizzando appunto le definizioni
sull'alimentazione infantile dell'OMS. Il TAS raccomanda di raccogliere
i tassi di allattamento al seno alla dimissione dall'ospedale, in
occasione della prima e seconda vaccinazione. In assenza di un
preferibile monitoraggio universale effettuato in occasione della prima
e seconda vaccinazione possono in alternativa risultare utili dati
epidemiologici raccolti mediante progetti di ricerca strutturati sul
territorio.

In base ai limitati dati attualmente disponibili si può stimare che nei
primi giorni di vita cominci ad allattare al seno (anche se in maniera
non esclusiva) oltre il 90 \% delle donne italiane, ma giunga ad
allattare esclusivamente al seno alla dimissione dall'ospedale il 77 \%,
a 4 mesi il 31 \% ed a 6 mesi di vita solamente il 10 \%.

I tassi di allattamento per le donne impossibilitate a ottenere
un'assistenza sanitaria di qualità a causa del basso livello economico,
sociale e culturale, etnia, o regione geografica di appartenenza possono
essere più bassi, in conseguenza delle barriere ancora maggiori che
queste donne incontrano nell'avviare e continuare l'allattamento al
seno.

Anche in Italia le differenze socioeconomiche e geografiche condizionano
l'accettazione e la prosecuzione dell'allattamento al seno da parte
delle madri; infatti allatta meno quella parte della popolazione
nazionale con livelli di istruzione e socioeconomico inferiori e quella
residente nelle regioni meridionali.

Questa variabilità nell'ambito delle realtà regionali è verosimilmente,
almeno in parte, in relazione ad un diverso impegno nei programmi di
promozione dell'allattamento materno, compresa la formazione del
personale delle strutture sanitarie.\textsuperscript{(8)} Anche
l'implementazione della Baby Friendly Hospital Initiative (BFHI) e della
Baby Friendly Community Initiative (BFCI) presenta attualmente un
gradiente Nord-Sud, con le strutture certificate come amiche del bambino
concentrate nell'Italia Centro Settentrionale. \textsuperscript{(9) }

\hypertarget{benefici-derivanti-dallallattamento-al-seno-per-il-bambino}{%
\subsection{Benefici derivanti dall'allattamento al seno per il
bambino}\label{benefici-derivanti-dallallattamento-al-seno-per-il-bambino}}

L'American Acaderny of Pediatrics nel 2012 pubblica una Policy Statement
intitolata "Breastfeeding and the use of human milk'' in cui mette in
relazione la durata dell'allattamento e i benefici sulla salute
neonatale:

\begin{itemize}
\item
  \begin{quote}
  \emph{Infezioni del tratto respiratorio e otite media}
  \end{quote}
\end{itemize}

\begin{quote}
Il rischio di ospedalizzazione per infezioni del basso tratto
respiratorio durante il primo anno di vita è ridotto del 72\% se il
neonato è allattato esclusivamente al seno per più di 4 mesi. I bambini
allattati esclusivamente al seno per 4-6 mesi hanno un rischio aumentato
di 4 volte di sviluppare polmonite rispetto a quelli allattati
esclusivamente al seno per più di 6 mesi. La severità (data dalla durata
della ospedalizzazione e dalla richiesta di ossigeno) della bronchiolite
dovuta al VRS è ridotta del 74\% nei bambini che sono stati allattati
esclusivamente al seno per almeno 4 mesi se paragonati a bambini mai
allattati o parzialmente allattati al seno. Anche l'incidenza di otite
media è ridotta del 23\% nei bambini allattati esclusivamente al seno se
paragonati a quelli allattati con latte ridotta di formula.
L'allattamento esclusivo al seno per più di 3 mesi riduce il rischio di
otite media del 50\% . Infezioni severe delle alte vie respiratorie sono
ridotte del 63\% nei bambini allattati esclusivamente al seno per almeno
6 mesi.
\end{quote}

\begin{itemize}
\item
  \emph{Infezioni del tratto gastrointestinale}
\end{itemize}

\begin{quote}
Ogni tipo di allattamento al seno è associato ad una riduzione del 64\%
dell'incidenza di infezioni del tratto gastrointestinale non specifiche,
e questo effetto permane per 2 mesi dopo la cessazione
dell'allattamento.
\end{quote}

\begin{itemize}
\item
  \emph{Enterocolite necrotizzante}
\end{itemize}

\begin{quote}
Una meta-analisi di 4 trials clinici randomizzati realizzati tra il 1983
e il 2005 supportano la conclusione che nutrire i neonati pretermine con
latte umano è associato ad una riduzione significativa (58\%)
dell'incidenza di NEC. Uno studio più recente mostra una riduzione del
77\% dell'incidenza.
\end{quote}

\begin{itemize}
\item
  \emph{SIDS e mortalità infantile}
\end{itemize}

\begin{quote}
Meta-analisi hanno messo in evidenza che l'allattamento al seno porta a
riduzione del rischio di SIDS (Sudden Infant Death Syndrome) del 36\%. È
stato calcolato che più di 900 bambini l'anno negli Stati Uniti sono
stati salvati dalla SIDS se il 90\% delle madri hanno allattato
esclusivamente al seno per almeno 6 mesi. Nei 42 Paesi in via di
sviluppo in cui avviene il 90\% delle morti infantili, l'allattamento
esclusivo al seno per almeno 6 mesi e svezzamento dopo l'anno di vita è
stato il più importante intervento di prevenzione della SIDS (1 milione
di bambini per anno), equivalente ad una prevenzione di mortalità
infantile nei mondo del 13\%.
\end{quote}

\begin{itemize}
\item
  \begin{quote}
  \emph{Allergie }
  \end{quote}
\end{itemize}

\begin{quote}
C'è un effetto protettivo dell'allattamento esclusivo per 3-4 mesi nella
riduzione di incidenza di asma, dermatite atopica ed eczema del 27\% in
una popolazione a basso rischio e del 42\% in bambini con storia
familiare positiva.

Il latte materno è un liquido immunologicamente attivo capace di dare al
sistema immunitario del bambino allattato segnali di segno opposto, sia
in termini di "sedazione" che di stimolazione della risposta allergica.
I meccanismi non sono del tutto noti, ma questo effetto contrastante
potrebbe almeno in parte spiegare perché gli studi portino a risultati
controversi, da una protezione dell'allattamento al seno sulle malattie
allergiche, a un effetto nullo, fino all'incremento delle allergie. E'
verosimile che ciò sia dovuto alla combinazione di più fattori, quali la
costituzione atopica, il livello socio-economico, i diversi livelli di
inquinamento ambientale, le modalità di alimentazione ed in particolare
di allattamento. Sembra che l'effetto documentato da alcuni studi di un
aumento dell'allergia nei bambini allattati al seno, sia dovuto a una
causalità inversa, ossia al fatto che quei bambini che manifestano segni
precoci di allergia o appartengono a famiglie atopiche in cui la
motivazione ad allattare risulti maggiore possano essere di fatto più
frequentemente allattati al seno. La meta-analisi di Lodge mostra come
l'allattamento al seno protegga in misura diversa dall'asma (fra i 5 ed
i 18 anni), eczema (\textless{}= 2 anni), rinite allergica (\textless{}=
5 anni), ma non dalle allergie alimentari. In definitiva, l'allattamento
al seno, in particolare quello esclusivo, viene incluso a pieno titolo
nella strategia di prevenzione delle malattie allergiche.
\end{quote}

\begin{itemize}
\item
  \begin{quote}
  \emph{Celiachia }
  \end{quote}
\end{itemize}

\begin{quote}
C'è una riduzione del 52\% del rischio di sviluppare la malattia celiaca
in bambini che sono stati allattati durante il periodo di esposizione al
glutine. Nel complesso c'è un'associazione tra l'aumentata durata
dell'allattamento e la riduzione del rischio di malattia celiaca quando
misurato come presenza di anticorpi per la celiachia.
\end{quote}

\begin{itemize}
\item
  \begin{quote}
  \emph{Malattie infiammatorie croniche intestinali }
  \end{quote}
\end{itemize}

\begin{quote}
L'allattamento al seno è associato con una riduzione del 31\% del
rischio di sviluppare IBD. L'effetto protettivo si ipotizza derivi da
una interazione tra l'effetto immunomodulante del latte umano e la
sottostante suscettibilità genetica del bambino. Differenti patterns di
colonizzazione batterica nei bambini allattati al seno rispetto a quelli
allattati con latte di formula possono determinare effetti protettivi.
\end{quote}

\begin{itemize}
\item
  \begin{quote}
  \emph{Obesità}
  \end{quote}
\end{itemize}

\begin{quote}
I tassi di obesità sono significativamente più bassi nei bambini
allattati al seno: dal 15\% al 30\% di riduzione del tasso di obesità
durante l'adolescenza e in età adulta. Il feto interagendo con
l'ambiente metabolico materno indirizza l'accrescimento tissutale e la
maturazione funzionale degli organi, con importanti riflessi a lungo
termine sulle successive capacità adattative alle situazioni ambientali
in cui l'organismo si verrà a trovare ("programming metabolico"). Questo
processo di maturazione anatomica e funzionale continua dopo la nascita
e si protrae nei primi anni di vita del bambino. I meccanismi
molecolari, che vengono a costruire questa sorta di "memoria"
metabolica, sono principalmente di carattere epigenetico. La nutrizione
delle prime età, e in primis l'allattamento naturale, svolge quindi un
ruolo di assoluto rilievo nel modulare in modo fisiologico
l'acquisizione/selezione di ``percorsi'' ottimali del metabolismo del
bambino, che tendono a persistere nel tempo, con evidenti benefici a
lungo termine. La composizione corporea del bambino alimentato con
formule lattee presenta innanzitutto caratteristiche diverse dal bambino
di riferimento normale, che è quello allattato al seno. In particolare,
l'allattato al seno ha nei primi 4 mesi di vita una massa adiposa
maggiore dell'allattato con formula. Successivamente avviene il sorpasso
e l'allattato con formula risulta più grasso dell'allattato al seno.
L'aumento della massa adiposa tende poi a persistere e a favorire la
comparsa dell'obesità. Il bambino allattato al seno ha un rischio di
sviluppare obesità inferiore del bambino non allattato, ma l'entità di
questo effetto è modesta. Anche la durata dell'allattamento al seno
esclusivo è importante. In particolare, lo studio IDEFIX condotto in 8
paesi europei ha evidenziato che l'allattamento esclusivo al seno fino
ai 6 mesi di vita è protettivo nei confronti dello sviluppo dell'obesità
fra i 2 ed i 9 anni di età. L'entità di questo effetto protettivo si
riduce col tempo, per il sommarsi di altri fattori di rischio per
l'obesità, che tendono ad accumularsi. L'effetto protettivo nei
confronti dell'obesità pare legato principalmente alla diversa
composizione del latte adattato rispetto a quello umano, in particolare
alla quota proteica, più elevata nella formula. In aggiunta il latte di
donna contiene, a differenza della formula, ormoni quali leptina,
insulina, GLP-1 (Glucagon-Like Peptide-1), peptide gastrointestinale YY
(PYY) e adiponectina, direttamente coinvolti nei complessi meccanismi di
regolazione metabolica e di equilibrio fame/sazietà. Il ruolo di questi
ormoni assunti con il latte potrebbe essere quello di contribuire a
ridurre il rischio di sviluppare eccesso ponderale nel lattante e anche
nelle età successive.

Anche le modalità proprie dell'allattamento col biberon rispetto al seno
giocano un ruolo importante nei confronti della sazietà, così come anche
la composizione della flora batterica intestinale del lattante. Sono
state ben documentate rilevanti differenze nella composizione della
flora batterica del bambino nato da parto vaginale o da cesareo, da
madre obesa o normopeso, allattato al seno o con formula. L'importante
coinvolgimento diretto della flora batterica intestinale nella
modulazione di alcuni processi metabolici, tra cui il metabolismo degli
acidi grassi e, di conseguenza. l'insulino-sensibilità, apre un nuovo
scenario nella relazione tra alimentazione delle prime età e rischio
metabolico nelle età successive.
\end{quote}

\begin{itemize}
\item
  \begin{quote}
  \emph{Diabete }
  \end{quote}
\end{itemize}

\begin{quote}
Una riduzione superiore al 30\% nell'incidenza di diabete mellito di
tipo I è riportata per i bambini che sono stati allattati esclusivamente
al seno per almeno 3 mesi, evitando l'esposizione a proteine del latte
bovino. Il meccanismo ipotizzato è quello che i bambini esposti alle
proteine del latte bovino, in particolare alla beta-lactoglobulina
sviluppino un processo immuno-mediato che porti a cross-reazione con le
cellule beta pancreatiche. Anche una riduzione di incidenza del diabete
mellito di tipo II è riportata e probabilmente riflette un positivo
effetto a lungo termine dell'allattamento al seno sul controllo del peso
corporeo e sulla capacità di autoregolazione dell'alimentazione.
\end{quote}

\begin{itemize}
\item
  \begin{quote}
  \emph{Leucemia infantile e linfoma }
  \end{quote}
\end{itemize}

\begin{quote}
C'è una riduzione nel rischio di sviluppare leucemia correlata all'
allattamento al seno. Una riduzione del 20\% del rischio di leucemia
linfatica acuta e del 15\% del rischio di sviluppare leucemia mieloide
acuta nei bambini allattati al seno per almeno 6 mesi o più. Il
meccanismo che determina questa riduzione è ancora sconosciuto.
\end{quote}

\begin{itemize}
\item
  \begin{quote}
  \emph{Sviluppo neurocognitivo }
  \end{quote}
\end{itemize}

\begin{quote}
Sono state riportate differenze sostanziali nello sviluppo
neurocognitivo tra bambini allattati esclusivamente al seno e bambini
allattati con latte di formula, ma esistono numerosi fattori di
confondimento che possono aver influenzato l'outcome, come differenze
nell'educazione da parte dei genitori, intelligenza, ambiente familiare,
status socioeconomico. \textsuperscript{(10) }
\end{quote}

\hypertarget{durata-ottimale-dellallattamento}{%
\subsection{Durata ottimale
dell'allattamento}\label{durata-ottimale-dellallattamento}}

Con l'obiettivo di ottenere crescita, sviluppo e salute ottimali, e
sulla base di una revisione sistematica della letteratura, 1'OMS
raccomanda l'allattamento al seno esclusivo per 6 mesi.

L'adesione internazionale a questa raccomandazione da parte delle
società scientifiche e degli enti governativi nazionali. specialmente in
Europa, è risultata incostante.

Si è determinato un generale dibattito fra l'opzione 4-6 mesi e
l'opzione 6 mesi. L'opzione 4-6 mesi è stata sostenuta sinteticamente
per i seguenti motivi:

a) le ridotte evidenze scientifiche di una protezione contro l'allergia
oltre i 4 mesi di allattamento esclusivo,

b) l'opportunità di non introdurre tardivamente gli alimenti diversi dal
latte materno per sfruttare il periodo di copertura immunologica fornita
dal latte materno contro nuovi antigeni,

c) il possibile calo di produzione del latte materno con rischio di
relativa malnutrizione,

d) la possibile sideropenia.

Nessuna di queste motivazioni tuttavia è sufficiente per riproporre il
ritorno alla vecchia raccomandazione 4-6 mesi.

Bisogna infatti considerare che una raccomandazione tutela la salute
della maggioranza della popolazione, ma comprensibilmente non di tutta
la popolazione. Come sempre però è possibile passare dalla
raccomandazione generale al consiglio individuale, che potrà risultare
differente dalla raccomandazione. Sta al pediatra. quindi, seguire la
crescita del singolo bambino e cogliere eventualmente la necessità di
attuare interventi di correzione della gestione dell'allattamento o di
rimandare mamma e bambino a figure specializzate e/o di sostegno, in
modo da garantire la crescita ottimale fino al momento
dell'alimentazione complementare. Spetta sempre al pediatra identificare
il momento in cui il latte materno eventualmente non basti più e non ci
sia ormai spazio per recuperarne la produzione dando consigli specifici
e competenti.

La crescita del bambino allattato al seno in maniera esclusiva va però
correttamente valutata, perchè è diversa da quella del bambino allattato
in maniera mista o con formula lattea. Il riferimento di crescita
adeguato è rappresentato oggi delle Growth Chart dell'OMS; la loro
utilità ai fini della promozione dell'allattamento al seno esclusivo si
colloca in particolare fra i 3 ed i 6 mesi di vita. Mentre nei primi 3
mesi di vita la crescita minima settimanale risulta di 150-200
gr/settimana, quella dai 4 ai 6 mesi risulta di soli 100-150
gr/settimana ( l 30).

L'obiettivo di migliorare la tolleranza all'introduzione dei vari
antigeni è in definitiva compatibile con un'alimentazione complementare
(svezzamento) avviata alla fine del 6° mese di vita, anche se le
indicazioni delle varie società scientifiche sono espresse in termini
diversi.

L'ESPGHAN, con visione prevalentemente gastroenterologica, suggerisce di
introdurre alimenti diversi dal latte materno non prima dei 4 mesi e non
dopo il 6° mese compiuto, pur mantenendo i 6 mesi ``a desirable goal".

La Canadian Society of Pediatrics indica 6 mesi.

L'American Academy of Pediatrics identifica il momento giusto con un
pragmatico ``all'incirca al 6° mese", che tiene conto non solo delle
istanze allergologiche e gastroenterologiche, ma anche del documentato
rapporto dose-effetto fra allattamento al seno esclusivo e beneficio di
salute per il bambino (riduzione del rischio di obesità) e per sua madre
(riduzione del rischio di cancro al seno).

L'Autorità Europea della Sicurezza Alimentare (EFSA) si allinea su una
formula di compromesso che spinge a sostenere ove possibile il goal dei
6 mesi: "\emph{Exclusive breast-feeding provides adeguate nutrition up
to 6 months of age for the majority of infants, while some infants may
need complementary foods before 6 months (but not before 4 months) in
addition to breast-feeding in order to support optimal growth and
development}".

Per quanto riguarda il ferro, alcuni autori hanno di recente suggerito
l'utilità di anticipare lo svezzamento a partire dai 4 mesi per fornire
cibi più ricchi di ferro rispetto al latte materno. In realtà la
revisione di Qasem è stata fatta sui soli 3 RCT disponibili, condotti su
piccoli campioni e con breve follow-up. Bisogna piuttosto considerare
che i nati a termine, allattati esclusivamente al seno, solitamente
mantengono normali scorte di ferro per 6 mesi e non richiedono una
supplementazione marziale. Nelle categorie a rischio di sideropenia (per
esempio bambini con basso peso alla nascita o figli di madre diabetica)
si attuerà una supplementazione marziale individualizzata, piuttosto che
generalizzare l'introduzione anticipata degli alimenti diversi dal latte
materno. Da ultimo non va dimenticato che la decisione di avviare
l'alimentazione complementare non può ignorare la valutazione dello
sviluppo psicomotorio del bambino. L'alimentazione complementare infatti
presuppone un bambino competente, che stia seduto, pronto alla
manipolazione, masticazione e deglutizione del cibo, tappe che riteniamo
siano solitamente raggiunte a 6 mesi. \textsuperscript{(1)}

\emph{\textbf{Bibliografia:}}

\begin{enumerate}
\def\labelenumi{\arabic{enumi}.}
\item
  United Nations International Children's Emergency Fund (UNICEF)
  (\url{http://www.unicefit/doc/148/ospedali-amici-deibambini.html})
\item
  World Health Organization -WHO - Baby Friendly Hospital Initiative;
  revised, updated and expanded for integrated care guidelines, 2009.
  \href{http://www.who.int/nutrition/publications/infantfeeding/bfhi\%2520trainingcourse/en/}{http://www.who.int/nutrition/publications/infantfeeding/bfhi
  trainingcourse/en/}
\item
  \emph{Protecting, promoting and supporting breastfeeding in facilities
  providing maternity and newborn services} Guideline Authors: World
  Health Organization. Publication details Number of pages: 120
  Publication date: 2017 ; Languages: English ISBN: 978 92 4 155008 6
\item
  Pediatrics March 2012, VOLUME 129 / ISSUE 3 From the American Academy
  of Pediatrics Policy Statement Breastfeeding and the Use of Human Milk
\item
  JAMA. 2016 Oct 25;316(16):1688-1693. doi: 10.1001/jama.2016.14697.
  Primary Care Interventions to Support Breastfeeding: US Preventive
  Services Task Force Recommendation Statement. US Preventive Services
  Task Force, Bibbins-Domingo Kl, Grossman DC2, Curry SJ3, Davidson KW4,
  Epling JW Jr5, Garcia FA6,Kemper AR7, Krist AH8, Kurth AE9, Landefeld
  CS 10, Mangione CM11, Phillips WR12, Phipps MG13, Pignone MP14.
\item
  Allattamento al seno e use del latte materno/umano Position Statement
  2015 di Society Italiana di Pediatria (SIP), Society Italiana di
  Neonatologia (SIN), Society Italiana delle Cure Primarie Pediatriche
  (SICuPP), Society Italiana di Gastroenterologia Epatologia e
  Nutrizione Pediatrica (SIGENP) e Society Italiana di Medicina
  Perinatale (SIMP) 1,10 Riccardo Davanzo, 2,10Claudio Maffeis,
  3,10Marco Silano, 4 Enrico Bertino, 5 Carlo Agostoni 6 Teresa Cazzato,
  4 Paola Tonetto, 7 Annamaria Staiano, 8 Renato Vitiello, 9 Fabio
  Natale Gruppo di Lavoro ad hoc di SW, SIN, SICuPP, SIGENP e SIMP.
  Documento condiviso dal TAS istituito presso it Ministero della Salute
  nella riunione del 15 settembre 2015.
  \url{http://www.saiute.gov.it/imgs/C17_pubblicazioni_2415_allegato.pdf}
\item
  Tracking progress for breastfeeding policies and programmes: Global
  breastfeeding scorecard 2017
  \url{http://wwwwho.intinutrition/publications/infantfeedingiglobal-bf-scorecard-2017/en/}
\item
  Ministero della Salute DIREZIONE GENERALE PER L'IGIENE E LA SICUREZZA
  DEGLI AL1MENTI E LA NUTRIZIONE UFFICIO v ALLATTAMENTO al SENO nelle
  strutture sanitarie in Italia Report sulla SURVEY NAZIONALE 2014 A
  cura del Tavolo Tecnico Operativo interdisciplinare sulla Promozione
  dell'Allattamento al Seno Report del 10 dicembre 2014; revisione
  dell'l PDJJLR 201
\item
  Ospedali \& Comunita amici dei bambini e delle bambine.
  \url{http://www.unicef.itidoc/148/ospedali-amici-dei-bambini.htm}
\item
  POLICY STATEMENT Breastfeeding and the Use of Human Milk.
  \url{http://pediatrics.aappublications.org/content/pediatrics/early/2012/02/22/peds.2011-3552.full.pdf}
\end{enumerate}

\hypertarget{fisiologia-della-lattazione}{%
\section{FISIOLOGIA DELLA
LATTAZIONE}\label{fisiologia-della-lattazione}}

La fisiologia della lattazione comprende 4 fenomeni:

\begin{enumerate}
\def\labelenumi{\arabic{enumi}.}
\item
  \textbf{Mammogenesi} (sviluppo e preparazione della ghiandola
  mammaria), inizia già durante la vita intrauterina, prosegue con
  l'adolescenza e si completa con la gravidanza. Si realizza soprattutto
  per pazione degli estrogeni e del progesterone. È dovuta
  prevalentemente all'aumento del tessuto ghiandolare, in minima parte
  all'accrescimento di tessuto adiposo e connettivo.
\item
  \textbf{Lattogenesi} (produzione di latte), si divide in due stadi:
\end{enumerate}

\begin{itemize}
\item
  {lattogenesi I} (sviluppo in senso secretorio della ghiandola
  mammaria) che si verifica durante la gravidanza;
\item
  {lattogenesi II} (produzione di latte e secrezione dello stesso) che
  si verifica in genere tra il 3° e il 5° giorno di puerperio dalla
  secrezione del colostro. L'ormone lattogeno più importante è la
  Prolattina e l'inizio della secrezione lattea coincide con la caduta
  del tasso ematico di estrogeni e progesterone.
\end{itemize}

\begin{enumerate}
\def\labelenumi{\arabic{enumi}.}
\setcounter{enumi}{2}
\item
  \textbf{Galattopoiesi} (mantenimento della secrezione lattea), dipende
  soprattutto da un meccanismo riflesso avviato dalla suzione.
\item
  \textbf{Eiezione del latte}, dipendente anch'essa dallo stimolo della
  \textbf{suzione}, che determina l'immissione in circolo di
  \textbf{Ossitocina} che provoca la contrazione delle cellule
  mioepiteliali della mammella\textsuperscript{.(1)}
\end{enumerate}

\hypertarget{lo-sviluppo-della-mammella-mammogenesi}{%
\subsection{Lo sviluppo della mammella
(mammogenesi)}\label{lo-sviluppo-della-mammella-mammogenesi}}

La ghiandola mammaria attraversa diversi stadi evolutivi e funzionali,
caratterizzati da aspetti fisiologici definiti. La ghiandola mammaria
origina dall'ectoderma da cui, dall'VIII settimana di vita embrionale,
migrano piccole gemmazioni che, attraversando il sottostante mesenchima,
formano agglomerati compatti di cellule epiteliali privi di
organizzazione duttale. All'intemo di queste gemmazioni, dal II
trimestre di gravidanza, prende origine una rudimentale rete di
ramificazioni (15-20 rami ogni gemma), che rappresenta il futuro sistema
duttale. La differenziazione cellulare è resa possibile dai livelli
plasmatici della prolattina fetale, che subisce un incremento dal terzo
trimestre di gravidanza, per cui alcune cellule si differenziano in
cellule duttali, creando lo scheletro del primordiale sistema duttale,
mentre altre si indirizzano verso la linea secernente, diventando
cellule alveolari. Alla nascita si assiste ad una modesta attivita
secretoria dovuta soprattutto all'influenza delle alte concentrazioni
ormonali materne nel circolo fetale, testimonianza della maturazione
recettoriale della mammella fetale. Contemporaneamente si sviluppa il
mesenchima: una fitta rete di capillari avvolge elementi mioepiteliali,
adipociti, terminazioni nervose (sensitivo/motorie) e linfatici,
insinuandosi tra i nidi ghiandolari. Sia il sistema nervoso che quello
linfatico hanno origine cutanea. La mammella alla nascita, tuttavia, si
presenta come una struttura stromale fibrosa, nell'ambito della quale
sono disseminati piccoli dotti ripieni di materiale latteo. Dopo circa
una settimana l'attività secretoria cala (per esaurimento dei livelli
ormonali mateni nel circolo fetale), fino a giungere allo stato inattivo
tipico dell'infanzia.

Lo sviluppo e la crescita della mammella sono strettamente dipendenti
dalla concentrazione degli ormoni sessuali, ma non solo; sono coinvolti
anche numerosi altri ormoni, nonché fattori di crescita. La prolattina
riveste un ruolo importante nella differenziazione cellulare della
mammella primordiale; infatti, assieme ai glucocorticoidi, al
progesterone e agli estrogeni, partecipa alla sviluppo del sistema
alveolare e duttale. Sotto l'influenza della prolattina, durante la
prima meta della gravidanza, le cellule dell'epitelio alveolare
proliferano, acquistando successivamente la capacita secretoria. La
sintesi e la secrezione di prolattina avvengono a livello ipofisario, ma
sono controllate dall'ipotalamo, mediante rilascio di dopamina
(feed-back negativo), ormone inibente l'attività secretoria tonica delle
cellule lattotrope ipofisarie. La serotonina ipotalamica e l'estradiolo,
invece ne aumentano la secrezione. Agonisti della dopamina, come la
bromocriptina, mimando l'effetto del neurotrasmettitore sul sito
recettoriale, innescano una cascata di eventi che conduce ad inibizione
della secrezione di prolattina; gli antidepressivi (fenotiazinici,
butirrofenoni, benzamidi), determinano iperprolattinemia, perchè
bloccano i recettori per la dopamina e ne impediscono l'azione.

Nella mammogenesi un importante ruolo è sostenuto dall'estradiolo e dal
progesterone. L'estradiolo determina riduzione dei recettori per il
progesterone; il progesterone, a sua volta ed a determinate
concentrazioni, blocca l'attività secretoria alveolare, cosicché in
gravidanza non si verifica la produzione di latte. Sarà dopo la nascita,
con crollo dei livelli di progesterone, che l'inibizione verrà rimossa.

Oltre agli ormoni sessuali, la genesi e lo sviluppo mammari sono
influenzati dagli ormoni tiroidei (ci sono evidenze che l'ipotiroidismo
congenito ritardi lo sviluppo dei dotti ed alteri la mammogenesi), dal
fattore di crescita insulino simile-1, nonché dall'insulina stessa (per
rapporto energetico). Non si può dire lo stesso per l'ormone della
crescita (GH), poiché la sua assenza congenita non compromette la
capacità di allattare della donna. \textsuperscript{(2) }

\hypertarget{pubertuxe0-e-periodo-fertile-i-cambiamenti-associati-al-ciclo-mestruale}{%
\subsection{Pubertà e periodo fertile: i cambiamenti associati al ciclo
mestruale}\label{pubertuxe0-e-periodo-fertile-i-cambiamenti-associati-al-ciclo-mestruale}}

Durante la pubertà la mammella femminile subisce modificazioni
strutturali di rilievo, cui corrispondono modificazioni endocrine. Il
tessuto mammario si fa prominente, l'areola diventa più pigmentata e più
grande e ciò prosegue fine al menarca. Le mammelle diventano compatte,
più piene via via che si supera l'adolescenza fino al raggiungimento
dell'eta adulta, in cui caratteristicamente, l'aspetto macroscopico è
condizionato dalla corporatura della donna (donne con una quota di
grasso corporeo ben rappresentato hanno mammelle larghe e compatte).

Le modificazioni cicliche dei livelli ematici degli ormoni sessuali
durante il ciclo mestruale influenzano profondamente la morfologia della
ghiandola, sia per l'impatto che queste hanno sulle cellule, sia per
l'effetto sui vasi; ciò si ripercuote a livello macroscopico,
determinando senso di congestione mammaria, pienezza, nonché fine
nodularità, tipici del periodo premestruale.

Le cellule epiteliali della ghiandola mammaria sono dotate di recettori
(citoplasmatici e di membrana) per estrogeni e progestinici, attraverso
i quali si realizzano le modificazioni fisiologiche suddette. Durante la
fase follicolare si riscontrano i massimi livelli di estrogeni, sotto
stimolo di FSH e LH, per cui l'epitelio alveolare cresce e acquista
potenzialità secretoria, mentre durante la fase luteinica del ciclo il
prevalere dell'effetto dei progestinici determina dilatazione dei dotti
mammari e differenziazione definitive dell'epitelio in cellule
secernenti monostratificate. L'aumento dei livelli di estrogeni
circolanti può esercitare un effetto istamino-simile sulla
microcircolazione mammaria, con incremento del flusso sanguigno massimo
nel periodo premestruale. Il concomitante aumento della pressione
endocapillare a livello del microcircolo determina edema interlobulare,
che, assieme all'aumentata proliferazione duttulo-acinare, è
responsabile del turgore della mammella in questa fase. Con l'inizio
della mestruazione, in fase luteinica, i livelli circolanti di estrogeni
e progestinici calano rapidamente e a questo fa seguito una riduzione
dell'attività secretoria dell'epitelio, fino all'inizio di una nuova
fase follicolare\textsuperscript{. (2) }

\hypertarget{la-mammella-durante-la-gravidanza}{%
\subsection{La mammella durante la
gravidanza}\label{la-mammella-durante-la-gravidanza}}

In gravidanza le mammelle, sotto l'influenza ormonale, subiscono
modificazioni, che coinvolgono l'apparato duttale, lobulare e alveolare.
Ciò si ripercuote sull'aspetto macroscopico delle ghiandole mammarie le
quali, fin dalle prime settimane di gestazione, appaiono ingrossate e
compatte; l'areola è piu pigmentata e il capezzolo più eretto e grande.
Nella seconda metà delta gravidanza dal capezzolo fuoriescono gocce di
liquido giallastro, il colostro, che sarà sostituito dal latte
definitivo poche ore dopo la nascita, al momento dell'allattamento.
L'ormone chiave in questi processi è rappresentato dalla prolattina, i
cui livelli plasmatici crescono in gravidanza. In particolare, la
prolattina aumenta lentamente nei primi mesi, fino a giungere a
concentrazioni plasmatiche 3-5 volte superiori durante gli ultimi due
trimestri di gravidanza. Tuttavia, l'azione della prolattina è
antagonizzata dall'ormone lattotropo placentare e dagli ormoni steroidi
(secreti dal corpo luteo gravidico), per cui la crescita delle mammelle
risulta equilibrata nel periodo gestazionale. Alla nascita, la riduzione
drastica dei livelli plasmatici di ormone lattotropo placentare e di
steroidi sessuali, ma non di prolattina, conduce al potenziamento degli
effetti di quest'ultima sull'epitelio mammario.

I cambiamenti rnorfologici a carico delle strutture epiteliali mammarie
sono il risultato dell'azione sinergica di ormoni che giocano un ruolo
importante nella gestazione, vale a dire: durante le prime quattro
settimane gli estrogeni determinano proliferazione duttale con
formazione di ramificazioni nei parenchima mammario; dalla V alla VIII
settimana di gestazione si rendono manifeste queste modificazioni a
livello macroscopico con ingrandimento delle mammelle e dilatazione
delle vene superficiali. Grazie al progesterone segue la proliferazione
dell'epitelio alveolare, che durante il II trimestre di gravidanza
supera quello del sistema duttale: gli alveoli sono numerosi, ma
incapaci di attività secretoria, poiché tale capacita e fornita dalla
prolattina, la cui influenza permette la secrezione di colostro. Il
costante aumento delle dimensioni mammarie è dovuto, da questo momento
in poi, non ai fenomeni proliferativi, bensì alla dilatazione
progressiva degli alveoli secernenti colostro e all'ipertrofia delle
cellule limitrofe, cellule mioepiteliali, connettivali e adipose.
Sarebbe errato pensare, tuttavia, che il nuovo ambiente microscopico sia
statico in quanto sottoposto a continui rimaneggiamenti
dell'architettura anatomica: le cellule epiteliali proliferano e si
sfaldano, la struttura pluristratificata è abbandonata per quella
monostratificata, tipica degli ultimi mesi di gravidanza; il tipo di
cellularità cambia, con predominanza di eosinofili, plasmacellule e
linfociti che circondano gli alveoli. Infatti, il colostro secreto a
fine gestazione presenta cellule di sfaldamento, aggregati linfocitari,
monociti e cellule schiumose o "foam cells" \textsuperscript{(2)} .

\hypertarget{la-mammella-durante-lallattamento}{%
\subsection{La mammella durante
l'allattamento}\label{la-mammella-durante-lallattamento}}

Al momento del parto le concentrazioni plasmatiche di lattotropo
placentare, estradiolo, progesterone e prolattina sono elevate. Dopo due
giorni solo le concentrazioni della prolattina restano tali, mentre
lattotropo placentare, estradiolo, progesterone calano, fino a
raggiungere livelli pre-gravidanza. A questo punto la mammella
fortemente influenzata dalla prolattina che, già durante la gestazione,
ne ha determinato lo sviluppo in senso secretorio.

La presenza di questo ormone permette la sintesi di caseina e di
lattoalbumina, essenziali per la produzione del principale componente
del latte materno, lattosio, ma non solo: la prolattina si rivela
indispensabile anche per la sua secrezione nel sistema duttale. Agonisti
della dopamina, infatti, determinando il blocco della secrezione di
prolattina dall'ipofisi, impediscono la lattazione.

L'inizio della poppata comporta aumento del livello sierico di
prolattina ma, poichè l'emivita è solo di trenta minuti, la sua
concentrazione decresce già nell'ora successiva, fino a raggiungere
valori basali minimi due ore dopo la poppata. L'ipofisi, secernendo
prolattina in maniera pulsatile, ripristina una concentrazione idonea
per la poppata successiva, anche se, con il trascorrere delle settimane,
ogni rilascio di prolattina endogena consta di livelli ematici
lievemente, ma costantemente più bassi dei precedenti; cosi nell'ultimo
periodo di allattamento la prolattina non ha più le concentrazioni
plasmatiche dell'immediato post-partum.

Quando la neo-mamma attacca al seno il bambino, la suzione di
quest'ultimo comporta l'attivazione di terminazioni nervose a livello
del capezzolo mammario, che attraverso afferenze sensitive determinano
la secrezione di ossitocina ipofisaria.

Questa ormone agisce sulle cellule mioepiteliali localizzate a ridosso
degli alveoli mammari, determinando la loro contrazione, cosicché il
latte è spinto nei duttuli alveolari. Anche i livelli di ossitocina sono
incrementati già in epoca gestazionale, come accade per la prolattina,
probabilmente in relazione alla presenza degli elevati livelli
estrogenici a cui l'ormone è sensibile.

Da notare, l'implicazione che l'ossitocina sembra avere con la sfera
psicologica e affettiva della donna, nonché madre: questo ormone è
presente in diverse aree cerebrali extra-ipotalamiche ed e coinvolto in
funzioni cerebrali attinenti al comportamento materno, sessuale, alla
memoria e all'apprendimento. Infatti, la sola vista del neonato o
l'udirne il pianto o anche solo pensare al neonato genera nella madre
aumento dei livelli sierici di ossitocina e se la madre è in una
situazione tranquilla. Mentre la sua secrezione a inibita da dolore,
stress, disagio psico-fisico dalla nicotina e dall'alcol. In caso di
mancato allattamento i valori di prolattina restano comunque elevati per
circa tre-quattro settimane, dopodiché si riducono. Microscopicamente,
si assiste a fenomeni di apoptosi cellulare, di fagocitosi con sinergica
riduzione del volume e della compattezza delle mammelle, che nell'arco
di poche settimane ripristinano le loro dimensioni pre-gravidanza.
\textsuperscript{(2-3) }

\hypertarget{la-lattogenesi}{%
\subsection{La lattogenesi}\label{la-lattogenesi}}

La {lattogenesi di tipo I} è uno switch di differenziazione in senso
secretorio delle cellule mammarie che inizia con la gravidanza. Come i
livelli di progesterone, prolattina e lattogeno placentare aumentano, le
unità terminali dutto-lobulari vanno incontro ad aumento delle
dimensioni tanto che i lobuli iniziano a sembrare dei grandi grappoli
d'uva. A metà gravidanza, la differenziazione in senso secretorio inizia
con un incremento della produzione di mRNA per enzimi e proteine del
latte ed un accumulo di gocce di grasso nelle cellule mammarie, che
diventano poi la componente predominante a fine gravidanza.

La ghiandola mammaria rimane quiescente ma pronta ad iniziare la
secrezione di latte dopo il parto. Questo periodo di quiescenza dipende
dalla presenza di alti livelli circolanti di progesterone. Quando si
verifica la caduta di questo al momento del parto, inizia lo stadio II
della lattogenesi con la produzione di latte. Più a lungo è mantenuta la
secrezione di prolattina e il latte è rimosso dalla ghiandola, più a
lungo la secrezione di latte e mantenuta. \textsuperscript{(3.4:5) }

Durante la {lattogenesi di tipo II} aumenta la quantità di latte che
viene prodotta e varia la sua composizione. Il volume di latte prodotto
il primo giorno postpartum è \textless{}100 ml, il secondo giorno circa
200 ml, poi 400 ml il terzo giorno, per arrivare a circa 500 ml al
quarto giorno postpartum.

Contemporaneamente varia anche la composizione della secrezione
mammaria, con una riduzione di cloruro di sodio e un incremento dei
livelli di lattosio, che raggiungono il picco entro 72h dal parto.
Questi cambiamenti nella composizione si verificano prima
dell'incremento della produzione di latte e riflettono la chiusura delle
tight junctions con relativo blocco della via paracrina di regolazione
delle cellule mammarie.

Successivamente a 24h postpartum aumentano i livelli di IgA e
lattoferrina che rimangono elevati fino a circa 48 ore dopo la nascita,
per poi ridursi drammaticamente, in parte per la diluizione dovuta
all'aumento del volume di latte. Il contenuto di oligosaccaridi è
elevato nella lattazione precoce, per poi cadere significativamente dopo
30 giorni. Questi zuccheri complessi sembrano avere un effetto
protettivo contro diversi tipi di infezioni.

Quindi durante i primi due giorni postpartum, grandi molecole con
effetti protettivi sembrano essere predominanti nella secrezione
mammaria (IgA ed oligosaccaridi) con un basso contenuto calorico,
semplicemente perché il volume di latte trasferito al neonato è basso.
Il sostanziale incremento di volume di latte si ha tra le 36 e le 96h
postpartum (quando avviene la montata lattea) e riflette il massivo
incremento del tasso di sintesi e/o secrezione di quasi tutte le
componenti del latte maturo, inclusi lattosio, proteine del latte
(principalmente caseina), lipidi, calcio, sodio, magnesio e potassio,
con incremento quindi del contenuto calorico \textsuperscript{(3,4,5)} .

\hypertarget{regolazione-dello-stadio-ii-della-lattogenesi}{%
\subsection{Regolazione dello stadio II della
lattogenesi}\label{regolazione-dello-stadio-ii-della-lattogenesi}}

L'avvio della fase II della lattogenesi avviene per brusca caduta dei
livelli di progesterone dopo il parto e per la presenza di elevati
livelli di Prolattina (circa 200 ng/mL). La caduta dei livelli di
progesterone si verifica dopo espulsione/rimozione della placenta. Se
questo non si verifica, come in caso di ritenzione di frammenti
placentari che conservano la capacità di secernere progesterone, è stato
riportato un ritardo nell'inizio della lattogenesi II.

Anche le variazioni dei livelli di altri ormoni funzionano da trigger,
ad esempio dopo il parto la caduta dei livelli di lattogeno placentare
promuove la lattogenesi insieme ad alti livelli di prolattina. Infatti
la Bromocriptina e altri analoghi della dopamina (farmaci che
effettivamente prevengono la secrezione di prolattina) inibiscono la
lattogenesi quando somministrati a dosi appropriate. Ma l'aumento di
prolattina da sola non è in grado di promuovere la lattogenesi.

I glucocorticoidi sono necessari sia per la lattogenesi che per la
secrezione di latte. Ma elevati livelli non promuovono la lattogenesi.

L'insulina è generalmente richiesta per l'induzione e il mantenimento
dell'espressione genica delle proteine del latte, non è ben conosciuto
il suo ruolo ma sembra importante per il mantenimento di uno stato
metabolico che porta ad avere a disposizione un flusso di nutrienti
verso la ghiandola mammaria.

I glucocorticoidi e la prolattina sono necessari a determinati livelli
perché la caduta dei livelli di progesterone sia efficace come trigger
della lattogenesi. Il presupposto, ovviamente, è che l'epitelio mammario
sia sviluppato e sufficientemente preparato dagli ormoni della
gravidanza per rispondere con la secrezione di latte.

I livelli postpartum di prolattina sono gli stessi sia nelle donne che
allattano sia nelle donne che non allattano, quindi questo processo di
base, la produzione della prolattina, si verifica comunque,
indipendentemente dal fatto che l'allattamento abbia inizio.
\textsuperscript{(3,4, 5) }

\hypertarget{cause-di-fallimento-della-lattogenesi-ii}{%
\subsection{Cause di fallimento della lattogenesi
II}\label{cause-di-fallimento-della-lattogenesi-ii}}

Un ritardo nell'onset della lattogenesi è stato osservato nelle pazienti
diabetiche con scarso controllo glicemico e sottoposte a stress durante
il parto. Il meccanismo è sconosciuto ma il ritardo potrebbe essere
correlato con alti livelli di glucosio e cortisolo nel sangue cordonale
e quindi nel neonato.

Un altro meccanismo necessario ad un adeguato sviluppo della lattogenesi
II è la rimozione precoce di latte dalla mammella, necessaria per
ottenere la chiusura delle tight junctions tra le cellule mammarie e
aumentarne la produzione da un lato, dall'altro la rimozione del secreto
porta a rimozione di fattori prodotti localmente, che contribuiscono, se
non rimossi nella fase precoce del postpartum, a inibire la lattogenesi
II, nonostante gli adeguati livelli degli altri ormoni coinvolti.

Possiamo classificare le cause di fallimento della lattogenesi come
preghiandolari, ghiandolari e postghiandolari. Tra le cause
preghiandolari possiamo citare un esempio, già presentato in precedenza,
come la ritenzione placentare oppure una scarsa produzione di prolattina
da parte della ghiandola pituitaria.

Cause ghiandolari possono essere pregresse procedure chirurgiche, come
una mammoplastica riduttiva oppure un tessuto mammario insufficiente.

Cause postghiandolari invece sono tutte quelle che portano a inefficace
infrequente rimozione di latte. Fondamentale per una corretta
lattogenesi II è la chiusura delle tight junctions, che puo essere
ritardata, può andare incontro a fallimento oppure queste possono
riaprirsi a causa della mancata rirnozione di latte.
\textsuperscript{(6) }

\hypertarget{galattopoiesi}{%
\subsection{Galattopoiesi}\label{galattopoiesi}}

Un riflesso neuromonale è responsabile del mantenimento della secrezione
lattea. La stimolazione meccanica dei capezzoli durante l'allattamento
al seno ha come effetto, attraverso un riflesso integrato a livello
dell'ipotalamo, un ridotto rilascio di dopamina che, nelle donne che non
allattano, inibisce la produzione e la secrezione della prolattina.
Inoltre, sembrerebbe esserci anche un rilascio, a livello ipotalamico,
di TRH. Il TRH ed altri neuropeptidi come la serotonina, promuovono la
produzione di prolattina da parte del lobo anteriore dell'ipofisi. La
concentrazione di prolattina, ogni volta che la madre allatta, aumenta
di circa 10 volte, stimola la produzione di latte per la volta
successiva. I peptidi oppioidi partecipano al controllo della secrezione
di prolattina sulla quale esplicano azione stimolante. Essi agiscono
inibendo il turn-over della dopamina ed il suo rilascio da parte dei
neuroni tubero-infundibulari e stimolando l'attivita serotoninergica. A
livello delle cellule lattotrope diminuiscono la risposta prolattinica
all'inibizione dopaminica e possono comportarsi come antagonisti
recettoriali per la dopamina.

La PRL ha molte funzioni nel processo di allattamento: ha azione trofica
a livello delle cellule degli alveoli mammari e interviene nella sintesi
delle proteine del latte; attiva la alfa-lattoalbumina che porta alla
formazione del lattosio; si lega a recettori presenti sulla superficie
mammaria con un meccanismo di UP-REGULATION: un aumento di PRL in
circolo provoca il concomitante aumento dei propri recettori cellulari;
Il complesso PRL-recettore entra nelle cellule dove induce: aumento
dell'RNA ribosomiale e aumento dell'RNA delle caseine. Gli effetti della
PRL suite caseine sono amplificati dai corticosteroidei e dall'insuline,
mentre sono inibiti dal progesterone. La PRL ha attività di
osmoregolazione: promuove la ritenzione di liquidi stimolando
l'escrezione dello ione sodio dalla mammella. Essa inizia ad aumentare
nell'ipofisi della madre già dalla V settimana fino al parto. L'effetto
della PRL potenziato dalla somatomammotropina corionica prodotta dalla
placenta. Il latte viene continuamente secreto negli alveoli mammari, ma
non scorre agevolmente nel sistema dei dotti e perciò non sgorga in modo
continuo dai capezzoli. \textsuperscript{(6) }

\hypertarget{eiezione-del-latte}{%
\subsection{Eiezione del latte}\label{eiezione-del-latte}}

Lo stesso riflesso neuro-ormonale indurrà anche la liberazione
dell'ossitocina necessaria a promuovere la fuoriuscita del latte mediata
dalla contrazione delle cellule a canestro mioepiteliali che circondano
l'alvelo mammario, determinando una spremitura dello stesso e la
fuoriuscita di latte attraverso i dotti galattofori, quindi l'eiezione
lattea al capezzolo. La suzione del bambino stimola i nuclei
paraventricolare e sovraottico dell'ipotalamo, con la conseguente
produzione di ossitocina. Questa a sua volta stimola la contrazione
delle cellule mioepiteliali che circondano gli alveoli. L'aumento della
pressione provoca il deflusso del latte dagli alveoli ai dotti.

Questa risposta ossitocinica può essere condizionata, oltre che dalla
suzione, da stimoli esterni; tutte le emozioni positive aumentano il
rilascio di ossitocina (per esempio il solo sentire il vagito del
bambino); in queste situazioni ci può essere la fuoriuscita di latte
anche in assenza di suzione da parte del bambino. Un altro fattore
importante sulla regolazione della produzione di latte e il Fattore di
Inibizione della Lattazione (FIL) prodotto localmente dalle cellule
alveolari che fa diminuire la produzione quando la mammella è troppo
piena. Solo la rimozione del latte, grazie alle poppate efficaci e
frequenti può ripristinare la produzione del latte. \textsuperscript{(4)
}

\emph{\textbf{Bibliografia}: }

Fisiologia. Klinke, Pape. Terza edizione. Pag.585-588

Guyton e Hall - Fisiologia medica 13 ed. Pag. 349-356

II sostegno dell'allattamento al seno: fisiologia e falsi miti. Sergio
Conti Nibali. http:www.quaderniacp.it/.N°2-2015

Fisiologia dell'allattamento al seno - EpiCentro
www.epicentro.iss.itiargomenti/allattamento/oms/PM03it.pdf

J Nutr. 2001 Nov;131(11):3005S-8S. Physiology and endocrine changes
underlying human lactogenesis II. Neville MC1, Morton J

J Am Diet Assoc. 1999 Apr;99(4):450-4; quiz 455-6. Identification of
risk factors for delayed onset of lactation. Chapman DJ1,
Pérez-Escamilla R.

\hypertarget{scale-di-valutazione-dellallattamento}{%
\section{SCALE DI VALUTAZIONE
DELL'ALLATTAMENTO}\label{scale-di-valutazione-dellallattamento}}

Le linee guida della American Academy of Pediatrics
\textsuperscript{(1)} stabiliscono che per la dimissione precoce il
neonato deve essere in grado di eseguire due sessioni di allattamento
adeguate ed essere in grado di coordinare suzione, deglutizione e
respirazione mentre è attaccato al seno.\textsuperscript{(2)}

Il coordinamento tra suzione, deglutizione e respirazione compare
durante la vita intrauterina, tra la 32a e 34a settimana di gestazione e
diventa efficiente dalla 36a settimana. \textsuperscript{(6,7,8) }

La valutazione dell'efficacia dell'allattamento per la diade
madre-figlio può iniziare immediatamente dopo la nascita o entro la
prima ora dalla nascita, quando il neonato è sveglio e vigile. Questa
valutazione include: come il neonato ricerca il capezzolo, come si
attacca ad esso e come avviene la suzione, la posizione del neonato al
seno e il livello di comfort della madre. \textsuperscript{(3)}

L'uso di strumenti oggettivi, come le scale di punteggio, può facilitare
tale valutazione e aiutare i sanitari a fornire alla diade madre-figlio
gli strumenti necessari affinché il processo di allattamento avvenga
correttamente e sia confortevole per entrambi.

Le scale più frequentemente utilizzate includono la Infant Breastfeeding
Assessment Tool (IBFAT), la Systematic Assessment of the Infant at
Breast (SAIB), la Mother-Baby assessment (MBA), la LATCH Assessment, la
LAT Lactation Assessment Tool, la Mother-Infant Breastfeeding Progress
Tool (MIBPD, PIBB Score (Premature Infant Breastfeeding Behavior Scale).
\textsuperscript{(4)}

La \textbf{Infant Breastfeeding Assessment Tool (IBFAT)}
\textsuperscript{(4,10,11)} è stata pubblicata per la prima volta nel
1988 e consiste in 4 categorie che valutano le maggiori componenti del
comportamento del bambino che si attacca al seno:

prontezza alla nutrizione

riflesso di ricerca del capezzolo

fissarsi (latch-on) al capezzolo

suzione

La Systematic Assessment of the Infant at Breast
(SAIE)\textsuperscript{(4,12)} è stata pubblicata per la prima volta nei
1990 ed è uno strumento semplice e lineare per valutare l'efficacia del
comportamento di nutrizione al seno del bambino. Non c'è uno score.
Serve come guida per insegnare alle madri come iniziare ad allattare.

\begin{longtable}[]{@{}lllll@{}}
\toprule
\textbf{Infant Breastfeeding Assessment Tool (IBFAT)} & & &
&\tabularnewline
\midrule
\endhead
\textbf{Punteggio} & 3 & 2 & 1 & 0\tabularnewline
\textbf{Prontezza alla nutrizione} & Inizia a nutrirsi prontamente senza
stimoli & Richiede modica stimolazione per iniziare a nutrirsi &
Richiede molta stimolazione per svegliarsi ed iniziare a nutrirsi & Non
può essere svegliato\tabularnewline
\textbf{Ricerca} & Ricerca efficacemente in una sola volta & Ha bisogno
di essere stimolato ed incoraggiato & Il comportamento di ricerca è
senza stimolazione & Non ricerca\tabularnewline
\textbf{Fissazione} & Si nutre immediatamente & Richiede da 3 a 10
minuti per iniziare & Richiede più di 10 minuti per iniziare & Non si
nutre\tabularnewline
\textbf{Modalità di Suzione} & Suzione valida ad entrambi i seni & La
suzione è alternata ed ha bisogno di essere incoraggiato & Suzione
debole, alternanza, per brevi periodi & Suzione assente\tabularnewline
\textbf{Punteggio massimo possibile} & 12 & 8 & 4 & 0\tabularnewline
\bottomrule
\end{longtable}

\begin{longtable}[]{@{}ll@{}}
\toprule
\textbf{Systematic Assessment of the Infant at Breast } &\tabularnewline
\midrule
\endhead
\begin{minipage}[t]{0.47\columnwidth}\raggedright
\textbf{Allineamento }\strut
\end{minipage} & \begin{minipage}[t]{0.47\columnwidth}\raggedright
Il bambino è in posizione flessa, rilassata e senza rigidità muscolare

La testa e il corpo del bambino sono a livello del seno

La testa del bambino è allineata con il tronco e non è girata di lato
iperestesa o iperflessa

Il corretto allineamento del corpo del bambino è confermato da una linea
immaginaria che va dall'orecchio alla spalla alla crescita iliaca

Il seno della madre è supportata con la mano a coppa durante le prima 2
settimane di allattamento\strut
\end{minipage}\tabularnewline
\begin{minipage}[t]{0.47\columnwidth}\raggedright
\textbf{Afferrare l'areola}\strut
\end{minipage} & \begin{minipage}[t]{0.47\columnwidth}\raggedright
La bocca è ben aperta; le labbra non sono seria

Le labbra sono visibili e rivolte verso l'esterno

Sigillamento completo ed effetto vacuum sono effettuate dalla bocca del
bambino

Approssimativamente 1,5 cm dall'areola intorno al capezzolo è centrata
nella bocca del neonato

La lingua copre il margine areolare inferiore

La lingua è curvata intorno e sotto l'areola

Durante la suzione non si sentono rumori

Non si osserva sporgenza delle guance durate la suzione\strut
\end{minipage}\tabularnewline
\begin{minipage}[t]{0.47\columnwidth}\raggedright
\textbf{Compressione areolare}\strut
\end{minipage} & \begin{minipage}[t]{0.47\columnwidth}\raggedright
La mandibola si muove in modo ritmico

Se indicato, la suzione di un dito rileva un movimento ad onda della
lingua della parte anteriore della bocca verso l'orofaringe (solitamente
non viene eseguito)\strut
\end{minipage}\tabularnewline
\begin{minipage}[t]{0.47\columnwidth}\raggedright
\textbf{Deglutizione udibile}\strut
\end{minipage} & \begin{minipage}[t]{0.47\columnwidth}\raggedright
Un suono quieto di deglutizione è percepito

Può essere preceduto da alcuni movimenti di suzione

Può aumentare in frequenza e consistenza dopo il riflesso di eiezione
del latte\strut
\end{minipage}\tabularnewline
\bottomrule
\end{longtable}

La \textbf{Mother-Baby assessment (MBA)} \textsuperscript{(4;13;14)} è
uno strumento pubblicato per la prima volta nel 1992 e consente di
valutare gli sforzi di nutrizione al seno del bambino utilizzando uno
score simile a quello di APGAR.

Per ognuno dei 5 criteri sia la madre che il bambino ricevono un punto.

\begin{longtable}[]{@{}lll@{}}
\toprule
\textbf{Mother-Baby Assessment Tool Scoring System (MBA)} &
&\tabularnewline
\midrule
\endhead
\textbf{Passi} & \textbf{Punteggio} & \textbf{Criteri/cosa
rilevare}\tabularnewline
\begin{minipage}[t]{0.30\columnwidth}\raggedright
\textbf{Segnalazione}\strut
\end{minipage} & \begin{minipage}[t]{0.30\columnwidth}\raggedright
1

1\strut
\end{minipage} & \begin{minipage}[t]{0.30\columnwidth}\raggedright
Madre: guarda e ascolta il bambino, può prenderlo, stringerlo;
stimolarlo se sonnolento, calmarlo se agitato

Bambino: segnali di prontezza come eccitazione, allerta, ricerca,
suzione, portarsi le mani alla bocca, emettere suoni, piangere\strut
\end{minipage}\tabularnewline
& &\tabularnewline
\begin{minipage}[t]{0.30\columnwidth}\raggedright
\textbf{Posizione}\strut
\end{minipage} & \begin{minipage}[t]{0.30\columnwidth}\raggedright
1

1\strut
\end{minipage} & \begin{minipage}[t]{0.30\columnwidth}\raggedright
Madre: prende il bambino in un buon allineamento per farlo attaccare al
capezzolo; il corpo del bambino è leggermente flesso; l'intera parte
ventrale del bambino è rivolta verso la madre; la testa e le spalle del
bambino sono sorrette.

Bambino: ricerca bene il capezzolo, apre bene la bocca, la lingua copre
la parte inferiore dell'areola\strut
\end{minipage}\tabularnewline
& &\tabularnewline
\begin{minipage}[t]{0.30\columnwidth}\raggedright
\textbf{Fissazione}\strut
\end{minipage} & \begin{minipage}[t]{0.30\columnwidth}\raggedright
1

1\strut
\end{minipage} & \begin{minipage}[t]{0.30\columnwidth}\raggedright
Madre: solleva il seno per assistere il bambino se necessario; tiene il
bambino vicino quando la sua bocca è ben aperta; può far uscire dal
capezzolo alcune gocce di latte.

Bambino: si attacca; prende in bocca il capezzolo e circa 2 cm di
areola; successivamente succhia, con un pattern caratterizzato da
suzioni e pause frequenti\strut
\end{minipage}\tabularnewline
& &\tabularnewline
\begin{minipage}[t]{0.30\columnwidth}\raggedright
\textbf{Trasferimento del latte}\strut
\end{minipage} & \begin{minipage}[t]{0.30\columnwidth}\raggedright
1

1\strut
\end{minipage} & \begin{minipage}[t]{0.30\columnwidth}\raggedright
Madre: riporta una di queste sensazioni: sete, crampi uterini, aumento
delle lochiazioni, dolore al seno o formicolio, rilassamento,
sonnolenza, fuoriuscita di latte dal capezzolo opposto

Bambino: deglutizione udibile; latte osservabile nella bocca del
bambino; il bambino può rigurgitare del latte quando sazio; passaggio di
suzioni rapide e frequenti a suzioni lente e nutritive.\strut
\end{minipage}\tabularnewline
& &\tabularnewline
\begin{minipage}[t]{0.30\columnwidth}\raggedright
\textbf{Fine}\strut
\end{minipage} & \begin{minipage}[t]{0.30\columnwidth}\raggedright
1

1\strut
\end{minipage} & \begin{minipage}[t]{0.30\columnwidth}\raggedright
Madre: il seno non è dolente: lascia che il bambino succhi il latte fino
alla fine, i seni sono più morbidi dopo l'allattamento, non ci sono
grumi, ingorghi o dolore ai capezzoli

Bambino: rilascia il seno spontaneamente; appare sazio; non ricerca se
stimolato; viso, braccia e mani sono rilassate, può addormentarsi\strut
\end{minipage}\tabularnewline
& &\tabularnewline
\bottomrule
\end{longtable}

La \textbf{LATCH Assessment} \textsuperscript{(4;15)} è stata modellata
sul sistema APGAR con un punteggio ottenibile da 0 a 10. È stata
pubblicata per la prima volta nel 1994.

\begin{longtable}[]{@{}llll@{}}
\toprule
\textbf{LATCH Scoring System} & & &\tabularnewline
\midrule
\endhead
\textbf{Punteggio} & 2 & 1 & 0\tabularnewline
\textbf{Attacco (Latch)} & Afferra il seno, la lingua è giù, le labbra
sono arricciare e i movimenti di suzione sono ritmici & Tentativi
ripetuti, prende il capezzolo in bocca, è stimolato alla suzione &
Troppo addormentato o riluttante; nessun tentativo di
attacco\tabularnewline
\textbf{Suzione udibile} & Spontanea e intermittente \textless{} 24h;
spontanea e frequente \textgreater{}24h & Minima con stimolazione &
Nessuna\tabularnewline
\textbf{Tipo di capezzolo} & In fuori & Piatto &
Invertito\tabularnewline
\textbf{Comfort} & Soffice, morbido & Pieno, piccole bollicine rosse,
graffi, medio/moderato disconfort & Ingorgo, crepe sanguinanti, grandi
bolle o graffi e disconfort severo\tabularnewline
\textbf{Presa (posizione)} & Non assistita dallo staff; la madre è
capace di posizionare e tenere il bambino & Massima assistenza (per es.
sollevamento della testa del letto, posizionamento di cuscini come
supporto), un assistente vicino può posizionare il bambino e la madre
continuare a tenerlo & Piena assistenza (lo staff tiene sia il bambino
che il seno della madre)\tabularnewline
\bottomrule
\end{longtable}

La \textbf{LATCH Assessment} \textbf{Tool}\textsuperscript{(4;15)} è
stata sviluppata per la prima volta nel 1999 per un progetto di ricerca.

\begin{longtable}[]{@{}ll@{}}
\toprule
\textbf{Location Assessment Tool (LAT)} &\tabularnewline
\midrule
\endhead
\textbf{Processi di attaccamento (ricerca, presa, chiusura, suzione)} &
Assicurarsi che il bambino inizi con la ricerca del capezzolo, poi la
presa, la chiusura della bocca attorno ad esse ed infine
suzione\tabularnewline
\textbf{Angolo della bocca aperto al seno} & Un angolo minimo di 160
gradi\tabularnewline
\textbf{Labbra arricciate } & Il labbro superiore e inferiore non sono
girati\tabularnewline
\textbf{Posizione della testa del bimbo } & Naso e mento sono vicini al
seno\tabularnewline
\textbf{Linea delle guance del bambino} & Linea morbida\tabularnewline
\textbf{Altezza del bambino al seno } & Naso opposto al capezzolo per
iniziare\tabularnewline
\textbf{Rotazione del corpo del bambino } & Il torace del bambino è
ruotato verso il seno materno\tabularnewline
\textbf{Relazioni del corpo del bambino } & Il bambino è orizzontale sul
seno materno\tabularnewline
\textbf{Dinamica di suzione } & Serie di suzioni (deglutizione 2:1 o
1:1) e il seno si muove ritmicamente con la suzione\tabularnewline
\bottomrule
\end{longtable}

Il \textbf{Mother-Infant Breastfeeding Progress Tool (MIBPT)}
\textsuperscript{(4)} è stato sviluppato secondo il concetto che sia la
madre che il bambino contribuiscono allo creazione e al successo della
relazione di allattamento. Lo strumento consiste in 8 classi che possono
essere osservate durante una sessione di allattamento.

\begin{longtable}[]{@{}l@{}}
\toprule
\textbf{Mother-Infant Breastfeeding Progress Tool (MIBPT)
}\tabularnewline
\midrule
\endhead
La madre risponde alle richieste nutrizionali del bambino (il bambino
cerca e si attacca)\tabularnewline
La madre non fa passare più di 3 ore tra un tentativo e l'altro di
allattamento\tabularnewline
Il bambino si aggrappa all'areola con la bocca ben aperta, le labbra
visibilmente arricciate\tabularnewline
Sono notati dei gruppi di suzioni nutritive\tabularnewline
La madre è capace di posizionarsi da sola per
l'allattamento\tabularnewline
La madre è capace di attaccare autonomamente il bambino al
seno\tabularnewline
Non sono presenti traumi sui capezzoli materni\tabularnewline
Non ci sono commenti negativi sull'allattamento\tabularnewline
\bottomrule
\end{longtable}

Il \textbf{PIBB Score (Premature Infant Breastfeeding Behavior Scale)}
\textsuperscript{(4;18)} serve ad accertare le capacità alimentari in
modo da identificare specifici problemi nutrizionali. Scala di
valutazione con punteggio differente (0-2, 0-3, 0-4). E' costituita da 6
categorie: riflesso di ricerca; capacità del neonato nell'afferrare
l'areola; tempo di attacco; efficacia della suzione; sequenza più lunga
di suzioni consecutive; presenza e frequenza della deglutizione. E'
utilizzabile sia dalle madri che dagli infermieri. La validità di questo
strumento espressa come Kappa di Cohen è 0,80, quindi molto alta. Lo
strumento è veloce e affidabile. \textsuperscript{(5) }

\begin{longtable}[]{@{}llll@{}}
\toprule
\textbf{PIBB scale} & \textbf{A 12h} & \textbf{A 48h} &\tabularnewline
\midrule
\endhead
\textbf{Riflesso di ricerca} & Non ricerca & 1 & 1\tabularnewline
& Mostra alcuni comportamenti di ricerca (apertura della bocca,
estensione della lingua, movimenti mano-bocca/viso, gira la testa & 2 &
2\tabularnewline
& Mostra chiari comportamenti di ricerca (apre la bocca e gira la testa
in maniera simultanea) & 3 & 3\tabularnewline
\textbf{Capacità di afferrare} & Nessuna, la bocca tocca solamente il
capezzolo & 1 & 1\tabularnewline
\textbf{L'areola (quale porzione della mammella è nella bocca del
neonato)} & Parte del capezzolo & 2 & 2\tabularnewline
& L'intero capezzolo ma non l'areola & 3 & 3\tabularnewline
& Il capezzolo e in parte l'areola & 4 & 4\tabularnewline
\textbf{Tempo di attacco (latch on)} & Non si attacca e la madre lo
percepisce & 1 & 1\tabularnewline
& Si attacca per meno di 1 minuto & 2 & 2\tabularnewline
& Si attacca per 1 -15 minuti o più (indicare minuti: 1 -5; 5-10; 10-15)
& 3 & 3\tabularnewline
\textbf{Efficacia della suzione (suckling)} & Nessuna suzione, non lecca
la mammella & 1 & 1\tabularnewline
& Lecca e tasta la mammella ma non c'è suzione & 2 & 2\tabularnewline
& Suzioni singole; serie di suzioni corte ed occasionali (2- 9 suzioni)
& 3 & 3\tabularnewline
& Serie di suzioni corte ma ripetute (2 o più serie consecutive); serie
di suzioni lunghe e occasionali (10 o più suzioni prima di una pausa) &
4 & 4\tabularnewline
& Serie di suzioni lunghe e ripetute & 5 & 5\tabularnewline
\textbf{Sequenza più lunga di suzioni consecutive} & Massimo numero di
suzioni consecutive (indicare numero: 1-9; 10-19; 20-30) & \ldots{}.. &
\ldots{}..\tabularnewline
\textbf{Deglutizione} & Non si evidenzia deglutizione & 1 &
1\tabularnewline
& Si evidenzia deglutizione occasionale & 2 & 2\tabularnewline
\textbf{TOTALE} & & &\tabularnewline
\bottomrule
\end{longtable}

La misura più accurata dell'adeguatezza dell'allattamento è il guadagno
ponderale documentato con misurazioni ripetute del peso.

Il picco del calo fisiologico del peso avviene circa al 3° giorno dalla
nascita e solitamente non supera il 7\% del peso neonatale. Dal momento
in cui inizia la montata lattea al secondo-quarto giorno postpartum, il
bambino può iniziare a prendere peso in uno-due giorni. Dal decimo al
quattordicesimo giorno postpartum il bambino può tornare al peso delle
nascita, e guadagnare circa 20-35 grammi al giorno durante i primi due
mesi di vita. \textsuperscript{(4)}

\emph{\textbf{Bibliografia:}}

American Academy of Pediatrics andAmerican College of Obstetricians and
Gynecologist (AAP/ACOG). Guidelines for perinatal care 5th ed. Elk Grove
Village, IL: American Academy of Pediatrics, 2002

Hall RT, Simon S, Smith MR,. Readmission of breastfed infant in the
first 2 weeks of life. J Perinatol 2000;20:432-7

American Academy of Pediatrics Workgroup on Breastfeeding. Breastfeeding
and the use of human milk. Pediatrics 2005;115:496-506

Pamela D. Assessment of Breastfeeding and Infant Growth. Journal of
Midwifery \& Women's Healt. 2007.52: 571-578

J. Crippa. Gli strumenti di valutazione delle competenze del neonato
pretermine nel soddisfacimento del bisogno di alimentazione. Rivista
L'Infermiere N°5 - 2013

Kenner C, McGrath J M(2004). Feeding. In: Mosby H, editor. Developmental
care of newborns \& infants: a guide for health professionals. National
association of neonatal nurses: Usa, 321-39.

da Costa S P, van den Engel-Hoek L, Bos A F (2000). Sucking and
swallowing in infants and diagnostic tools. J Perinatol, 28(4), 247-57.

Barlow S M (2009). Oral and respiratory control for preterm feeding.
Curr Opin Otolaryngol Head Neck Surg, 17(3), 179-86.

Arvedson J, Clark H, Lazarus C, Schooling T, Frymark T (2010).
Evidence-based systematic review: effects of oral motor interventions on
feeding and swallowing in preterm infants. Am J Speech Lang Pathol,
19(4), 321-40.

Matthews MK. Developing an instrument to assess infant breastfeeding
behaviour in the early neonatal period. Midwifery 1988;4:154-65.

Groer MW, Humenick SS, Hill PD. Characterizations and
psychoneuroimmunologic implications of secretory immunoglobin A and
cortisol in preterm and term breast milk. J Perinat Neonat Nurs
1994;23:27-32.

Shrago LC, Bocar DL. The infant's contribution to breastfeeding. J
Obstet Gynecol Neonatal Nurs 1990;19:209-15.

Mulford C. The mother-baby assessment (MBA): An "Apgar acore" for
breastfeeding. J Hum Lact 1992;8:79-82.

Riordan JM, Koehn M. Reliability and validity testing of three
breastfeeding assessment tools. J Obstet Gynecol Neonatal Nurs
1997;26:181-7.

Jensen D, Wallace s, Kelsay P. LATCH: A breastfeeding charting system
and documentation tool. J Obstet Gynecol Neonatal Nurs 1994;23:27-32.

Blair A, Cadwell K. Turner-Maffei C, Brimdyr K. The relationship between
positioning, the breastfeeding dynamic, the latching process and pain in
breastfeeding mothers with sore nipples. Breastfeed Rev 2003;11:5-10.

Johnson TS., Mulder Pi, Strube K. Mother-infant breastfeeding progress
tool: A guide for education and support of the breastfeeding dyad. JOGNN
J Obstet Gynecol Neonat Nurs 2007;36: 319-27.

Nyqvist KH1, Rubertsson C, Ewald U, SjOden PO. Development of the
Preterm Infant Breastfeeding Behavior Scale (PIBBS): a study of
nurse-mother agreement. J Hum Lact. 1996 Sep;12(3):207-19.

\hypertarget{partoanalgesia}{%
\section{PARTOANALGESIA}\label{partoanalgesia}}

Il controllo del dolore durante il travaglio ed il parto ha grande
importanza nella pratica ostetrica; prima però di arrivare a controllare
il dolore, è necessario avere una piena conoscenza sia della
localizzazione topografica che delle funzioni delle vie nervose
afferenti di tutti gli organi e le strutture coinvolte durante il parto
così come di tutti gli effetti che il fenomeno dolore comporta nella
partoriente.

Soltanto con queste premesse e conoscenze sarà possibile evitare ogni
interferenza con i meccanismi fisiologici del travaglio e del parto e
con l'attività muscolare determinante per esercitare sufficienti forze
espulsive.

Il dolore del parto è probabilmente il dolore più intenso che la maggior
parte delle donne sperimenta nel corso della vita.

L'era moderna della partoanalgesia inizia nel 1847, quando il Dr James
Young Simpson somministrò il dietiletere ad una donna durante il parto.
Nello stesso periodo al St. Bartholomiew Hospital di Londra Skey e
Tracey eseguono un taglio cesareo in anestesia eterea, estraendo una
bambina in condizioni ottime, mentre lo stesso Simpson sperimenta con
successo l'analgesia con cloroformio durante il parto. Nel 1853 John
Snow sottopose alla stessa metodica la regina Vittoria per la nascita
del principe Leopoldo, e quattro anni dopo la ripete per la nascita
della principessa Beatrice. Da allora l'interesse sia pubblico che
scientifico per l'analgesia in travaglio dilaga.

Nel 1885 Cornig realizza la prima analgesia epidurale, partendo
dall'ipotesi che un medicamento iniettato nel canale vertebrale possa
essere assorbito dalle vene intervertebrali e quindi essere trasportato
al midollo spinale. Il primo ad utilizzare con successo le tecniche di
analgesia loco-regionale in campo ostetrico e Soeckel nel 1909, ma è
Aburel, nel 1931 a codificare la tecnica dell'analgesia epiduraIe
continua in travaglio, messa poi a punto, tra gli anni 40 e 50 da
Flowers e Dogliotti, su cui si basano le metodologie attuali.

\emph{Anatomia e fisiologia del dolore del travaglio e del parto}

\hypertarget{fasi-del-travaglio}{%
\subsection{Fasi del travaglio}\label{fasi-del-travaglio}}

La suddivisione tradizionale delle fasi del travaglio e del parto è
quella del ginecologo americano Milton Friedman, che ha distinto tre
stadi:

PRIMO STADIO

Dall'inizio delle contrazioni alla dilatazione completa della cervice,
diviso a sua volta in due fasi:

\begin{itemize}
\item
  FASE LATENTE, prodromica (della durata di 8 ore in media), in cui non
  c'è dilatazione, il collo dell'utero si appiattisce e si ammorbidisce.
  Le contrazioni si verificano ogni 5-10 minuti, durano meno di 45
  secondi e non superano l'inensità di 30 mmHg. Nella primipara di
  solito termina con una dilatazione di 3-4 cm.
\item
  FASE ATTIVA, suddivisa in FASE DI ACCELERAZIONE, in cui si ha una
  dilatazione di 1,2-1,5 cm/h, FASE DI MASSIMA ATTIVITA', FASE DI
  DECELERAZIONE. Le tre fasi durano circa due ore ognuna. Le contrazioni
  hanno una frequenza di una ogni 2-3 minuti, con una intensità di 40-60
  mmHg e una durata di più di 50 secondi.
\end{itemize}

Durante il primo stadio, periodo dilatante, le contrazioni delle fibre
muscolari sono di tipo isometrico, per cui non si osserva una riduzione
di volume dell'utero, ma solo distensione e assottigliamento del
segmento inferiore, appianamento del collo e sua dilatazione fino alla
scomparsa (dilatazione completa). Le pareti uterine vengono a formare un
canale unico con le pareti della vagina, costituendo il canale del
parto. Il primo stadio termina con la dilatazione completa, di solito
con la parte presentata a livello dello stretto medio.

SECONDO STADIO

Si accompagna alla discesa del corpo mobile lungo il canale del parto.
Le contrazioni uterine provocano una riduzione di volume dell'utero,
accompagnando il feto verso l'espulsione. Questa fase dura da mezz'ora a
due ore.

TERZO STADIO

Secondamento, che si realizza con l'espulsione della placenta.

\hypertarget{le-vie-del-dolore}{%
\subsection{Le vie del dolore}\label{le-vie-del-dolore}}

Il dolore generato durante l'intero travaglio ha origine diversa e
diversa conduzione nervosa: quello del primo stadio del travaglio è
dovuto alla contrazione del muscolo uterino, allo stiramento e
dilatazione della cervice e del segmento inferiore, alla trazione sui
legamenti, fenomeni che si verificano durante le contrazioni. Queste
sono di tipo isometrico, in quanto non sono accompagnate da variazioni
di volume dell'utero, che trova un ostacolo posto dalla cervice e dal
perineo. Le contrazioni uterine contribuiscono al dolore da parto
stimolando recettori e fibre appartenenti al sistema Nervoso Vegetativo,
fibre C, sottili, prive di rivestimento mielinico, a lenta velocità di
conduzione, che raccolgono la sensibilità dolorifica dei plessi
intrauterini, la convogliano attraverso gangli cervicali di
Frankenhauser, poi attraverso i plessi ipogastrici inferiore, medio e
superiore, li trasmettono alle aree somatiche T10-L1, attraversando la
catena del simpatico laterale. Lungo la stessa via, in senso opposto,
decorrono le fibre efferenti, neurovegetative simpatiche, che partono
dalle corna laterali del simpatico a livello T5-L2 e arrivano alle
fibrocellule muscolari uterine, per governarne la contrattilità e il
flusso ematico. Il blocco simpatico provocato dalla peridurale blocca
anche queste fibre favorendo la dilatazione della cervice.

Durante la fase iniziale del primo stadio il dolore è limitato ai
dermatomeri T11 e T12. Quando il travaglio procede verso la fase attiva
del primo stadio, corrispondente a una dilatazione dì 3-4 cm, diventa
più severo, e si estende ai dermatomeri contigui T10 e L1 .

Clinicamente questo si traduce nel tipico "mal di schiena" con dolore a
fascia che colpisce i lombi, fino alla radice delle cosce. Il
doloregenerato da questo tipo di fibre è di tipo viscerale, sordo,
indistinto, difficilmente localizzabile, ``riferito" verso aree cutanee
distanti dal sito d'insorgenza della stimolazione dolorifica, ma
corrispondenti ai metameri innervati dagli stessi segmenti spinali.

Quando la cervice uterina è completamente dilatata inizia il secondo
stadio del travaglio: il dolore che si aggiunge proviene dalla
distensione, stiramento e lacerazione delle strutture perineali in
concomitanza della progressione fetale. Questo dolore tardivo è di
origine perineale e condotto da fibre mielinizzate, somatiche di tipo
A-delta e C, che decorrono nei nervi pudendi, afferenti al secondo,
terzo e quarto metamero sacrale, con coinvolgimento anche dei nervi
genitofemorale, ileoinguinale e femorocutaneo.

Altri fattori che incidono sull'insorgenza del dolore durante questa
fase sono le dimensioni del feto, il tipo di presentazione, l'intensità
e la durata delle contrazioni, la velocità di dilatazione del collo, la
durata della fase di riposo.

Una volta dilatata la cervice, il dolore da essa generato diminuisce,
mentre continuano a generare dolore le contrazioni del corpo dell'utero
e la distensione del segmento uterino inferiore, come durante il primo
stadio; inoltre divengono elementi algogeni la stimolazione della parte
presentata sulle strutture pelviche e la distensione dello stretto
inferiore della pelvi e del perineo. Progressivamente la maggiore
distensione della fascia e dei tessuti sottocutanei ne provoca ulteriore
stiramento e tensione, fino alla lacerazione, incrementando il dolore
perineale.

Nell'ultima parte del primo stadio, e durante il secondo, possiamo avere
irradiazione del dolore alle gambe, alle cosce, per stimolazione di
strutture viscerali come il peritoneo e i legamenti uterini, la vescica,
il retto, fasce muscolari pelviche, pressione della parte presentata
sulle radici sacrali (come nella rotazione posteriore). Insieme alle
fibre somatiche in questi nervi sacrali decorrono anche fibre
parasimpatiche, che sarebbero responsabili del riflesso di Fergusson,
cioè della increzione ossitocinica che seguirebbe alla distensione del
canale del parto.

L'entità di questo dolore varia anche in relazione a fattori fisici,
psicologici, emozionali, socioculturali della madre.

\hypertarget{effetti-sistemici-del-dolore-del-parto}{%
\subsection{Effetti sistemici del dolore del
parto}\label{effetti-sistemici-del-dolore-del-parto}}

Ogni stimolo nocicettivo, sia che venga condotto da fibre somatiche che
neurovegetative, al suo arrivo nelle corna posteriori del midollo genera
risposte segmentarie, soprasegmentarie e corticali. Il corno dorsale è
infatti un vero e proprio crossing di vie neuronali, per la modulazione
e la elaborazione dello stimolo doloroso, attraverso collegamenti con le
corna anteriori dello stesso e di altri metameri, o con altri sistemi di
trasmissione e di collegamento con altri distretti encefalici, fino ai
nuclei della base e alla corteccia.

Alcuni impulsi nocicettivi quindi, dopo essere stati esposti ad
influenze modulatrici nel corno dorsale, passano direttamente, tramite
interneuroni, alle cellule del corno anteriore ed anterolaterale dove,
stimolano neuroni somatomotori e neuroni pregangliari simpatici,
determinando risposte riflesse nocicettive difensive vegetative e
somatiche (risposte segmentarie: reattività motoria, sudorazione,
irradiazione a tessuti limitrofi). Altri impulsi nocicettivi sono
trasmessi a neuroni i cui assoni formano i sistemi ascendenti che li
convogliano al tronco mesencefalico e a nuclei neurovegetativi,
generando le risposte soprasegmentarie, fino alla corteccia, dove viene
elaborata la percezione del dolore, generando le risposte corticali. E
sono proprio queste risposte soprasegmentarie che danno origine a quei
coinvolgimenti respiratori, emodinamici, neuroendocrini, che attraverso
l'organismo materno, arrivano in ultima analisi a determinare effetti
anche deleteri sul feto.

\begin{itemize}
\item
  \emph{Effetti respiratori}
\end{itemize}

\begin{quote}
La stimolazione dolorosa comporta aumento della ventilazione, con
aumento sia dei volume corrente che del volume minuto, con aumenti da 10
a 20 1/min. Da ciò riduzione della PPaCO2 al di sotto dei 15-20 mmHg e
aumento del pH fino a 7,50-7,60. Alla fine della contrazione cessa lo
respiratorio causato dal dolore, per cui si ha una fase di
ipoventilazione causata dall'ipocapnia, che provoca una riduzione della
PaO2 di circa il 20\%. Effetto questo potenziato dall'eventuale
somministrazione di oppiacei. Conseguenza dell'alcalosi e della
riduzione della saturazione di ossigeno \textless{}70\%, è la
vasocostrizione del letto placentare con ipoafflusso fetale,
decelerazioni tardive.

L'analgesia peridurale è in grado di interferire con queste risposte,
sia bloccando la risposta allo dolorifico, sia inibendo l'insorgenza
dello stress. Quindi nei riguardi della ventilazione, blocca
iperventilazione stimolata dal dolore e quindi permettendo una riduzione
dell'ipocapnia e impedisce l'ipoventilazione ipocapnica, normalizzando
la PaO2.
\end{quote}

\begin{itemize}
\item
  \emph{Effetti emodinamici}
\end{itemize}

\begin{quote}
Durante il travaglio la gittata cardiaca aumenta del 40-50\%, con un
ulteriore aumento del 20-30\% durante le contrazioni, a causa di un
effetto di spremitura dall'utero di 250-300 ml di sangue, e per un
aumentato ritorno venoso dalla pelvi e dagli arti inferiori. Questo
aumento del ritorno venoso che si ha durante le contrazioni è dovuto
all'azione dei legamenti dell'utero che ad ogni contrazione lo sollevano
e lo allontanano dalla colonna vertebrale, riducendo l'occlusione vasale
e producendo aumento della gittata cardiaca sistolica, del volume minuto
e del lavoro cardiaco. La componente dolorifica, attraverso una
stimolazione simpatica e increzione adrenalinica, produce un incremento
aggiuntivo della gittata, con aumento della pressione arteriosa di 20-30
mmHg, con ulteriore del lavoro cardiaco, che può sfociare nello
scompenso in pazienti cardiopatiche, ipertese, gestosiche. Questo
identifica la partoanalgesia addirittura come atto terapeutico in queste
pazienti, in quanto la frequenza cardiaca rimane stabile, la gittata non
si modifica oltre valori previsti durante le contrazioni e il sistema
cardiovascolare non subisce pericolosi incrementi di lavoro.
\end{quote}

\begin{itemize}
\item
  \emph{Effetti neuroendocrini}
\end{itemize}

\begin{quote}
Il dolore provoca aumento dell'increzione di catecolamine, specialmente
di noradrenalina ad effetto alfa stimolante, con vasocostrizione e
ipertono uterino, con decremento del flusso uterino che va dal 35 al
70\% e quindi riduzione degli scambi materno-fetali. L'aumento
dell'adrenalina rientra nella risposta neuroendocrina allo stress
generato dalla componente dolorosa e da quella ansiosa: questo provoca
aumento del metabolismo anaerobio, con produzione di lattato e acidi
grassi liberi e aumento del consumo di ossigeno. Il rene da parte sua
cerca di compensare l'alcalosi respiratoria eliminando bicarbonato e
contribuendo all'instaurazione di acidosi metabolica. La noradrenalina,
attraverso una stimolazione alfa recettoriale, provoca un effetto
uterotonico, ma anche regolarizzante l'attività contrattile uterina
attraverso un meccanismo di dominanza fundica, mentre l'adrenalina, ad
effetto beta stimolante, produce un effetto tocolitico. Lo stress
indotto dal travaglio aumenta in maniera abnorme l'increzione di
adrenalina, ma non quella di noradrenalina, determinando
regolarizzazione dell'attività uterina. Anzi la riduzione dello stress
può consentire il ripristino di un normale ritmo di contrazioni in quei
travagli che lo stesso stress rende distocici.

L'increzione di catecolamine fetali non viene influenzata dalla
partoanalgesia, per cui rimane invariato lo "stress" che il feto
affronta al momento della nascita e che gli permette di adattarsi al
nuovo ambiente che deve affrontare. Gli consente inoltre di avviare
tutti quei processi fisiologici iniziali, quali la produzione di
surfattante polmonare, la termogenesi, l'omeostasi glucidica, gli
adattamenti cardiovascolari e idroelettrolitici.
\end{quote}

\begin{itemize}
\item
  \emph{Effetti psicologici}
\end{itemize}

\begin{quote}
Da non sottovalutare gli effetti a distanza che possono reliquare dopo
un'esperienza dolorosa e stressante di questo grado: la puerpera può
sviluppare uno stato depressivo che può riflettersi sui rapporti della
donna verso neonato e verso il partner.
\end{quote}

\begin{itemize}
\item
  \emph{Effetti sul feto}
\end{itemize}

\begin{quote}
Già fisiologicamente il flusso intervilloso placentare subisce una
riduzione in concomitanza delle contrazioni uterine. Se a ciò
aggiungiamo l'iperventilazione materna causata dal dolore e dallo
stress, avremo un'ulteriore riduzione di perfusione placentare dovuta
all'alcalosi respiratoria che si manifesta nella partoriente. L'alcalosi
provoca uno spostamento a sinistra della curva di dissociazione
dell'emogloina materna con ridotta cessione di ossigeno al feto,
vasocostrizione ombelicale. Inoltre l'aumento di noradrenalina provoca
vasocostrizione uterina e ulteriore ipoaffiusso ematico fetale.

Se questo ipoafflusso intermittente viene comunque tollerato dal feto
normale, grazie a un accumulo di ossigeno che si viene a creare negli
spazi intervillosi e nella circolazione fetale e a un aumento della
gittata cardiaca fetale, in caso di eccessiva attività contrattile
uterina, o per problemi fetali generati da gestosi, IUGR, diabete, il
feto risentirà particolarmente di questa situazione, potendo arrivare a
sviluppare ipossia, ipercapnia, acidosi metabolica, che possono
comprometterne seriamente la prognosi. L'analgesia peridurale si
inserisce in questi meccanismi aumentando il flusso intervilloso
attraverso una regolarizzazione del pH e i suoi meccanismi di
vasodilatazione. Questo diventa addirittura terapeutico nelle situazioni
patologiche come preeclampsia e diabete, in cui alla base c'è
unapatologia del flusso placentare. Inoltre è estremamente utile nei
feti small for date o comunque patologici.
\end{quote}

\hypertarget{obiettivi-della-partoanalgesia}{%
\subsection{Obiettivi della
partoanalgesia}\label{obiettivi-della-partoanalgesia}}

Il primo obiettivo per non interferire con l'evoluzione del travaglio, è
di limitarsi al controllo della sensibilità dolorifica senza interferire
con la sensibilità propriocettiva o profonda (pressoria, vibratoria. di
posizione, discriminativa) e con la sensibilità esterocettiva o
superficiale (protopatica ed epicritica).

Per incidere il meno possibile nel travaglio bisogna perseguire degli
obiettivi: adattare l'analgesia alla fase del travaglio, evitare
l'ipotensione, favorire la deambulazione, salvaguardare la sensibilità,
fare emergere un dolore di significato patologico.

Non interferendo con la sensibilità propriocettiva, inoltre, garantiremo
la deambulazione considerandola un marker di analgesia ben condotta,
senza dimenticare che la deambulazione produce un miglioramento dei
diametri pelvici materni, una migliore coordinazione, riduzione della
frequenza e maggiore intensità delle contrazioni uterine, migliora
l'outcome neonatale per minore compressione aorto-cavale, riduce la
percezione delle contrazioni e quindi vi è una minore richiesta di
analgesico. Pertanto tratteremo il dolore viscerale bloccando solo le
fibre C mieliniche e il dolore somatico allargando il blocco alle fibre
Adelta, ma soprattutto dovremo ottenere tutto ciò con basse
concentrazioni per evitare l'ipotensione secondaria al blocco delle
fibre B simpatiche pregangliari situate come dimensione tra le fibre C e
le fibre Adelta.

Il secondo obiettivo da perseguire è: evitare di interferire con la
percezione di spinta dal momento che condiziona il rifleso di Fergusson
che determina la normale secrezione di ossitocina endogena. Un'analgesia
che rispetti queste prerogative non potrà essere causa di distocia,
piuttosto il mancato o difficile controllo del dolore potrà essere
considerato un marker indiretto di distocia o di dolore patologico.

\hypertarget{timing-dellanalgesia-epidurale-in-travaglio}{%
\subsection{Timing dell'analgesia epidurale in
travaglio}\label{timing-dellanalgesia-epidurale-in-travaglio}}

I STADIO - PERIODO DILATANTE

Nella fase I del travaglio è necessario un blocco delle fibre C
amieliniche: in questo stadio sono sufficienti soluzioni di anestetico
locale a bassa concentrazione; se la testa non è ancora impegnata
possono essere utilizzati soltanto farmaci oppioidi.

Durante la fase dei travaglio le fibre nervose interessate sono le fibre
A-delta per cui sarà necessario utilizzare soluzioni più concentrate di
anestetico locale. In accordo con la fisiopatologia del dolore, quanto
più precocemente si inizia una analgesia, tanto meno farmaco occorre per
abolire il dolore. L'innervazione simpatica del collo uterino lo rende
sensibile agli effetti del blocco simpatico operato dall'analgesia
perimidollare, che favorisce in tal modo la dilatazione cervicale se la
testa fetale è impegnata. Come ogni atto medico anche l'analgesia
neuroassiale necessita di alcuni monitoraggi materni e fetali (PA, Fc,
SaO2, cardiotocogramma) che possono tuttavia essere limitati ai primi
15-20 minuti da ogni somministrazione di farmaco, salvo diversa
indicazione medica. L'analgesia epidurale a basso dosaggio è compatibile
usualmente con la deambulazione della partoriente.

II STADIO - PERIODO ESPULSIVO

In questo stadio le spinte volontarie della partoriente non vanno
consentite e/o incoraggiate fino a quando il livello della parte
presentata ha superato il piano dello stretto medio (livello 0) e la
rotazione della testa fetale è completata. Se non intervengono
alterazioni cardiotocografiche significative per sofferenza fetale tali
da consigliare un intervento strumentale o operativo d'urgenza, il
secondo stadio deve essere caratterizzato dall'osservazione della
progressione della parte presentata che nella nullipara può durare fino
a due ore. In questo stadio l'analgesia va mantenuta costante per
evitare l'improvviso ed intenso dolore che si avrebbe nel caso di una
sospensione accidentale della stessa. Se si inizia un'analgesia
perimidollare a dilatazione quasi completa è bene tenere presente che le
dosi di anestetico locale necessarie per abolire dolore sono maggiori e
possono talvolta ridurre temporaneamente la frequenza e l'intensità
delle contrazioni uterine e rendere necessaria la somministrazione di
ossitocina. Di norma, se l'analgesia neurassiale stata precedentemente
ben condotta, la partoriente è perfettamente in grado di avvertire la
sensazione di spinta e di compiere sforzi espulsivi efficaci.

\hypertarget{tecniche-neuroassiali-di-analgesia-in-travaglio}{%
\subsection{Tecniche neuroassiali di analgesia in
travaglio}\label{tecniche-neuroassiali-di-analgesia-in-travaglio}}

Le tecniche neurassiali per l'analgesia in travaglio di parto hanno
dimostrato di essere il metodo migliore di fornire analgesia se
paragonate ad altre.

Attualmente l'analgesia loco-regionale perimidollare è la tecnica di più
largo impiego per efficacia e sicurezza; è l'unica, in definitiva, che
rispettai criteri summenzionati con il più basso rapporto rischio
beneficio. Quando si parla di analgesia anestesia loco-regionale
perimidollare o neurassiale si fa riferimento essenzialmente al:

\begin{itemize}
\item
  blocco epidurale (continuo o intermittente)
\item
  blocco spinale o subaracnoideo continuo o single shot
\item
  blocco combinato spinale-epidurale (CSE).
\end{itemize}

A tutt'oggi, il blocco epidurale è la tecnica più utilizzata; la
spinale, infatti, ha un uso molto più limitato nel travaglio di parto,
mentre si sta diffondendo l'uso della tecnica combinata
spinale-epidurale (CSE).

\begin{itemize}
\item
  \emph{Analgesia epidurale}
\end{itemize}

\begin{quote}
L'analgesia epidurale consiste nel posizionamento di un cateterino nello
spazio epidurale che si trova dietro al midollo spinale. La manovra
viene effettuata, previa disinfezione cutanea, a livello lombare tra la
2° e la 3° o tra la 3° e la 4° vertebra lombare. La procedura inizia con
la infiltrazione della cute e dei piani profondi con anestetico locale.
Si procede quindi all'introduzione, nello spazio peridurale lombare, di
un apposito cateterino inserito tramite un particolare ago (ago di Tuohy
16 G) , posizionato nello spazio presente tra due vertebre lombari.
inserito il cateterino peridurale ed estratto l'ago, un'idonea
medicazione assicura la sterilità ed il fissaggio del cateterino stesso
alla cute. All'estremità del cateterino viene collegato un filtro
antibatterico di sicurezza tramite il quale si può iniettare la miscela
analgesica.
\end{quote}

\begin{itemize}
\item
  \emph{Analgesia subaracnoidea}
\end{itemize}

\begin{quote}
Il blocco si ottiene con l'introduzione di oppioidi e/o anestetici
locali direttamente nel Liquido Cefalo Rachídiano (LCR), attraversando
con un ago di piccole dimensioni le due meningi che proteggono il
midollo spinale, la dura madre e l'aracnoide. La somministrazione in
questo caso è unica e non può essere ripetuta come per il blocco
peridurale. In realtà esiste la possibilità di eseguire una
cateterizzazione dello spazio subaracnoideo ma attualmente la
letteratura ha ancora delle riserve per questa tecnica nella sua
attuazione in caso del travaglio di parto.
\end{quote}

\begin{itemize}
\item
  \emph{Analgesia combinata spino-epidurale (CSE)}
\end{itemize}

\begin{quote}
Questa tecnica consiste nella combinazione delle procedure utilizzate
per l'analgesia epidurale e spinale. La metodologia più diffusa è quella
dell'''ago-attraverso-ago''. Nello spazio intervertebrale prescelto si
inserisce un ago epidurale all'interno del quale viene introdotto un ago
subaracnoideo. La punta di quest'ultimo viene spinta fino a penetrare
nella dura madre e si inietta la dose appropriata di miscela analgesia
nello spazio subaracnoideo. Dopo aver ritirato l'ago subaracnoideo si
inserisce un catetere epidurale, il cui ruolo è quello di prolungare
l'analgesia o indurre anestesia mediante iniezione di boli successivi,
una volta terminato l'effetto della prima dose. Il principale vantaggio
di questa tecnica rispetto all'epidurale standard è dato da una più
rapida insorgenza dell'effetto analgesico determinata dalla
somministrazione subaracnoidea iniziale.
\end{quote}

\hypertarget{farmaci}{%
\subsection{Farmaci}\label{farmaci}}

La miscela farmacologica utilizzata per effettuare l'analgesia in
travaglio di parto è costituita da farmaci oppioidi e anestetici locali.
Gli oppioidi (Sufentanyl o Fentanyl) sono utili, da soli, nella prima
parte del travaglio, quando il dolore è prevalentemente di tipo
viscerale ed è mediato fondamentalmente dalle fibre amieliniche C.
Questo è vantaggioso se si tiene conto che il meccanismo d'azione
spinale degli oppioidi non prevede l'interruzione della trasmissione
nervosa, come avviene con gli anestetici locali; pertanto non avremo mai
un blocco motorio né una simpaticolisi.

Quando dolore comincia a diventare somatico (mediato dalle fibre
A-delta) è indispensabile affiancare all'oppioide l'anestetico locale
(Ropivacaina o Levobupivacaina). Il razionale dell'aggiunta di oppioide
alla miscela dell'anestetico locale permette di ridurre la
concentrazione di quest'ultimo risparmiando le fibre motorie cosi da
ottenere un blocco differenziale con separazione degli effetti sensitivi
da quelli motori. Inoltre l'aggiunta di oppiacei è in grado
d'incrementare la durata dell'analgesia e l'estensione metamerica del
blocco. Un'azione selettiva sulle fibre nervose permette di sfruttare al
meglio la possibilità di ottenere un blocco differenziale. Nel corso di
una analgesia di parto si dovrebbe mirare a bloccare la trasmissione
nocicettiva afferente delle sole fibre sottili C e A delta entrambe a
lenta conduzione risparmiando le fibre nervose motorie A alfa mediante
la scelta di una concentrazione di anestetico locale ridotta.

\emph{\textbf{Bibliografia:}}

\begin{enumerate}
\def\labelenumi{\arabic{enumi}.}
\item
  Analgesia in travaglio di parto. Frigo, Todde, Furicchia.
\item
  Analgesia Epidurale in travaglio di parto, istruzioni per l'uso.
  Ferdinando Fant
\item
  Anatomia e fisiologia del dolore del travaglio e del parto. Antonio
  Cardarelli. Chiacchio, Savoia.
\item
  Manuale pratico di analgesia epidurale in travaglio di parto. Giorgio
  Capogna, Michela Camorcia.
\item
  Il blocco epidurale: tecnica, controindicazioni e complicanze. Dando
  Celleno, Mariagrazia Frigo.
\item
  Epidural Labor Analgesia, Childbirth without pain. Giorgio Capogna.
  Springer 2015
\end{enumerate}

\hypertarget{obiettivi-dello-studio-materiali-e-metodi}{%
\section{OBIETTIVI DELLO STUDIO, MATERIALI E
METODI}\label{obiettivi-dello-studio-materiali-e-metodi}}

L'{obiettivo primario} era quello di valutare se l'analgesia di parto
per via epidurale influenzi l'inizio e l'adeguatezza dell'allattamento
al seno nel periodo di ricovero in ospedale e a 7 giorni dal parto;
l'{obiettivo secondario} è valutare quale relazione esiste tra
partoanalgesia e difficoltà/fallimento dell'allattamento al seno:

\begin{itemize}
\item
  l'effetto dell'intenzionalità materna prepartum sul successo
  dell'allattamento al seno;
\item
  la relazione tra dose totale di oppioide e anestetico locale
  somministrati per vie epidurale e il punteggio della PIBB scale nel
  post-partum;
\item
  il tempo di comparsa della montata lattea;
\item
  il peso corporeo neonatale ad una settimana dal parto;
\item
  allattamento esclusivo al seno al momento della dimissione e a 7
  giorni dal parto.
\end{itemize}

\emph{Disegno dello studio}

Lo studio è stato osservazionale prospettico di coorte ed ha valutato
l'allattamento al seno delle madri che hanno ricevuto la partoanalgesia
epidurale con anestetico locale e oppiode. La procedura di
partoanalgesia è avvenuta su richiesta materna durante il travaglio.
L'arruolamento è avvenuto in sala parto: le donne che hanno fatto
richiesta di partoanalgesia epidurale sono state arruolate nel gruppo di
studio, mentre le donne che non hanno fatto richiesta di partoanalgesia
sono state arruolate in maniera randomizzata nel gruppo controllo,
tramite procedura di estrazione da un bussolotto della dicitura
"arruolabile"/ "non arruolabile" eseguita dal personale anestesiologico
di guardia e reperibile per il servizio di partoanalgesia.

I dati inerenti al travaglio di parto, all'inizio dell'allattamento al
seno, al punteggio della PIBB SCALE, alla comparsa della montata lattea
e alle caratteristiche dell'allattamento durante il ricovero in
ospedale, sono stati registrati su appositi moduli cartacei e raccolti
mediante intervista telefonica da parte del personale anestesiologico
appositamente formato.

\emph{Criteri d'inclusione }

Nello studio sono state arruolate le donne primipare a 37 o più
settimane di gestazione che in corso di travaglio hanno richiesto la
partoanalgesia (gruppo di studio). Le donne che non hanno richiesto la
partoanalgesia sono state arruolate come caso-controllo. Tutte le
partecipanti allo studio hanno compreso la procedura di partoanalgesia e
le modalità di esecuzione dello studio che sono state spiegate durante
la visita anestesiologica finalizzata a raccogliere i dati necessari per
eventuale partoanalgesia e che viene effettuata a partire dalla 32
settimana di gestazione; hanno inoltre firmato i moduli di consenso alla
partoanalgesia e allo studio durante il colloquio con il medico
anestesista presso l'Ambulatorio di Partoanalgesia.

\emph{Criteri d'esclusione }

presenza di controindicazioni assolute alla puntura epidurale (diatesi
emorragica, sepsi sistemica o infezione nel sito di puntura,
ipertensione endocranica);

somministrazione di farmaci per via endovenosa/intramuscolare
alternativi alla partoanalgesia neuroassiale (tramadolo, meperidina);

multiparità;

presenza di diabete mellito gestazionale;

peso del neonato \textless{} 2500 gr e/o neonato con anomalie anatomiche
del cavo orale e/o cromosomopatie;

esecuzione di manovre di rianimazione neonatale (incubazione tracheale,
massaggio cardiaco) e ricovero del neonato in terapia intensive:

esito del parto in taglio cesareo.

\emph{Popolazione e modalita' di arruolamento}

E' costituita da tutte le donne primipare che hanno richiesto la
partoanalgesia per via epidurale (gruppo di studio) e dalle donne che,
non avendo richiesto la partoanalgesia, sono state arruolate nel gruppo
di controllo. Il reclutamento è avvenuto presso la sala parto
dell'Ospedale Santa Maria della Misericordia di Perugia al momento della
richiesta della partoanalgesia, previa verifica dell'acquisizione del
consenso informato alla partecipazione allo studio durante colloquio
presso l'Ambulatorio di Partoanalgesia nel periodo pre-partum.

Le donne che al momento del travaglio non hanno richiesto la
partoanalgesia ma che hanno espresso il consenso a partecipare allo
studio, sono state arruolate nel gruppo controllo, in maniera
randomizzata, tramite metodo descritto precedentemente.

\emph{Durata dello studio }

I dati di ciascuna paziente arruolata utili all'analisi dello studio
sono stati raccolti dal momento della visita anestesiologica fino alla
prima settimana postpartum. La durata totale dello studio sarà di
ventiquattro mesi. Sono state reclutate finora 20 pazienti nel gruppo
partoanalgesia e 20 pazienti nel gruppo di controllo nel periodo che va
dal 01/01/2018 al 30/04/2018 previa approvazione da parte del Comitato
Etico.

\emph{Parametri valutati }

Vedi documento "variabili". Di seguito sono elencati i macro-gruppi di
variabili: dati materni, intenzione materna prepartum di allattamento,
sala parto e dati neonato, dati partoanalgesia, PIBB score alla 3°-4°
poppata postpartum, PIBB scale alla dimissione, allattamento alla
dimissione, colloquio telefonico a 7 giorni dal parto.

\emph{Analisi dei dati}

I dati raccolti, suddivisi per gruppo di soggetti, sono stati presentati
in tabella con media e deviazione standard nel caso delle variabili
numeriche o con tabelle di frequenza percentuale nel caso di variabili
qualitative. L'andamento dei parametri nei gruppi è stato rappresentato
graficamente.

Per i dati quantitativi, l'analisi statistica ha utilizzato il test
Kolmogorov-Smimov, per stabilire se i dati si distribuiscono
normalmente. In base al risultato abbiamo utilizzato il t-test e il test
Wilcoxon Mann Whitney per verificare le differenze tra i gruppi dei
pazienti trattati. Per i dati qualitativi si sono utilizzati il test
Chi-quadrato e il test esatto di Fisher. Il livello di significatività è
stato fissato con p \textless{} 0,05. L'elaborazione è stata effettuata
con software SAS 9.2 per Windows (SAS institute Inc. Cary, USA).

\emph{Eventi avversi e criteri di valutazione degli stessi}

In questo studio non sono stati utilizzati farmaci nuovi/sperimentali
per l'analgesia di parto. La partoanalgesia per via epidurale mediante
associazione di un anestetico locale e di un farmaco oppioide è una
tecnica validata dalla comunità scientifica internazionale e applicata
nella normale pratica clinica. Per quanto riguarda l'impiego epidurale
di un oppioide, immediatamente dopo la sua somministrazione
epidurale/spinale è possibile osservare variazioni transitorie della
frequenza cardiaca fetale (ridotta variabilità, bradicardia) che
tuttavia non si associano ad effetti neonatali avversi (\emph{Reynolds
F. International Journal of Obstetric Anesthesia (2011) 20, 38-50}). Per
quanto riguarda l'impiego epidurale dell'anestetico locale, la
concentrazione della soluzione utilizzata è efficace per bloccare solo
le fibre nervose che trasportano la sensazione dolorosa ma non per
bloccare quelle motorie, il rischio di blocco motorio dopo
somministrazioni ripetute di anestetico locale è in pratica nullo alla
concentrazione utilizzata nella nostra pratica clinica ma, qualora
dovesse verificarsi tale situazione, il blocco motorio è transitorio e
sarà sufficiente che il medico anestesista che assiste la paziente si
astenga da ulteriori somministrazioni fino alla ripresa della capacità
motoria degli arti inferiori.

\emph{Materiali e metodi}

Lo studio si è svolto nelle seguenti fasi:

{I FASE}

\begin{itemize}
\item
  {Informazione e acquisizione del consenso informato}: le donne che
  potevano essere arruolate nello studio erano quelle che, a partire
  dalla 32° settimana di gestazione, si sono presentate al colloquio
  informativo con il medico anestesista presso l'Ambulatorio di
  Partoanalgesia dell'Azienda Ospedaliera di Perugia ed hanno firmano il
  consenso per l'eventuale partoanalgesia. Nella stessa occasione le
  donne sono state informate anche dello studio clinico in atto e hanno
  espresso la loro adesione mediante firma del consenso dedicato. In
  quest'ultimo caso alle donne è stato fornito un questionario per
  valutare prepartum l'intenzione o meno di allattare al seno
  {[}\emph{vedi "moduli raccolta dati" - allegato 1}{]} (tale
  questionario non è stato allegato alla cartella clinica
  anestesiologica né visionato dall'anestesista presente al colloquio,
  ma conservato in apposito archivio in busta chiusa recante il codice
  identificativo della paziente: iniziali del cognome/nome e data di
  nascita gg/mm/aaaa).
\end{itemize}

{II FASE}

\begin{itemize}
\item
  \begin{quote}
  {Partoanalgesia:} le donne che hanno richiesto la partoanalgesia
  (gruppo studio) sono state sottoposte a partoanalgesia con tecnica
  epidurale e inserimento di catetere epidurale per l'analgesia
  continua. I farmaci utilizzati sono stati l'anestetico locale
  Levobupivacaina 0,0625\% + Fentanyl. Il volume del bolo iniziale di
  anestetico locale è stato di 10-20 ml a discrezione del medico
  anestesista, il volume dei boli anestetici successivi è stato di 10-15
  ml. La prima dose di Fentanyl è stata di 30-50 mcg; le dosi successive
  di 20-30 mcg. In caso di controllo insoddisfacente del dolore sono
  state somministrate una dose aggiuntiva di Fentanyl e/o 5-10 ml di
  Levobupivacaina 0,09\% - 0,125\% dopo almeno 20 minuti dalla dose
  precedente. In caso di somministrazione endovenosa di ossitocina per
  incrementare l'intensità delle contrazioni, fin dall'inizio della
  partoanalgesia la concentrazione di Levobupivacaina è stata di
  0,125\%. A dilatazione cervicale completa (10 cm) l'anestesista ha
  somministrato un bolo di anestetico locale di 10 ml alla
  concentrazione di 0,125\% +/- Fentanyl. Quanto la testa del feto era
  fissa al piano perineale durante la fase espulsiva, l'anestesista ha
  somministrato un bolo di Lidocaina 2\% 5-10 ml. Il medico anestesista
  ha annotato nella cartella le dosi parziali di Fentanyl e la
  concentrazione della soluzione anestetica usata nonché il volume di
  somministrazione dell'anestetico locale. Tutti i farmaci sono stati
  somministrati tramite bolo dal medico anestesista. Nel caso in cui la
  partoanalgesia sia stata richiesta mentre era in atto infusione di
  ossitocina, è stato necessario sospendere tale infusione per 20 minuti
  prima di procedere alla puntura epidurale.
  \end{quote}
\item
  \begin{quote}
  {Nopartoanalgesia}: le donne che hanno acconsentito a partecipare allo
  studio ma che non hanno richiesto la partoanalgesia sono state
  arruolate nel gruppo controllo (vedi paragrafo "selezione del gruppo
  di controllo"). Per queste pazienti sono stati unicamente registrati i
  dati utili allo studio non inerenti alla partoanalgesia e descritti
  nel documento ``variabili".
  \end{quote}
\item
  \begin{quote}
  {Raccolta dei dati materni:} per ciascuna donna arruolata sono stati
  registrati età, peso prima e a termine della gravidanza, altezza, Body
  Mass Index, livello di istruzione scolastica, professione,
  nazionalità, residenza, da quanto tempo era residente in Italia se
  immigrata da parte del personale anestesiologico formato{[}\emph{vedi
  allegato 2}{]};
  \end{quote}
\item
  {Dati del travagalio-parto:} per ciascuna donna arruolata è stato
  annotato se il travaglio è stato sottoposto a procedure d'induzione;
  se è stata somministrata ossitocina, per quanto tempo e a quale
  velocità (ml/ora); se, in caso dipartoanalgesia, c'è stata comparsa di
  brivido e qual è stata la temperatura corporea dopo l ora dall'inizio
  della stessa; la durata del II stadio del travaglio in minuti
  (dall'ora della prima notifica della dilatazione completa fino all'ora
  di estrazione); la modalità di parto (vaginale spontaneo, strumentale
  con Kiwi/forcipe, taglio cesareo) e l'eventuale episiotomia da parte
  del personale anestesiologico formato {[}\emph{vedi allegato 2}{]}.
\item
  {Dopo il parto ed entro il periodo di permanenza nella sala parto}:
  alla nascita è stato registrato l'orario di nascita (h:min), il peso e
  l'Apgar score (1°/5° minuto) del neonato, il Ph del sangue venoso
  cordonale. Inoltre è stato registrato da parte del personale
  anestesiologico il tempo intercorso dall'estrazione al primo contatto
  madre-figlio (entro 10 minuti, tra 10 e 30 minuti, oltre 30 minuti),
  la durata del contatto in minuti ("skin-to-skin": fino a 30 min, tra
  30 e 60 min, oltre 60 min) e il momento del primo attacco del bambino
  al seno (entro 1 ora dal parto, tra 1 e 2 ore, oltre 2 ore) da parte
  del personale anestesiologico formato {[}\emph{vedi allegato 2}{]}.
\end{itemize}

{III FASE}

\begin{itemize}
\item
  {Nelle successive ore dopo il parto fino alla dimissione}: si è
  valutata la correttezza dell'allattamento mediante \emph{PIBB scale}
  {[}\emph{vedi allegato 3}{]} alla 3°/4° poppata dopo il parto e alla
  dimissione in 2°/3° giornata postpartum; alla dimissione è stato
  registrato il peso del neonato, la percentuale di calo ponderale e
  compilato il questionario SIP sull'allattamento {[}\emph{vedi "moduli
  raccolta dati"- allegato 4}{]} (dati raccolti dal personale del
  Nido/Reparto di Ostetricia in cieco).
\end{itemize}

{IV FASE}

\begin{itemize}
\item
  {Al colloquio telefonico dopo 7 giorni dal parto}: effettuato dal
  personale anestesiologico formato è stato annotato quando è comparsa
  la montata lattea e il peso neonatale a 7 giorni dalla dimissione
  {[}\emph{vedi allegato 5}{]}.
\end{itemize}

{RACCOLTA DATI - ALLEGATO 1}

\textbf{{QUESTIONARIO PREPARTUM - PREVISIONE MATERNA DI ALLATTAMENTO}}

\begin{longtable}[]{@{}ll@{}}
\toprule
ID
MADRE:\ldots{}\ldots{}\ldots{}\ldots{}\ldots{}\ldots{}\ldots{}\ldots{}\ldots{}\ldots{}\ldots{}\ldots{}\ldots{}\ldots{}\ldots{}
settimana attuale di gestazione:.........................
&\tabularnewline
\midrule
\endhead
1) HA FREQUENTATO/STA FREQUENTANDO UN CORSO PRE-PARTUM? &\tabularnewline
si & 2\tabularnewline
no & 1\tabularnewline
2) HA FREQUENTATO/STA FREQUENTANDO UN CORSO SPECIFICO SULL'ALLATTAMENTO
MATERNO? &\tabularnewline
si & 2\tabularnewline
no & 1\tabularnewline
3) SI E' INFORMATA IN MANIERA AUTONOMA SULL'ALLATTAMENTO AL SENO
(riviste e/o libri, siti internet, consi li/o inioni da altre mamme
etc..) &\tabularnewline
si & 2\tabularnewline
no & 1\tabularnewline
4) CHE METODO PREVEDE DI ADOTTARE PER ALLATTARE SUO FIGLIO/A NELLE PRIME
SETTIMANE POSTPARTUM? &\tabularnewline
solo allattamento al seno {(andare alla domanda n.5)} & 4\tabularnewline
sia allattamento al seno che artificiale {(andare alla domanda n.6}) &
3\tabularnewline
solo allattamento artificiale {(andare alla domanda n.9)} &
2\tabularnewline
ancora non lo so {(andare alla domanda n.9)} & 1\tabularnewline
5) QUANTO PENSA CHE SARA' GRANDE SUO FIGLIO/A QUANDO COMINCERA' AD
INTEGRARE L'ALLATTAMENTO AL SENO CON IL LATTE ARTIFICIALE O ALTRI
ALIMENTI? &\tabularnewline
Meno di 1 mese & 1\tabularnewline
Da 1 a 2 mesi & 2\tabularnewline
Da 3 a 4 mesi & 3\tabularnewline
Da 5 a 6 mesi & 4\tabularnewline
Da 7 a 9 mesi & 5\tabularnewline
Più di 9 mesi & 6\tabularnewline
6) HA UN LAVORO? &\tabularnewline
si {(andare alla domanda n.7) } & 2\tabularnewline
no {(andare alla domanda n.8)} & 1\tabularnewline
7) PENSA CHE CONTINUERA' AD ALLATTARE QUANDO SARA' TORNATA A LAVORO?
&\tabularnewline
si & 3\tabularnewline
no & 2\tabularnewline
non penso di tornare a lavoro una volta che sarà nato/a mio figlio/a &
1\tabularnewline
8) QUANDO PENSA CHE SOSPENDERA' COMPLETAMENTE L'ALLATTAMENTO AL SENO?
&\tabularnewline
entro il 6° mese postpartum & 1\tabularnewline
tra il 6° e il 12° mese postpasrtum & 2\tabularnewline
dopo il 12° mese postpartum & 3\tabularnewline
9) PER IL BENESSERE DI SUO FIGLIO/A, PENSA CHE L'ALLATTAMENTO AL SENO:
&\tabularnewline
sia fondamentale e insostituibile & 3\tabularnewline
abbia delle qualità particolari ma non necessarie & 2\tabularnewline
abbia le stesse qualità di quello artificiale & 1\tabularnewline
10) PENSA CHE IL PERSONALE SANITARIO CHE HA INCONTRATO IN QUESTI MESI LE
ABBIA FORNITO SULL'ALLATTAMENTO UN'INFORMAZIONE: &\tabularnewline
adeguata & 3\tabularnewline
insufficiente & 2\tabularnewline
non ho ricevuto alcuna informazione fino ad ora & 1\tabularnewline
\bottomrule
\end{longtable}

{RACCOLTA DEI DATI - ALLEGATO 2 }

\textbf{DATI MATERNI: }

ID MADRE: { } ASA: I II III IV V

FUMO: SI NO

PESO PRE-GRAVIDANZA: Kg\_\_\_\_\_\_\_\_\_\_\_\_ PESO FINE GRAVIDANZA:
Kg\_\_\_\_\_\_\_\_\_\_\_\_

ALTEZZA: m\_\_\_\_\_\_\_\_\_\_\_\_

BMI: 18,50-24,99 25-29,99 30-34,99 35-39,99 \textgreater{}40

LIVELLO DI ISTRUZIONE: medie inferiori medie superiori laurea

PROFESSIONE:\_\_\_\_\_\_\_\_\_\_\_\_\_\_\_\_\_\_\_\_\_\_\_\_\_\_
NAZIONALITÀ:\_\_\_\_\_\_\_\_\_\_\_\_\_\_\_\_\_\_\_\_\_\_\_\_\_\_\_\_

RESIDENZA:\_\_\_\_\_\_\_\_\_\_\_\_\_\_\_\_\_\_\_\_\_\_\_\_ SE IMMIGRATA,
DA QUANTI ANNI È RESIDETE IN ITALIA?: \_\_\_\_\_\_\_\_\_

\textbf{SALA PARTO:}

SETTIMANA DI GESTAZIONE: \_\_\_\_\_\_\_\_\_INDUZIONE CON PROSTAGLANDINE:
SI NO

PROM: SI NO POSIZIONAMENTO PA: DILAZTAZIONE (cm) \_\_\_\_\_\_\_\_\_\_\_

ORA (h:min) \_\_\_\_\_\_\_\_\_\_

OSSITOCINA PRE-PA: SI NO OSSITOCINA POST-PA: SI NO

ml/ora: \_\_\_\_\_\_\_dalle ore (h:min): \_\_\_\_\_\_\_ ml/ora
\_\_\_\_\_\_\_dalle ore (h:min): \_\_\_\_\_\_\_

ml/ora: \_\_\_\_\_\_\_dalle ore (h:min): \_\_\_\_\_\_\_ ml/ora
\_\_\_\_\_\_\_dalle ore (h:min): \_\_\_\_\_\_\_

TEMP. MATERNA (dopo 1 ora dal 1° bolo PA): °C\_\_\_\_\_\_\_ BRIVIDO
POST-PA: SI NO

ORA NOTIFICA DILATAZIONE COMPLETA (h:min): \_\_\_\_\_\_\_\_\_\_\_ ORA
NASCITA (h:min): \_\_\_\_\_\_\_\_\_

PARTO: VAGINALE SPONTANEO VAGINALE STRUMETALE CESAREO

EPIFISIOTOMIA: SI NO

\textbf{DATI NEONATALI:} PESO: Kg\_\_\_\_\_\_\_\_ APGAR: 1°
\_\_\_\_\_\_\_/5°\_\_\_\_\_\_\_ Ph cordonale:\_\_\_\_\_\_\_

SKIN-TO- SKIN: entro 10 min dalla nascita

\begin{quote}
tra 10 min e 30 min dalla nascita

oltre 30 min dalla nascita

non effettuato

per 10-30 minuti

per 30-60 minuti

per oltre 60 minuti
\end{quote}

{PRIMO ATTACCO AL SENO: entro 1 ora dal parto tra 1 e 2 ore dal parto
oltre 2 ore}

{RACCOLTA DATI - ALLEGATO 3 }

\textbf{PIBBS - PRETERM INFANT BREASTFEEDING BEHAVIOR SCALE }

\begin{longtable}[]{@{}llll@{}}
\toprule
\begin{minipage}[b]{0.22\columnwidth}\raggedright
ID
MADRE:\_\_\_\_\_\_\_\_\_\_\_\_\_\_\_\_\_\_\_\_\_\_\_\_\_\_\_\_\_\_\_\_\_\_\_\_\_\_\_\_\_\_\_

DATA:

ORA:\strut
\end{minipage} & \begin{minipage}[b]{0.22\columnwidth}\raggedright
a 12h\strut
\end{minipage} & \begin{minipage}[b]{0.22\columnwidth}\raggedright
a 48h\strut
\end{minipage} & \begin{minipage}[b]{0.22\columnwidth}\raggedright
\strut
\end{minipage}\tabularnewline
\midrule
\endhead
\textbf{Riflesso di ricerca} (rooting) & non ricerca & 1 &
1\tabularnewline
& mostra alcuni comportamenti di ricerca (apertura bocca estensione
della lingua, movimenti mano-bocca/viso, gira la testa & 2 &
2\tabularnewline
& mostra chiari comportamenti di ricerca (apre la bocca e gira la testa
in maniera simultanea) & 3 & 3\tabularnewline
\textbf{Capacità di afferrare l'areola} (quale {porzione} della mammella
è nella bocca del neonato) & nessuna, la bocca tocca solamente il
capezzolo & 1 & 1\tabularnewline
& parte del capezzolo & 2 & 2\tabularnewline
& l'intero capezzolo ma non l'areola & 3 & 3\tabularnewline
& il capezzolo e in parte l'areola & 4 & 4\tabularnewline
\textbf{Tempo di attacco} (latch on) & non si attacca e la madre lo
percepisce & 1 & 1\tabularnewline
& si attacca per meno di 1 minuto & 2 & 2\tabularnewline
& si attacca per 1-15 minuti o più (indicare minuti: 1-5/5-10/10-15) & 3
& 3\tabularnewline
\textbf{Efficacia della suzione} (sucking) & nessuna suzione, non lecca
la mammella & 1 & 1\tabularnewline
& lecca e tasta la mammella ma non c'è suzione & 2 & 2\tabularnewline
& suzioni singole; serie di suzioni corte e occasionali (2-9 suzioni) &
3 & 3\tabularnewline
& serie di suzioni corte ma ripetute (2 o più serie consecutive); serie
di suzioni lunghe e occasionali (10 o più suzioni prima di una pausa) &
4 & 4\tabularnewline
& serie di suzioni lunghe e ripetute & 5 & 5\tabularnewline
\textbf{Sequenza più lunga di suzioni consecutive} & massimo numero di
suzioni consecutive (indicare numero: 1-9/10-19/20-30) &
\ldots{}\ldots{}\ldots{}\ldots{} &
\ldots{}\ldots{}\ldots{}\ldots{}\tabularnewline
\textbf{Deglutizione} (swallowing) & non si evidenzia deglutizione & 1 &
1\tabularnewline
& si evidenzia deglutizione occasionale & 2 & 2\tabularnewline
\textbf{TOTALE} & & &\tabularnewline
\bottomrule
\end{longtable}

{RACCOLTA DATI - ALLEGATO 4}

\textbf{QUESTIONARIO SULL'ALLATTAMENTO ALLA DIMISSIONE DALL'OSPEDALE}

\begin{longtable}[]{@{}lll@{}}
\toprule
\begin{minipage}[b]{0.30\columnwidth}\raggedright
ID
MADRE:\ldots{}\ldots{}\ldots{}\ldots{}\ldots{}\ldots{}\ldots{}\ldots{}\ldots{}\ldots{}\ldots{}\ldots{}\ldots{}\ldots{}\ldots{}\ldots{}\ldots{}\ldots{}\ldots{}\ldots{}\ldots{}\ldots{}\ldots{}\ldots{}\ldots{}\ldots{}\ldots{}\ldots{}\ldots{}\ldots{}\ldots{}\ldots{}\ldots{}\ldots{}\ldots{}\ldots{}\ldots{}

Data nascita
bimbo/a:\ldots{}\ldots{}../\ldots{}\ldots{}\ldots{}/\ldots{}\ldots{}\ldots{}
recapito
telefonico:\ldots{}\ldots{}\ldots{}\ldots{}\ldots{}\ldots{}\ldots{}\ldots{}\ldots{}\ldots{}..\ldots{}\ldots{}\ldots{}\ldots{}\strut
\end{minipage} & \begin{minipage}[b]{0.30\columnwidth}\raggedright
\strut
\end{minipage} & \begin{minipage}[b]{0.30\columnwidth}\raggedright
\strut
\end{minipage}\tabularnewline
\midrule
\endhead
\textbf{QUESTIONARIO ALLA DIMISSIONE data}
\ldots{}\ldots{}../\ldots{}\ldots{}\ldots{}/\ldots{}\ldots{}\ldots{} &
&\tabularnewline
1) IL BAMBINO ASSUME LATTE MATERNO? & SI & NO\tabularnewline
2) IL BAMBINO ASSUME SOLGLUCOSATA/ACQUA/SUCCHI/TISANE? & SI &
NO\tabularnewline
3) VIENE USATO IL BIBERON O ALTRO STRUMENTO PER SOMMINISTRARE IL LATTE
MATERNO? & SI & NO\tabularnewline
4) IL BAMBINO ASSUME LATTE ARTIFICIALE O CIBI SOLIDI E SEMI SOLIDI? & SI
& NO\tabularnewline
\begin{minipage}[t]{0.30\columnwidth}\raggedright
\textbf{PESO ALLA DIMISSIONE (2°-3° giornata postpartum):}
\ldots{}..\ldots{}\ldots{}.\%

calo ponderale\ldots{}\ldots{}\ldots{}\ldots{}\ldots{}..\strut
\end{minipage} & \begin{minipage}[t]{0.30\columnwidth}\raggedright
\strut
\end{minipage} & \begin{minipage}[t]{0.30\columnwidth}\raggedright
\strut
\end{minipage}\tabularnewline
\bottomrule
\end{longtable}

riferimento bibliografico: \emph{{Allattamento al seno e uso del latte
materno/umani. Position Statement 2015}} - \emph{Società Italiana
Pediatria, Neonatologia, delle Cure Primarie Pediatriche, di
Gastroenterologia Epatologia e Nutrizione Pediatrica e Società Italiana
di Medicina Perinatale}

{RACCOLTA DATI - ALLEGATO 5}

\textbf{VISITA DI CONTROLLO A 7 GIORNI DAL PARTO }

\begin{longtable}[]{@{}lllll@{}}
\toprule
\begin{minipage}[b]{0.17\columnwidth}\raggedright
ID
MADRE:\ldots{}\ldots{}\ldots{}\ldots{}\ldots{}\ldots{}\ldots{}\ldots{}\ldots{}\ldots{}\ldots{}\ldots{}\ldots{}\ldots{}\ldots{}.\ldots{}\ldots{}\ldots{}\ldots{}\ldots{}\ldots{}\ldots{}\ldots{}\ldots{}\ldots{}\ldots{}\ldots{}\ldots{}\ldots{}\ldots{}\ldots{}

Data nascita
bimbo/a:\ldots{}\ldots{}../\ldots{}\ldots{}\ldots{}/\ldots{}\ldots{}\ldots{}
recapito
telefonico:\ldots{}\ldots{}\ldots{}\ldots{}\ldots{}\ldots{}\ldots{}\ldots{}\ldots{}\ldots{}..\strut
\end{minipage} & \begin{minipage}[b]{0.17\columnwidth}\raggedright
\strut
\end{minipage} & \begin{minipage}[b]{0.17\columnwidth}\raggedright
\strut
\end{minipage} & \begin{minipage}[b]{0.17\columnwidth}\raggedright
\strut
\end{minipage} & \begin{minipage}[b]{0.17\columnwidth}\raggedright
\strut
\end{minipage}\tabularnewline
\midrule
\endhead
\textbf{COMPARSA MONTATA LATTEA (segnare con una croce)} & & &
&\tabularnewline
Dal giorno del parto, quando è comparsa la sensazione di arrivo del
latte (indurimento del seno, senso di gonfiore-pienezza-pesantezza del
seno, perdita di latte dal capezzolo)? & & & &\tabularnewline
1° giorno postp. & 2° giorno postp. & 3° giorno postp. & 4° giorno
postp. & 5° giorno postp.\tabularnewline
\textbf{CRESCITA PONDERALE } & & & &\tabularnewline
peso alla nascita & peso al controllo & guadagno medio giornaliero (gr)
& &\tabularnewline
\bottomrule
\end{longtable}

{\\
}

\hypertarget{risultati}{%
\section{RISULTATI}\label{risultati}}

La popolazione in studio è stata di 40 donne, reclutate presso la Sala
Parto dell'Ospedale S. Maria della Misericordia di Perugia , distribuite
in due gruppi, 20 nel gruppo di studio che è rappresentato dalle donne
che hanno richiesto la partoanalgesia (PA) durante il travaglio e 20 nel
gruppo controllo rappresentato dalle donne che non hanno fatto richiesta
di partoanalgesia (nPA).

I dati anagrafici sono riportati nella tabella sottostante (TAB.1)

\begin{longtable}[]{@{}lll@{}}
\toprule
\textbf{DATI ANAGRAFICI} & \textbf{Gruppo nPA (valori medi)} &
\textbf{Gruppo PA (valori medi)}\tabularnewline
\midrule
\endhead
ETÀ & 30 anni & 31 anni\tabularnewline
PESO PREGRAVIDICO & 59,7 Kg & 60,8 Kg\tabularnewline
PESO A TERMINE & 71,75 Kg & 75,1 Kg\tabularnewline
ALTEZZA & 1,66 m & 1,66 m\tabularnewline
BMI & 21,74 & 21,89\tabularnewline
GUADAGNO PONDERALE & 11,55 Kg & 14,3 Kg\tabularnewline
\bottomrule
\end{longtable}

TABELLA 1 - Dati anagrafici delle due popolazioni

L'età media e il peso pregravidico sono risultati simili nei due gruppi
e non sono state riscontrate differenze statisticamente significative.
Il peso a termine è risultato essere più alto nel gruppo di donne PA con
un valore medio di 75,1 Kg contro una valore medio del gruppo nPA di
71,75 Kg, ma i test statistici non hanno evidenziato una significatività
per questo dato.

Anche altezza e BMI o Body Mass Index, che è il valore numerico che si
ottiene dividendo il peso (espresso in Kg) per il quadrato dell'altezza
(espressa in metri), sono risultati molto vicini come valori tra i due
gruppi e non è stata evidenziata alcuna significatività statistica. Per
quanto riguarda invece il guadagno ponderale (ossia l'aumento medio di
Kg durante la gravidanza) si è evidenziata una differenza
statisticamente significativa tra i due gruppi PA (11,55 Kg) e nPA (14,3
Kg) con un p=0,04 applicando il test T- Student.

Anche per quanto riguarda la classe ASA (VI classi che indicano lo stato
di salute fisica del paziente stilate dalla American Society
Anesthesiology) non si sono evidenziate differenze significative. La
popolazione PA era suddivisa per un 50\% in classe ASA I e per l'altro
50\% in classe ASA II, mentre la popolazione nPA risultava suddivisa in
classe ASA I per il 45\% e in classe ASA II per il 55\%. Così come
simili sono le abitudini al fumo tra i due gruppi (TAB.2), anche qui
senza differenze statisticamente significative.

\begin{longtable}[]{@{}llll@{}}
\toprule
\textbf{VARIABILI} & \textbf{Gruppo nPA (\%)} & \textbf{Gruppo PA (\%)}
&\tabularnewline
\midrule
\endhead
\textbf{ASA} & \textbf{ASA I} & 45\% & 50\%\tabularnewline
& \textbf{ASA II} & 55\% & 50\%\tabularnewline
\textbf{Totale:} & \textbf{100\%} & \textbf{100\%} &\tabularnewline
\textbf{FUMO} & \textbf{NO} & 80\% & 80\%\tabularnewline
& \textbf{SI} & 10\% & 20\%\tabularnewline
& \textbf{Non risposta} & 10\% & /\tabularnewline
\textbf{Totale:} & \textbf{100\%} & \textbf{100\%} &\tabularnewline
\bottomrule
\end{longtable}

TABELLA 2 - Classe ASA e Abitudini al fumo

Per quanto riguarda il titolo di studio la maggioranza delle donne nei
due gruppi risulta laureata, 55\% nel gruppo PA e 45\% nel gruppo nPA,
poi seguono le donne con diploma di scuola superiore 35\% nel gruppo PA
e 25\% nel gruppo nPA e in ciascuno dei due gruppi il 5\% ha un diploma
di scuola inferiore, senza riscontro di differenze statisticamente
significative.

La professione più frequente nei due gruppi è quella di dipendente di
azienda pubblica 65\% nel gruppo PA e 35\% nel gruppo nPA. Alta è anche
la disoccupazione nei due gruppi, 25\% per le donne del gruppo PA e 35\%
per le donne del gruppo nPA (TAB. 3) senza riscontro di differenze
statisticamente significative.

\begin{longtable}[]{@{}llll@{}}
\toprule
\textbf{VARIABILI} & \textbf{Gruppo nPA (\%)} & \textbf{Gruppo PA (\%)}
&\tabularnewline
\midrule
\endhead
\textbf{TITOLO DI STUDIO} & \textbf{Medie inferiori} & 5\% &
5\%\tabularnewline
& \textbf{Medie superiori} & 25\% & 35\%\tabularnewline
& \textbf{Laurea} & 45\% & 55\%\tabularnewline
& \textbf{Non risposta} & 25\% & 5\%\tabularnewline
\textbf{Totale:} & \textbf{100\%} & \textbf{100\%} &\tabularnewline
\textbf{PROFESSIONE} & \textbf{Disoccupata} & 35\% & 25\%\tabularnewline
& \textbf{Impreditrice} & / & /\tabularnewline
& \textbf{Dipendente azienda pubblica} & 35\% & 60\%\tabularnewline
& \textbf{Dipendente azienda privata} & 5\% & /\tabularnewline
& \textbf{Coltivatrice diretta} & 10\% & /\tabularnewline
& \textbf{Altro} & / & /\tabularnewline
& \textbf{Non risposta} & 15\% & /\tabularnewline
\textbf{Totale:} & \textbf{100\%} & \textbf{100\%} &\tabularnewline
\bottomrule
\end{longtable}

TABELLA 3 --- Titolo di studio e Professione

La maggioranza delle donne nei due gruppi è di nazionalità Italiana
(90\% nel gruppo PA e 85\% nel gruppo nPA). Le altre provenienze sono
state da Paesi dell'est Europeo: Romania, Ucraina, Moldavia e dall'
America del sud: Ecuador, senza evidenziare differenze statisticamente
significative tra i due gruppi. Per provincia di residenza nel gruppo PA
il 70\% proviene da Perugia, il 30\% dalla provincia di PG, mentre nel
gruppo nPA le donne provenienti da Perugia sono state il 25\% mentre
dalla provincia di PG il 45\%, con una differenza statisticamente
significativa tra i due gruppi p = 0.02 (TAB.4) applicando il test di
Fisher.

\begin{longtable}[]{@{}llll@{}}
\toprule
\textbf{VARIABILI} & \textbf{Gruppo nPA (\%)} & \textbf{Gruppo PA (\%)}
&\tabularnewline
\midrule
\endhead
\textbf{NAZIONE} & \textbf{Italia} & 85\% & 90\%\tabularnewline
& \textbf{Romania} & 5\% & /\tabularnewline
& \textbf{Ecuador} & 5\% & /\tabularnewline
& \textbf{Ucraina} & 5\% & /\tabularnewline
& \textbf{Perù} & / & 5\%\tabularnewline
& \textbf{Moldavia} & / & 5\%\tabularnewline
\textbf{Totale:} & \textbf{100\%} & \textbf{100\%} &\tabularnewline
\textbf{RESIDENZA} & \textbf{Comune di Perugia} & 25\% &
70\%\tabularnewline
& \textbf{Provincia di Perugia} & 45\% & 30\%\tabularnewline
& \textbf{Altre Regioni} & 10\% & /\tabularnewline
& \textbf{Non risposta} & 20\% & /\tabularnewline
\textbf{Totale:} & \textbf{100\%} & \textbf{100\%} &\tabularnewline
\textbf{VARIABILI} & \textbf{Gruppo nPA (\%)} & \textbf{Gruppo PA (\%)}
&\tabularnewline
\textbf{ANNI DI RESIDENZA SE IMMIGRATA} & \textbf{13,5} & \textbf{14,5}
&\tabularnewline
\bottomrule
\end{longtable}

TABELLA 4 --- Nazionalità, Residenza e anni di residenza se immigrate

Per i dati raccolti in sala parto non sono state individuate differenze
statisticamente significative tra i due gruppi PA e nPA sia per quanto
riguarda la settimana di gestazione al momento in cui è avvenuto il
parto, sia per il verificarsi della rottura prematura delle membrane,
che è avvenuta con una frequenza del 40\% nel gruppo PA e del 25\% nel
gruppo nPA (TAB. 5).

\begin{longtable}[]{@{}llll@{}}
\toprule
\textbf{DATI SALA PARTO} & \textbf{Gruppo nPA (\%)} & \textbf{Gruppo PA
(\%)} &\tabularnewline
\midrule
\endhead
\textbf{settimana DI GESTAZIONE (valore medio)} & \textbf{37° settimana}
& / & 10\%\tabularnewline
& \textbf{38° settimana} & 5\% & 20\%\tabularnewline
& \textbf{39° settimana} & 25\% & 20\%\tabularnewline
& \textbf{40° settimana} & 45\% & 40\%\tabularnewline
& \textbf{\textgreater{} 40° settimana} & 25\% & 20\%\tabularnewline
\textbf{Totale:} & \textbf{100\%} & \textbf{100\%} &\tabularnewline
\textbf{PROM} & \textbf{NO} & 75\% & 60\%\tabularnewline
& \textbf{SI} & 25\% & 40\%\tabularnewline
\textbf{Totale:} & \textbf{100\%} & \textbf{100\%} &\tabularnewline
\bottomrule
\end{longtable}

TABELLA 5 --- Settimana di gestazione al momento del parto e PROM

Le donne del gruppo PA hanno effettuato induzione con Prostaglandine nel
20\% dei casi, mentre le donne del gruppo nPA nel 15\%. L'Ossitocina è
stata somministrata nel 55\% dei casi nel gruppo PA e nel 20\% dei casi
nel gruppo nPA con una differenza risultata statisticamente
significativa applicando il test del chi quadrato p=0.01. Il valore
medio della dose totale di Ossitocina somministrata è stata di 1.7 U.I.
nel gruppo PA e 1.2 U.I. nel gruppo nPA con riscontro di una differenza
statisticamente significativa p=0.01(TAB.6) applicando il test
T-Student.

\begin{longtable}[]{@{}llll@{}}
\toprule
\textbf{DATI SALA PARTO} & \textbf{Gruppo nPA (\%)} & \textbf{Gruppo PA
(\%)} &\tabularnewline
\midrule
\endhead
\textbf{INDUZIONE CON PG} & \textbf{NO} & 80\% & 75\%\tabularnewline
& \textbf{SI} & 15\% & 20\%\tabularnewline
& \textbf{Non risposta} & 5\% & 5\%\tabularnewline
\textbf{Totale:} & \textbf{100\%} & \textbf{100\%} &\tabularnewline
\textbf{OSSITOCINA} & \textbf{NO} & 80\% & 45\%\tabularnewline
& \textbf{SI} & 20\% & 55\%\tabularnewline
\textbf{Totale:} & \textbf{100\%} & \textbf{100\%} &\tabularnewline
\textbf{DATI SALA PARTO} & \textbf{Gruppo nPA (\%)} & \textbf{Gruppo PA
(\%)} &\tabularnewline
\textbf{DOSE TOT OXI (UI) (valore medio)} & 1,2 & 1,7 &\tabularnewline
\bottomrule
\end{longtable}

TABELLA 6 - Introduzione con PG, Ossitocina, dose tot. Ossitocina

Il tipo di parto è stato vaginale spontaneo per 1'85\% delle donne nel
gruppo PA e per 1'80\% delle donne nel gruppo nPA. Mentre il parto
strumentale (con Kiwi) è avvenuto nel 15\% delle donne del gruppo PA e
nel 20\% delle donne del gruppo nPA. Non si sono evidenziate differenze
statisticamente significative. Mentre l'Eisiotomia (il taglio del
perineo al momento del parto) è stata effettuata con la stessa frequenza
nei due gruppi: 30\% (TAB.7)

\begin{longtable}[]{@{}llll@{}}
\toprule
\textbf{DATI SALA PARTO} & \textbf{Gruppo nPA (\%)} & \textbf{Gruppo PA
(\%)} &\tabularnewline
\midrule
\endhead
\textbf{TIPO PARTO} & \textbf{Vaginale spontaneo} & 80\% &
85\%\tabularnewline
& \textbf{Vaginale strumentale} & 20\% & 15\%\tabularnewline
\textbf{Totale:} & \textbf{100\%} & \textbf{100\%} &\tabularnewline
\textbf{EPISIOTOMIA} & \textbf{NO} & 65\% & 70\%\tabularnewline
& \textbf{SI} & 30\% & 30\%\tabularnewline
& \textbf{Non risposta} & 5\% & /\tabularnewline
\textbf{Totale:} & \textbf{100\%} & \textbf{100\%} &\tabularnewline
\bottomrule
\end{longtable}

Totale: TABELLA 7 - Tipo di parto ed Episiotomia

Il peso neonatale medio nei due gruppi è stato di 3.44 Kg per le donne
del gruppo PA e di 3.47 Kg per le donne del gruppo nPA senza differenze
statisticamente significative tra i due gruppi. L'APGAR medio al 1°
minuto per il gruppo PA è stato di 9.3, per il gruppo nPA di 9.4, senza
differenze statisticamente significative tra i due gruppi. Mentre
l'APGAR al 5° minuto per il gruppo PA è stato di 9.95, per il gruppo nPA
9.85 con una differenza risultata statisticamente significativa p=0.01
applicando il test T-student. Il pH neonatale è stato di 7.26 nel gruppo
PA e 7.23 nel gruppo nPA (TAB 8) senza differenze statisticamente
significative.

\begin{longtable}[]{@{}lll@{}}
\toprule
\textbf{DATI SALA PARTO} & \textbf{Gruppo nPA (valori medi)} &
\textbf{Gruppo PA (valori medi)}\tabularnewline
\midrule
\endhead
\textbf{PESO NEONATO} & 3,47 Kg\% & 3,44 Kg\tabularnewline
\textbf{APGAR al 1° minuto} & 9,3 & 9,4\tabularnewline
\textbf{APGAR al 5° minuto} & 9,85 & 9,95\tabularnewline
\textbf{pH NEONATO} & 7,23 & 7,26\tabularnewline
\bottomrule
\end{longtable}

TABELLA 8 - Peso, APGAR, pH neonatali

Confrontando la durata delle varie fasi del travaglio tra i due gruppi
non sono emerse differenze statisticamente significative (TAB 9)

\begin{longtable}[]{@{}lll@{}}
\toprule
\textbf{TEMPI TRAVAGLIO} & \textbf{Gruppo nPA (valori medi)} &
\textbf{Gruppo PA (valori medi)}\tabularnewline
\midrule
\endhead
\textbf{FASE DILATANTE} & 4h 50m & 6h 4m\tabularnewline
\textbf{FASE ESPULSIVA} & l h 29m & 1h 6m\tabularnewline
\textbf{DURATA TOTALE TRAVAGLIO} & 6h 28m & 7h 23m\tabularnewline
\bottomrule
\end{longtable}

TABELLA 9 - Tempi travaglio

Lo Skin-to-Skin (il momento in cui il neonato nudo viene messo sul petto
nudo della madre dopo essere stato sottoposto a visita dal neonatologo o
dall'ostetrica) è stato effettuato nel 90\% dei casi nel gruppo PA con
frequenza maggiore entro i 10 minuti dal parto e nell'80\% dei casi nel
gruppo nPA con frequenza maggiore sempre entro i 10 minuti dal parto. La
durata è stata più breve per la maggior parte delle donne nel gruppo PA
(il 35\% ha effettuato lo skin-to-skin entro 10-30 minuti) e più lunga
nel gruppo nPA (il 45\% ha effettuato lo skin-to-skin per 30-60 minuti)
senza comunque evidenziare differenze statisticamente significative. Il
primo attacco al seno (il bambino cerca l'areola e si attacca) è stato
effettuato entro 1 h dal parto per l'80\% delle donne del gruppo PA e
per il 70\% delle donne del gruppo nPA. Dopo 2h dal parto lo hanno
effettuato solo le donne del gruppo PA (15\%) (TAB.10), ma dal punto di
vista statistico non è stato significativo.

\begin{longtable}[]{@{}llll@{}}
\toprule
\textbf{SKIN-TO-SKIN} & \textbf{Gruppo nPA (\%)} & \textbf{Gruppo PA
(\%)} &\tabularnewline
\midrule
\endhead
\textbf{ENTRO QUANTO TEMPO E' STATO EFFETTUATO} & \textbf{\textless{} 10
minuti} & 40\% & 45\%\tabularnewline
& \textbf{10 --- 30 minuti} & 35\% & 35\%\tabularnewline
& \textbf{dopo 30 minuti} & 5\% & 10\%\tabularnewline
& \textbf{Non effettuato} & 20\% & 10\%\tabularnewline
\textbf{Totale:} & \textbf{100\%} & \textbf{100\%} &\tabularnewline
\textbf{DURATA SKIN-TO-SKIN} & \textbf{10 --- 30 minuti} & 15\% &
35\%\tabularnewline
& \textbf{30-60 minuti} & 45\% & 20\%\tabularnewline
& \textbf{\textgreater{} 60 minuti} & 10\% & 25\%\tabularnewline
& \textbf{\textgreater{} 120 minuti} & 30\% & 20\%\tabularnewline
\textbf{Totale:} & \textbf{100\%} & \textbf{100\%} &\tabularnewline
\textbf{PRIMO ATTACCO AL SENO} & \textbf{Entro 1 h dal parto} & 70\% &
80\%\tabularnewline
& \textbf{Tra 1 e 2h dal parto} & 20\% & 5\%\tabularnewline
& \textbf{Dopo 2h dal parto} & \textbf{/} & \textbf{15\%}\tabularnewline
& \textbf{Non pervenuto} & 10\% & \textbf{/}\tabularnewline
\textbf{Totale:} & \textbf{100\%} & \textbf{100\%} &\tabularnewline
\bottomrule
\end{longtable}

TABELLA 10 --- Skin-to-skin

Il punteggio PIBBS totale misurato a 12h post partum ha evidenziato
differenze statisticamente significative tra i due gruppi PA e nPA con
un p=0.04 (TAB 11) applicando il test T-student.

\begin{longtable}[]{@{}llll@{}}
\toprule
\textbf{PIBBS a 12h} & \textbf{Gruppo nPA (\%)} & \textbf{Gruppo PA
(\%)} &\tabularnewline
\midrule
\endhead
\textbf{RIFLESSO DI RICERCA} & \textbf{0} & 20\% & 5\%\tabularnewline
& \textbf{1} & 35\% & 50\%\tabularnewline
& \textbf{2} & 40\% & 45\%\tabularnewline
& \textbf{Non pervenuto} & 5\% & /\tabularnewline
\textbf{Totale:} & \textbf{100\%} & \textbf{100\%} &\tabularnewline
\textbf{CAPACITA' DI AFFERRARE L'AREOLA} & \textbf{0} & 35\% &
10\%\tabularnewline
& \textbf{1} & 5\% & 10\%\tabularnewline
& \textbf{2} & 5\% & 25\%\tabularnewline
& \textbf{3} & 45\% & 55\%\tabularnewline
& \textbf{Non pervenuto} & 10\% & /\tabularnewline
\textbf{Totale:} & \textbf{100\%} & \textbf{100\%} &\tabularnewline
\textbf{TEMPO DI ATTACCO (latch on)} & \textbf{0} & 25\% &
5\%\tabularnewline
& \textbf{1} & 20\% & 25\%\tabularnewline
& \textbf{2} & 50\% & 70\%\tabularnewline
& \textbf{Non pervenuto} & 5\% & \textbf{/}\tabularnewline
\textbf{Totale:} & \textbf{100\%} & \textbf{100\%} &\tabularnewline
\textbf{TEMPO DI ATTACCO (latch on)} & \textbf{\textless{} 5 minuti} &
10\% & 20\%\tabularnewline
& \textbf{5-10 minuti} & 10\% & 20\%\tabularnewline
& \textbf{10-15 minuti} & 15\% & 15\%\tabularnewline
& \textbf{Non pervenuto} & 65\% & 45\%\tabularnewline
\textbf{Totale:} & \textbf{100\%} & \textbf{100\%} &\tabularnewline
\textbf{EFFICACIA DELLA SOLUZIONE} & \textbf{0} & 20\% &
10\%\tabularnewline
& \textbf{1} & 10\% & 15\%\tabularnewline
& \textbf{2} & 30\% & 50\%\tabularnewline
& \textbf{3} & 25\% & 25\%\tabularnewline
& \textbf{4} & 10\% & 10\%\tabularnewline
& \textbf{Non pervenuto} & 5\% & /\tabularnewline
\textbf{Totale:} & \textbf{100\%} & \textbf{100\%} &\tabularnewline
\textbf{SEQUENZA PIU' LUNGA DI SUZIONI CONSECUTIVE} & \textbf{0} & 60\%
& 55\%\tabularnewline
& \textbf{1} & 10\% & 15\%\tabularnewline
& \textbf{2} & / & 15\%\tabularnewline
& \textbf{Non pervenuto} & 30\% & 15\%\tabularnewline
\textbf{Totale:} & \textbf{100\%} & \textbf{100\%} &\tabularnewline
\textbf{DEGLUTIZIONE} & \textbf{0} & 80\% & 85\%\tabularnewline
& \textbf{1} & 15\% & 15\%\tabularnewline
& \textbf{Non pervenuto} & 5\% & \textbf{/}\tabularnewline
\textbf{Totale:} & \textbf{100\%} & \textbf{100\%} &\tabularnewline
\textbf{PIBBS a 12h} & \textbf{Gruppo nPA (\%)} & \textbf{Gruppo PA
(\%)} &\tabularnewline
\textbf{PUNTEGGIO TOTALE} & 10,05 & 12,55 &\tabularnewline
\bottomrule
\end{longtable}

TABELLA 11 - Punteggio PIBBS a 12h dal parto

Mentre il punteggio PIBBS totale a 48h post partum non ha evidenziato
differenze statisticamente significative (TAB. 12)

\begin{longtable}[]{@{}llll@{}}
\toprule
\textbf{PIBBS a 48h} & \textbf{Gruppo nPA (\%)} & \textbf{Gruppo PA
(\%)} &\tabularnewline
\midrule
\endhead
\textbf{RIFLESSO DI RICERCA} & \textbf{0} & / & /\tabularnewline
& \textbf{1} & 15\% & /\tabularnewline
& \textbf{2} & 85\% & 100\%\tabularnewline
& \textbf{Non pervenuto} & / & /\tabularnewline
\textbf{Totale:} & \textbf{100\%} & \textbf{100\%} &\tabularnewline
\textbf{CAPACITA' DI AFFERRARE L'AREOLA} & \textbf{0} & 10\% &
/\tabularnewline
& \textbf{1} & / & 5\%\tabularnewline
& \textbf{2} & 20\% & 10\%\tabularnewline
& \textbf{3} & 65\% & 85\%\tabularnewline
& \textbf{Non pervenuto} & 5\% & /\tabularnewline
\textbf{Totale:} & \textbf{100\%} & \textbf{100\%} &\tabularnewline
\textbf{TEMPO DI ATTACCO (latch on)} & \textbf{0} & 5\% &
/\tabularnewline
& \textbf{1} & 15\% & /\tabularnewline
& \textbf{2} & 80\% & 100\%\tabularnewline
& \textbf{Non pervenuto} & / & \textbf{/}\tabularnewline
\textbf{Totale:} & \textbf{100\%} & \textbf{100\%} &\tabularnewline
\textbf{TEMPO DI ATTACCO (latch on)} & \textbf{\textless{} 5 minuti} & /
& /\tabularnewline
& \textbf{5-10 minuti} & 25\% & 5\%\tabularnewline
& \textbf{10-15 minuti} & 45\% & 75\%\tabularnewline
& \textbf{Non pervenuto} & 30\% & 20\%\tabularnewline
\textbf{Totale:} & \textbf{100\%} & \textbf{100\%} &\tabularnewline
\textbf{EFFICACIA DELLA SOLUZIONE} & \textbf{0} & / & /\tabularnewline
& \textbf{1} & 10\% & /\tabularnewline
& \textbf{2} & 5\% & 10\%\tabularnewline
& \textbf{3} & 65\% & 25\%\tabularnewline
& \textbf{4} & 20\% & 65\%\tabularnewline
& \textbf{Non pervenuto} & / & /\tabularnewline
\textbf{Totale:} & \textbf{100\%} & \textbf{100\%} &\tabularnewline
\textbf{SEQUENZA PIU' LUNGA DI SUZIONI CONSECUTIVE} & \textbf{0} & 5\% &
20\%\tabularnewline
& \textbf{1} & 45\% & 40\%\tabularnewline
& \textbf{2} & 20\% & 30\%\tabularnewline
& \textbf{Non pervenuto} & 30\% & 10\%\tabularnewline
\textbf{Totale:} & \textbf{100\%} & \textbf{100\%} &\tabularnewline
\textbf{DEGLUTIZIONE} & \textbf{0} & 25\% & 25\tabularnewline
& \textbf{1} & 75\% & 75\%\tabularnewline
& \textbf{Non pervenuto} & / & \textbf{/}\tabularnewline
\textbf{Totale:} & \textbf{100\%} & \textbf{100\%} &\tabularnewline
\textbf{PIBBS a 12h} & \textbf{Gruppo nPA (\%)} & \textbf{Gruppo PA
(\%)} &\tabularnewline
\textbf{PUNTEGGIO TOTALE} & 14,5 & 15,25 &\tabularnewline
\bottomrule
\end{longtable}

TABELLA 12 - Punteggio PIBBS a 48h dal parto

Abbiamo analizzato se esisteva correlazione tra dose di Fentanyl
\textless{}/= 50 gamma e \textgreater{}50 gamma e punteggio PIBBS
\textless{}/= 10 o \textgreater{}10, sia a 12h che a 48h ed abbiamo
trovato una correlazione tra dose di Fentanyl e punteggio PIBBS a 12h
applicando il test di Fisher riportando una signifícatività statistica
p=0.01.

\begin{longtable}[]{@{}lll@{}}
\toprule
PIBBS a 12h & PIBBS \textless{}/=10 & PIBBS
\textgreater{}10\tabularnewline
\midrule
\endhead
nPA & 11 & 9\tabularnewline
PA con fentanyl \textless{}/=50 gamma & 3 & 11\tabularnewline
PA con fentanyl \textgreater{}50 gamma & 0 & 4\tabularnewline
PIBBS a 48h & PIBBS \textless{}/=10 & PIBBS
\textgreater{}10\tabularnewline
nPA & 2 & 18\tabularnewline
PA con fentanyl \textless{}/=50 gamma & 1 & 13\tabularnewline
PA con fentanyl \textgreater{}50 gamma & 0 & 6\tabularnewline
\bottomrule
\end{longtable}

La percentuale di allattamento al seno al momento della dimissione è
stata del 100\% per i neonati nel gruppo PA e per il 95\% per i neonati
nel gruppo nPA, senza differenze statisticamente significative tra i due
gruppi, con una percentuale di allattamento esclusivo al seno del 55\%
per il gruppo PA e del 40\% per il gruppo nPA. Non si sono riscontrate
differenze statisticamente significative tra i due gruppi.

Nel gruppo nPA il 30\% dei neonati ha assunto glucosata come
integrazione al latte materno, mentre nel gruppo PA questo è avvenuto
solo per il 5\% dei neonati, con una significatività statistica p=0.01
evidenziata tramite test di Fisher. Anche l'uso del biberon o altri
strumenti somministrazione del latte materno è stata più alta nel gruppo
nPA (10\%) rispetto al gruppo PA (5\%) riscontrando una significatività
statistica p=0.004. Alla dimissione assumeva latte artificiale il 40\%
dei neonati di madri sottoposte a PA e il 55\% dei neonati delle madri
del gruppo nPA (TAB. 13) senza differenze statisticamente significative
tra i due gruppi.

\begin{longtable}[]{@{}llll@{}}
\toprule
\textbf{ALLATTAMENTO ALLA DIMISSIONE} & \textbf{Gruppo nPA (\%)} &
\textbf{Gruppo PA (\%)} &\tabularnewline
\midrule
\endhead
\textbf{ASSUNZIONE LATTE MATERNO} & \textbf{NO} & 5\% & /\tabularnewline
& \textbf{SI} & 95\% & 100\%\tabularnewline
\textbf{Totale:} & \textbf{100\%} & \textbf{100\%} &\tabularnewline
\textbf{Dei SI quelli che hanno preso SOLO LATTE MATERNO} &
\textbf{40\%} & \textbf{55\%} &\tabularnewline
\textbf{ASSUNZIONE GLUCOSATA} & \textbf{NO} & 70\% & 95\%\tabularnewline
& \textbf{SI} & 30\% & 5\%\tabularnewline
\textbf{Totale:} & \textbf{100\%} & \textbf{100\%} &\tabularnewline
\textbf{ASSUNZIONE LATTE ARTIFICIALE} & \textbf{NO} & 45\% &
60\%\tabularnewline
& \textbf{SI} & 55\% & 40\%\tabularnewline
\textbf{Totale:} & \textbf{100\%} & \textbf{100\%} &\tabularnewline
\textbf{USO BIBERON / ALTRO STRUMENTO} & \textbf{NO} & 90\% &
95\%\tabularnewline
& \textbf{SI} & 10\% & 5\%\tabularnewline
\textbf{Totale:} & 100\% & 100\% &\tabularnewline
\bottomrule
\end{longtable}

TABELLA 13 --- Tipo di allattamento alla dimissione

Il peso neonatale medio alla dimissione era più alto nel gruppo PA
(3.262 Kg) che nel gruppo nPA (2.904 Kg) con riscontro di una
significatività statistica p=0.01 mediante test T-student.

II calo ponderale medio è stato mediamente più elevato (6,14\%) nei
neonati di madri non sottoposte a PA che nei neonati di madri sottoposte
a PA (6\%) con riscontro di una significatività statistica p=0.01 (TAB.
14) mediante il test T-student.

\begin{longtable}[]{@{}lll@{}}
\toprule
VARIABILI & \textbf{Gruppo nPA (valori medi)} & \textbf{Gruppo PA
(valori medi)}\tabularnewline
\midrule
\endhead
\textbf{PESO NEONATO ALLA DIMISSIONE} & 2,904 Kg & 3,262
Kg\tabularnewline
\textbf{CALO PONDERALE} & 6.14\% & 6\%\tabularnewline
\bottomrule
\end{longtable}

TABELLA 14 - Peso neonatale alla dimissione e calo ponderale

La montata lattea è avvenuta entro le 36h postpartum nel 70\% delle
donne del gruppo PA e nel 45\% delle donne del gruppo nPA, senza
riscontro di differenze statisticamente significative. Mentre oltre le
36h è avvenuta nel 30\% delle donne del gruppo PA e nel 55\% delle donne
del gruppo nPA (TAB. 15) senza riscontro di differenze statisticamente
significative tra i due gruppi.

\begin{longtable}[]{@{}llll@{}}
\toprule
\textbf{VISITA DI CONTROLLO A 7 GIORNI DAL PARTO} & \textbf{Gruppo nPA
(valori medi)} & \textbf{Gruppo PA (valori medi)} &\tabularnewline
\midrule
\endhead
\textbf{COMPARSA MONTATA LATTEA} & 1° giorno postpartum & 5\% &
10\%\tabularnewline
& 2° giorno postpartum & 5\% & 20\%\tabularnewline
& 3° giorno postpartum & 35\% & 40\%\tabularnewline
& 4° giorno postpartum & 10\% & 15\%\tabularnewline
& oltre 4° giorno postpartum & 45\% & 15\%\tabularnewline
\textbf{Totale:} & \textbf{100\%} & \textbf{100\%} &\tabularnewline
\bottomrule
\end{longtable}

TABELLA 15 - Montata Lattea

Il peso neonatale medio a 7 giorni (rilevato mediante intervista
telefonica alla madre) è stato di 3.471 Kg nel gruppo PA e di 3.437 Kg
nel gruppo nPA, senza riscontro di differenze statisticamente
significative. Anche il guadagno ponderale medio giornaliero è stato di
46.45 gr nei neonati del gruppo PA e di 45.36 gr nei neonati del gruppo
nPA (TAB 16), senza differenze statisticamente significative tra i due
gruppi.

\begin{longtable}[]{@{}lll@{}}
\toprule
\textbf{VARIABILI} & \textbf{Gruppo nPA (valori medi)} & \textbf{Gruppo
PA (valori medi)}\tabularnewline
\midrule
\endhead
\textbf{PESO NEONATALE ALLA DIMISSIONE} & 2,904 Kg & 3,262
Kg\tabularnewline
\textbf{PESO NEONATALE AL CONTROLLO} & 3,437 Kg & 3,471
Kg\tabularnewline
\textbf{GUADAGNO PONDERALE MEDIO GIORNALIERO} & 45.36 gr & 46.45
gr\tabularnewline
\bottomrule
\end{longtable}

TABELLA 16 - Peso neonatale al controllo e guadagno medio giornaliero

Nel gruppo delle donne che hanno richiesto la partoanalgesia (PA) la
tecnica di partoanalgesia è stata per tutte quella Epidurale con
cateterino, la dose totale media di Fentanyl somministrata è stata di 59
rnicrogrammi, la dose totale media di levobupivacaina è stata di 22.99
mg. La dilatazione cervicale media al momento del posizionamento del
caterere peridurale era di 5.45 cm. La percentuale di donne che ha
ricevuto ossitocina è stata del 55\% rispetto alle donne del gruppo nPA
in cui è stata del 20\%, con riscontro di differenza statisticamente
significativa, come abbiamo visto sopra, applicanto il test Chi quadrato
p=0.01. Solo il 15\% ha sviluppato brivido post PA e la temperatura
media è stata di 36.3°C (TAB 17), senza riscontro di febbre o
iperpiressia in nessuna delle partorienti di entrambi i gruppi.

\begin{longtable}[]{@{}ll@{}}
\toprule
\textbf{VARIABILI} & \textbf{Gruppo PA}\tabularnewline
\midrule
\endhead
\textbf{DOSE TOT. FENTANYL (valore medio)} & 59 mcg\tabularnewline
\textbf{DOSE TOT. LEVOBUPIVACAINA (valore medio)} & 22,99
mg\tabularnewline
\textbf{TECNICA DI PARTOANALGESIA} & 100\% EPIDURALE\tabularnewline
\textbf{DILATAZIONE AL POSIZIONAMENTO (valore medio)} & 5,45
cm\tabularnewline
\textbf{\% DONNE CHE HA RICEVUTO OX PRE PA} & 25\%\tabularnewline
\textbf{\% DONNE CHE HA RICEVUTO OX POST PA} & 50\%\tabularnewline
\textbf{BRIVIDO POST PA} & 15\%\tabularnewline
\textbf{TEMPERATURA POST PA (valore medio)} & 36,3 °C\tabularnewline
\bottomrule
\end{longtable}

TABELLA 17 --- Dati donne sottoposte a partonalgesia

\hypertarget{discussione-e-conclusioni}{%
\section{DISCUSSIONE E CONCLUSIONI}\label{discussione-e-conclusioni}}

La necessità di ulteriori studi sull'impatto della analgesia epidurale
in corso di travaglio di parto sull'allattamento è stata inclusa nelle
linee guida commissionate dal NICE (National Institute for Healt and
Clinical Excellence) già nel 2007 quindi più di dieci anni fa.
\textsuperscript{(1)} Attualmente varie tipologie di farmaci sono
somministrati di routine durante il travaglio e il parto con indicazioni
diverse e rientrano nelle categorie farmacologiche degli uterotonici per
l'induzione del travaglio e/o per l'aumento dell'intensità delle
contrazioni uterine (prostaglandine e ossitocina) oppure per la
prevenzione dell'emorragia post partum (ossitocina e metilergometrina),
degli analgesici per il controllo del dolore (oppiacei ed anestetici
locali sia per via endovenosa/intramuscolo che per via peridurale).

Ci sono pochi trials clinici che mettono in relazione l'uso intrapartum
di farmaci e l'allattamento come out-come primario, ma vari studi con
dati retrospettivi offrono una alternativa per cercare di capire
l'interazione tra questi e l'allattamento.\textsuperscript{(2)}
L'analgesia neurassiale con oppioidi è utilizzata per la maggior parte
dei parti negli Stati Uniti e nel resto dei Paesi industrializzati e il
tasso di donne che richiede tale tipo di analgesia durante il parto sta
aumentando.\textsuperscript{(4)} L'aggiunta di oppioidi alle soluzioni
di anestetico locale somministrate per via epidurale presenta diversi
vantaggi, il principale è che determinano un effetto sinergico
sull'analgesia riducendo il rischio di blocco motorio, la cui intensità
è direttamente correlata con la dose di anestetico locale somministrata
ed è causa di discomfort per le partorienti perché ne riduce la
mobilità. \textsuperscript{(5;6) }

L'analgesia epidurale a causa del passaggio dell'oppiaceo dal circolo
materno a quello fetale attraverso la placenta, potrebbe deprimere il
SNC del bambino e quindi compromettere l'allattamento andando ad
alterare i parametri neurocomportamentali del neonato indispensabili per
nutrirsi dal seno materno, come la suzione, la ricerca del capezzolo e
la deglutizione, durante l'immediato post-partum. \textsuperscript{(7;
8)} Durante questo periodo critico madre e figlio fanno i loro primi
tentativi di allattamento e stabiliscono un precedente per le successive
interazioni. Il comportamento nutrizionale del neonato nel primo periodo
post-partum è un importante fattore predittivo di successo a lungo
termine dell'allattamento. Quei bambini che mostrano comportamento
nutritivi più vigorosi durante il primo giorno di vita hanno una
maggiore probabilità di essere allattati al seno a 3 e 6 mesi di vita,
rispetto a quelli che mostrano comportamenti meno
vigorosi.\textsuperscript{(9)} Diversi strumenti e score sono stati
utilizzati in diversi studi per valutare il comportamento neonatale e
l'interazione madre-figlio durante l'allattamento. Tra questi i più
significativi sono stati il NACS score ed il PIBBS score. Il NACS score,
ad esempio, non misura specificamente i comportamenti nutritivi, ma
analizza 5 aree generali: capacità adattative del neonato, tono passivo,
tono attivo, riflessi primari e attività di allerta/pianto/attività
motoria. La scala PIBBS (the Preterm Infant Breastfeeding Behavior
Scale) utilizzata nel nostro studio, invece, esamina nello specifico il
comportamento di ricerca, di attacco al capezzolo, di suzione e di
deglutizione, e il livello generale di attività del bambino, che portano
a definire in modo più completo l'interazione tra madre e figlio nel
processo dell'allattamento. \textsuperscript{(10)}

Nel 2016 una review sistematica ha incluso 23 studi che indagavano
l'associazione tra l'analgesia neuroassiale per il travaglio di parto e
l'allattamento come outcome. \textsuperscript{(11)} I risultati sono
stati conflittuali: metà degli studi non hanno trovato associazioni tra
analgesia epidurale e allattamento, l'altra metà ha identificato
associazioni negative ed un solo studio ha individuato una associazione
positiva. La maggior parte di questi erano però di natura
osservazionale, soltanto 3 erano trials clinici randomizzati
controllati, quindi una possibile spiegazione per questi risultati
conflittuali può essere che molti studi non erano controllati per le
variabili di confondimento già conosciute per influenzare il successo
dell'allattamento, altri erano sottodimensionati e la gestione
dell'analgesia epidurale era differente tra gli studi.
\textsuperscript{(36; 12)}

Dei 3 trials clinici randomizzati controllati due hanno esaminatogli
effetti del fentanyl epidurale sull'allattamento riportando risultati
opposti. Beilin \textsuperscript{(13)} nel suo studio riporta che le
madri randomizzate a ricevere una dose totale di Fentanyl
\textgreater{}/= 150 microgrammi erano più soggette a interrompere
l'allattamento 6 settimane dopo il parto (tasso di interruzione
dell'allattamento del 17\%) se comparate a donne che o non avevano
ricevuto Fentanyl (tasso del 2\%) oppure lo avevano ricevuto con un
dosaggio inferiore a 150 microgrammi (tasso del 5\%). L'outcome primario
dello studio era la "difficoltà di allattamento" (nessuna, lieve,
moderata, severa) rilevato dalla madre e dagli operatori durante il
primo giorno postpartum, utilizzando come scala di valutazione
neurocomportamentale del neonato il NACS score a 2 e a 24h post partum.
Questo risultava essere significativamente più basso nei bambini del
gruppo partoanalgesia con dosaggio del Fentanyl totale somministrato
\textgreater{}150 gamma.

Al contrario Wilson \textsuperscript{(4)} in una analisi secondaria di
un grande trial clinico randomizzato, lo studio COMET, conclude che
l'analgesia neuroassiale, indipendentemente dalla somministrazione o
meno di fentanyl epidurale, non ostacola l'allattamento fino a 12 mesi
dopo il parto. Anche in questo studio le donne sono state randomizzate
in 3 gruppi che hanno ricevuto diverse tecniche di analgesia neurassiale
(il primo gruppo solo Bupivacaina per via epidurale; il secondo e il
terzo gruppo sia analgesia epidurale che combinata con Bupivacaina e
Fentanyl) e messi a confronto con un quarto gruppo che o non ha ricevuto
analgesia oppure ha ricevuto analgesia sistemica con oppioidi
somministrati per via endovenosa (Petidina). La dose cumulativa media
dei due gruppi epidurale e combinata è stata rispettivamente di 163
microgrammi e 107 microgrammi. Il gruppo di controllo che aveva ricevuto
Meperidina ev aveva un tasso di successo di inizio dell'allattamento più
basso rispetto ai gruppi che non avevano ricevuto analgesia o avevano
ricevuto analgesia per via neurassiale. La durata totale
dell'allattamento, indagata con un questionario spedito a casa, è stata
in media di 15 settimane e non si sono rilevate differenze tra i gruppi
analgesia neurassiale e quello dl controllo nell'allattamento ad un anno
dal parto. La dose media totale massima somministrata in questo studio è
stata di 139 microgrammi, mentre in quello di Beilin è stata di 200
microgrammi. In questo studio il tasso di donne che ancora allattavano a
6 settimane dal parto tra le donne che hanno ricevuto meno di 150 gamma
di fentanyl è stato del 96.4\% se comparato con quello delle donne che
hanno ricevuto più di 150 gamma di fentanyl, che era del 96.7\% e il
tasso di abbandono dell'allattamento è stato del 10.2\% nel gruppo
\textgreater{}/=150 gamma e del 6.5\% nel gruppo che ha ricevuto
\textless{}150 gamma. Lo studio quindi ha concluso che non c'erano
differenze statisticamente significative nei tassi di abbandono
dell'allattamento tra il gruppo di donne che aveva ricevuto oppioidi per
via neurassiale e quelle che non avevano ricevuto oppioidi. Anche uno
studio successivo del 2017 di Lee et al. \textsuperscript{(15)} che
comparava gli effetti di tre differenti soluzioni utilizzate per
l'analgesia epidurale in travaglio di parto somministrate in tre gruppi
diversi di donne (rispettivamente Bupivacaina 1mg/ml; Bupivacaina 0.8
mg/ml + Fentanyl 1 mcg/ml; e Bupivacaina 0.625 mg/ml + Fentanyl 2
mcg/ml) mostrava che il tasso di insuccesso dell'allattamento a 6
settimane dal parto, inteso come sospensione dello stesso per un
qualsiasi motivo, era rispettivamente del 2\% nel primo gruppo (solo
Bupi), del 5\% nel secondo gruppo (Bupi + Fentanyl 1 mcg/ml) e del 6\%
nel terzo gruppo (Bupi + Fentanyl 2 mcg/ml) e che, quindi, non era
influenzato né dalla concentrazione di Fentanyl somministrata nei
singoli boli epidurali né dalla dose totale di Fentanyl somministrata
durante tutto il travaglio. Lo studio inoltre analizzava peso neonatale,
EGA sul sangue cordonale, l'incidenza di APGAR score al 1° minuto
\textless{}7 e il tasso di ammissione dei neonati in UTIN senza rilevare
differenze significative tra i due gruppi. Anche il LATCH score e la
durata dello skin-to-skin nelle prime 24h tra i due gruppi risultavano
simili. Le concentrazioni di Fentanyl e Bupivacaina sia nel sangue
materno che nel sangue venoso del cordone ombelicale e la dose
cumulativa di Fentanyl non determinavano differenze nella sospensione
dell'allattamento a 6 settimane e a 3 mesi dal parto.

Fentanyl è stata la molecola maggiormente presa in causa per la sua
elevata lipofilicità e quindi per la sua tendenza ad accumularsi nei
tessuti lipofili del neonato e per la scarsa capacità neonatale di
eliminarlo, vista l'immaturità dei sistemi enzimatici epatici coinvolti
nel suo metabolismo.

Anche nel nostro studio si è evidenziato che la somministrazione
peridurale di oppiacei (nel nostro protocollo in particolare è stato
utilizzato il Fentanyl, lasciando al singolo professionista la
possibilità di scelta per quanto riguarda la dose totale da
somministrare, non ha mai superato i 110 mcg, tra l'altro raggiunti in
una sola paziente con un dosaggio medio somministrato di 59 microgrammi)
non determina una riduzione degli score neurocomportamentali neonatali
(PIBBS score in particolare utilizzato nel nostro studio). Anzi i
punteggi a 12h, quindi quelli più vicini alla somministrazione
peridurale di Fentanyl nella madre, quindi quelli rilevati in tempi
vicini all'eventuale picco plasmatico nel bambino, sono risultati
mediamente più alti nel gruppo di neonati nati da donne sottoposte a PA
(punteggio PIBBS medio a 12h: 12,5) rispetto al gruppo di neonati nati
da donne non sottoposte a PA (punteggio PIBBS medio a 12h: 10,05) con un
p=0.04 applicando il test del Chi quadrato.

Una dose totale di Fentanyl al di sotto dei 100 mcg nel nostro studio si
è comunque associata a un soddisfacente controllo del dolore, tant'è che
il personale anestesiologico impegnato nel servizio di partoanalgesia
non ha ritenuto necessario soministrare dosi più elevate. Questo
potrebbe deporre a favore dell'ipotesi che l'analgesia epidurale per il
travaglio di parto non inibisca l'allattamento al seno, ma anzi
riducendo lo stress materno-fetale potrebbe migliorare le condizioni di
interazione post-partum tra madre e figlio. Le migliori condizioni sia
fisiche che psicologiche apportate alla madre dall'analgesia durante il
travaglio e il parto, potrebbero predisporla a volere un contatto più
precoce con il neonato. Neonato che presenta un comportamento più
attivo, probabile conseguenza della riduzione dello stress fetale
durante il travaglio e il parto, con la conseguenza che quest'ultimo
mette in atto e sviluppa prima comportamenti di ricerca del seno materno
e di attacco. La letteratura riporta che questi comportamenti si
presentano già dopo 50 minuti dalla nascita. E si sviluppano tanto più
velocemente quanto più precoce è il contatto madre-figlio.
\textsuperscript{(16: 17; 18; 19; 20)} .

Un altro dato interessante si è presentato mettendo in correlazione il
punteggio PIBBS a 12h dalla nascita nei due gruppi Pa, nPA con il
dosaggio del Fentanyl somministrato per via peridurale.

\begin{longtable}[]{@{}lll@{}}
\toprule
PIBBS a 12h & PIBBS \textless{}/=10 & PIBBS
\textgreater{}10\tabularnewline
\midrule
\endhead
nPA & 11 & 9\tabularnewline
PA con Fentanyl \textless{}/=50 gamma & 3 & 11\tabularnewline
PA con Fentanyl \textgreater{}50 gamma & 0 & 4\tabularnewline
\bottomrule
\end{longtable}

In effetti è stato riscontrato che esiste una correlazione positiva tra
dosaggio dell'oppiaceo somministrato in peridurale e riduzione del
punteggio neonatale a 12h nel gruppo PA, soprattutto per dosaggi di
Fentanyl \textgreater{}50 gamma, con un valore di p applicando il test
del Chi quadrato per un grado di libertà risulta pari a p=0.005.

Quindi anche se i punteggi totali del PIBBs a 12h risultano migliori nei
bambini nati da madri sottoposte a PA, all'interno di questo gruppo, i
punteggi totali del PIBBs risultano mediamente più bassi nei bambini
nati da madri che hanno ricevuto più di 50 gamma di Fentanyl in
peridurale, rispetto a quelli nati da madri che hanno ricevuto dosaggi
più bassi \textless{}/= 50 gamma. Nella rilevazione successiva del
punteggio PIBBS a 48h si assiste però ad una normalizzazione dei
punteggi fra i due gruppi PA ed nPA, portando a non rilevare più
differenze statisticamente significative con un valore di p=0.2373
applicando sempre il test del Chi quadrato per un grado di libertà.

Questo potrebbe essere correlato alla curva di eliminazione del farmaco
dal parte del neonato, farmaco che comunque, al dosaggio da noi
utilizzato, sembra non aver inficiato le abilità neurocomportamentali
del neonato stesso necessarie per attaccarsi adeguatamente al seno e
nutrirsi. \textsuperscript{(15; 14; 8; 21: 32) }

\begin{longtable}[]{@{}lll@{}}
\toprule
PIBBS a 48h & PIBBS \textless{}/=10 & PIBBS
\textgreater{}10\tabularnewline
\midrule
\endhead
nPA & 2 & 18\tabularnewline
PA con fentanyl \textless{}/=50 gamma & 1 & 13\tabularnewline
PA con fentanyl \textgreater{}50 gamma & 0 & 6\tabularnewline
\bottomrule
\end{longtable}

Un altro farmaco, analogo della sua forma endogena, che può essere
somministrato durante il travaglio di parto è l'ossitocina. La
secrezione pulsatile di ossitocina endogena è molto importante per il
corretto stabilirsi dell'allattamento. Questa pulsatilità può risultare
alterata da diversi fattori e sia la somministrazione esogena di
ossitocina che lo stress e il dolore dovuti al travaglio stesso hanno
mostrato questa capacità. Anche se la partonalgesia potenzialmente
potrebbe favorire il rilascio di ossitocina grazie alla sua capacità di
ridurre lo stress e il dolore da travaglio, alcuni studi hanno
dimostrato che l'analgesia epidurale interferisce con il ciclo
dell'ossitocina bloccandone il rilascio endogeno nel SNC e quindi nella
circolazione sanguigna durante il travaglio. Probabilmente il meccanismo
coinvolto e bloccato è quello del Riflesso di Ferguson (ossia il
rilascio di ossitocina mediato dallo stiramento della cervice da parte
della testa fetale durante le contrazioni uterine del travaglio)
\textsuperscript{(3,36,37,38,39,46)}. L'effetto dell'ossitocina
sintetica sull'inizio della lattazione, sul processo di galattopoiesi e
sulla durata dell'allattamento al seno è stato approfondito in alcuni
studi \textsuperscript{(2, 40, 41, 42,43)}. Jordan et al. (2009) ha
dimostrato che l'ossitocina esogena ha un effetto ritardante sulla
lattazione posticipandone l'inizio anche a 48 ore dopo il parto
\textsuperscript{(2)}. In un lavoro retrospettivo condotto da Fortea et
al. (2014) si dimostra che l'uso dell'ossitocina aumenta del 50\% il
rischio di dover utilizzare strumenti come il biberon per somministrare
integrazioni ai neonati e che questo effetto è maggiore nelle donne
\textless{}27 anni di età (probabilmente a causa di una minore
scolarizzazione) e per neonati di età gestazionale compresa tra le 37 e
le 40 settimane \textsuperscript{(40)}. Gli effetti negativi
dell'ossitocina esogena inoltre non riguarderebbero solo l'onset della
lattazione ma anche la sua durata. Il motivo potrebbe essere che le
elevate concentrazioni di ossitocina esogena durante il travaglio vadano
ad inibire la secrezione pulsatile endogena di questa e/o ad interferire
con la disponibilità di recettori per l'ossitocina stessa durante il
periodo sensibile del primo puerperio, momento in cui iniziano a
innescarsi meccanismi neuroendocrini e psicologici attraverso i quali si
stabilisce il normale legame madre-figlio, fondamentale per la
prosecuzione non solo dell'allattamento, ma di tutti i processi di cura
verso il neonato stesso \textsuperscript{(36;37;2)}.

Altri studi dimostrano invece che l'ossitocina somministrata
esogenamente durante il travaglio può compensare la ridotta produzione
di ossitocina endogena, andando quella sintetica a determinare gli
stessi effetti di quella endogena \textsuperscript{(44;45)}.

Nel nostro studio si è evidenziata una maggiore tendenza a somministrare
ossitocina nel gruppo di donne sottoposte a PA (55\% contro il 20\%
delle donne nel gruppo nPA) rilevando una significatività statistica
mediante test del Chi quadrato con un valore di p=0.05 e con dosaggi
mediamente superiori (1.7 U.I. contro 1.2 U.I.) evidenziando anche qui
una differenza statisticamente significativa mediante test T-student con
un valore di p=0.01.

Dai dati preliminari del nostro studio comunque, considerando il ruolo
che l'ossitocina endogena riveste ai fini dell'allattamento, non abbiamo
osservato influenza negativa né della PA né dell'ossitocina esogena
sull'allattamento stesso fino ad una settimana dal parto e possiamo
presumere che la secrezione dell'ossitocina endogena non sia stato
alterata, ma non ne abbiamo la certezza perché non è stato possibile
misurarne livelli effettivi nel plasma. In particolare proprio nel
gruppo PA, dove la somministrazione di ossitocina è stata più frequente
e a dosaggi maggiori, non si si sono verificati né un ritardo
nell'insorgenza della montata lattea né una precoce sospensione
dell'allattamento ad una settimana dal parto. Dato confermato anche da
un punteggio Apgar al 5° più alto nel gruppo PA e da un punteggio PIBBS
più alto (considerando un adeguato un cut-off arbitrario di punteggio
PIBBs \textgreater{}10, che significa una sufficienza per ciascuno degli
item del PIBBs) sempre nel gruppo PA a 12h. Inoltre la necessità di
integrare l'alimentazione neonatale con glucosata o latte artificiale e
di utilizzare strumenti aggiuntivi per la somministrazione del latte,
come il biberon, è stata significativamente più bassa nei neonati del
gruppo PA (test del chi quadrato p=0.01873 per la glucosata e p= 0.004
per biberon/altri strumenti) che non per i neonati del gruppo nPA.

Al di là dello scopo principale della nostra analisi che si focalizza
sul rapporto parto-analgesia e allattamento al seno, alcuni dati ci
hanno permesso di valutare le caratteristiche della nostra popolazione
di studio che ha richiesto la parto-analgesia.

Per quanto riguarda le caratteristiche demografiche delle popolazioni in
studio non sono emerse differenze significative tra i 2 gruppi in merito
a l'età media al momento del parto, il peso pregravidico medio e il peso
medio a termine, l'altezza e il BMI (Body Mass Index). Una differenza
significativa (test T-student p\textless{}0.05) si è potuta riscontrare
invece sul valore del guadagno ponderale medio in gravidanza tra i due
gruppi: 14.3 Kg nel gruppo PA e 11.55 Kg nel gruppo nPA. Ciò potrebbe
essere in linea con quanto visto nello studio di Alehagen (2005) nel
quale viene sottolineato come i tratti di personalità pre-gravidici,
quindi i diversi profili psicologici delle donne, possono influenzare la
scelta di usufruire o meno dell'analgesia in corso di travaglio di
parto. Nel suo studio le donne che richiedevano tale procedura
risultavano essere più spaventate delle altre e meno tolleranti al
dolore del travaglio e all'evento nascita in generale. Si può
presupporre che il diverso incremento ponderale nei due gruppi sia
dovuto ad una minore tolleranza verso l'impulso della fame,
considerabile come uno "stress" fisico, così come mostrano una minore
tolleranza verso altre situazioni di stress come appunto sono quelle del
travaglio e del parto, per le donne del gruppo PA rispetto al gruppo
nPA. \textsuperscript{(47) }

Anche la zona di residenza delle donne ha influenzato la scelta di
usufruire o meno della partoanalgesia, mettendo in evidenza che le donne
residenti nel comune di Perugia, rispetto a quelle residenti nel resto
della provincia o provenienti da altre regioni, si sono avvalse di più
di questo servizio. Una possibile spiegazione è la diversa possibilità
di accesso a strutture pubbliche con diversa frequenza ai corsi di
preparazione al parto o corsi sull'allattamento, dovuta alla ridotta
distanza da casa per chi risiedeva nel comune stesso. La classe ASA e le
abitudini tabagiche non hanno influito sulla diversa richiesta di
analgesia tra i due gruppi in esame, così come il titolo di studio, la
professione, e la nazionalità (pur essendo esiguo il numero di donne non
italiane) né sull'outcome principale e sull'outcome secondario che era
quello del successo dell'allattamento al seno alla dimissione e a 7
giorni dal parto.

Non ha influito su questi outcomes nemmeno la settimana di gestazione al
parto (lo sviluppo delle capacità neurocomportamentali del neonato per
quanto riguarda l'allattamento avvengono già a partire dalla
30°settimana di gestazione) come dimostrato da vari studi su neonati
prematuri \textsuperscript{(23.24)}.

Anche il fatto che si sia verificata o meno la rottura prematura delle
membrane fetali non ha influito sugli outcomes dell'allattamento alla
dimissione e a 7 giorni dal parto. Quasi tutte le donne di entrambi i
gruppi PA e nPA allattavano al seno sia alla dimissione che al controllo
a 7 giorni. In letteratura non sono riportati dati sull'associazione tra
rottura prematura delle membrane e ritardo nell'inizio della montata
lattea o sull'associazione tra questa e il fallimento dell'allattamento
nel primo periodo post partum.

In merito alle caratteristiche del travaglio, la percentuale di parti
vaginali spontanei (85\% nel gruppo PA e 80\% nel gruppo nPA) e
strumentali (15\% nel gruppo PA e 20\% nel gruppo nPA), così come quella
di episiotomia (30\% sia per il gruppo PA che per quello nPA) non è
risultato significativamente diverso per i due gruppi della nostra
analisi. Questo dato è in contrasto con quello che viene rilevato in
letteratura, dove sia il tasso di parti operativi (PA 6.5\% vs. nPA 2.9
\%) che di episiotomia (PA 27.8\% vs. nPA 13.1\%) risulta aumentato nel
gruppo di donne che richiedono la partoanalgesia \textsuperscript{(25;
26, 27, 28; 29, 14)}. Per quanto riguarda l'influenza di queste pratiche
sull'allattamento si è visto nel nostro studio che non determinano
effetti a breve termine sull'allattamento.

Anche il valore della durata totale media del travaglio e i valori medi
di durata delle due fasi, quella dilatante e quella espulsiva, non
risultano aumentati nel gruppo PA nel nostro studio. Riguardo a questo
argomento in letteratura i dati sono contrastanti: alcuni studi
riportano un allungamento della durata totale del travaglio, sia per
allungamento della fase dilatante che di quella espulsiva
\textsuperscript{(48;49;50)}, altri invece non evidenziano differenze
significative \textsuperscript{(55; 56)} con possibile interferenza
sull'allattamento sia a breve che a lungo termine.
\textsuperscript{(48;49)}

Per quanto riguarda invece i dati neonatali una differenza significativa
(test T-student p=0.01) si è riscontrata nei valori di APGAR al 5°
minuto tra i due gruppi, risultando migliori i punteggi dei neonati da
madri sottoposte a PA (punteggio medio 9.95) rispetto a quelli nati da
madri non sottoposte a PA (punteggio medio 9.85). Questo può essere
spiegato, come da altri riscontri in letteratura, dalla riduzione dello
stress materno durante il travaglio ed il parto nelle donne sottoposte
ad analgesia peridurale che ha portato a condizioni generali migliori
nei neonati di queste donne. \textsuperscript{(51; 52; 53; 54)}
Considerando che l'indice di APGAR fornisce un'indicazione sul benessere
generale del neonato e quindi anche sulla sua reattività/adattamento
alla vita extrauterina \textsuperscript{(30)}, questo dato potrebbe
motivare il migliore punteggio della PIBB SCALE entro le prime 12 ore
dal parto per neonati da madri che hanno ricevuto la parto-analgesia e
la ridotta necessità di integrazioni al latte materno. Lo skin-to-skin,
ossia il contatto pelle a pelle tra il neonato e la madre subito dopo il
parto o nei minuti successivi è un elemento importante per lo stabilirsi
sia del legame affettivo e che per l'instaurarsi di quei processi
fisiologici che portano ad iniziare la lattazione.
\textsuperscript{(62;63)}

Se possibile primo contatto tra madre e figlio dovrebbe avvenire
nellaprima ora post partum \textsuperscript{(57)}. Il neonato, se
lasciato indisturbato, mostra un comportamento istintivo di ricerca del
capezzolo e di inizio della suzione ad approssimativamente 1 ora dal
parto. Anche la durata di questo contatto è importante, più a lungo il
neonato rimane con la madre, maggiore sarà la produzione di ossitocina
da parte della madre stessa, importante per l'istaurarsi sia della
relazione madre-figlio e di prolattina indispensabile per l'avvio
dell'allattamento \textsuperscript{(58)}. La letteratura
\textsuperscript{(33)} riporta tassi di allattamento al seno più alti
nel gruppo di donne che ha eseguito lo skin-to-skin precoce (non c'è una
definizione dei tempi dello skin-to-skin, per precoce si intende il
prima possibile) (59.6\%) rispetto a quelle che non l'hanno avuto
(45.8\%) p=0.009, e i bambini che hanno avuto uno skin-to-skin precoce
(subito dopo il parto), ininterrotto e che si sono attaccati
spontaneamente al capezzolo, hanno continuato a farlo anche in seguito e
con maggiore successo rispetto a quelli che non avevano avuto uno
skin-to-skin precoce. Inoltre sempre lo skin-to-skin sembra aumentare la
produzione di latte e il peso del neonato che si alimenta al seno.
\textsuperscript{(34;35)}

Nel nostro studio il primo contatto madre-neonato è avvenuto nella
maggioranza dei casi entro i 30 minuti dal parto, sia nel gruppo PA
(80\%) che nel gruppo nPA (75\%) senza differenze statisticamente
significative. La durata è risultata essere inferiore nel gruppo PA
(10-30 minuti) contro i 30-60 minuti del gruppo nPA, probabilmente per
la maggiore medicalizzazione del parto nel primo gruppo e quindi per una
necessità di maggiore assistenza verso la madre, che ha portato a
ridurre la durata del contatto con il neonato, senza però determinare
differenze statisticamente significative sia per quanto riguarda i tempi
di insorgenza della montata lattea sia per il tipo di allattamento alla
dimissione e a 7 giorni.

In letteratura gli studi che valutano il successo dell'attacco al seno
nella prima ora post-partum in relazione alla partoanalgesia forniscono
dati contrastanti. Baumagardner et al. (2003) nel suo studio ha
dimostrato che la partoanalgesia non influenzava la percentuale di
tentativi di allattamento nella prima ora di vita, ossia non andava ad
influenzare il comportamento di ricerca e attacco del bambino nella
prima ora dopo il parto, ma aveva poi un impatto negativo
sull'allattamento al seno nelle successive 24h di vita del neonato
(misurato come 2 sessioni di allattamento di successo misurate con il
Latch score) dimostrando che nel gruppo di donne sottoposte a PA la
percentuale di successo era del 69.6\% contro 1'81\% delle donne non
sottoposte a PA (p=0.04) con un maggior rischio per i bambini di donne
sottoposte a PA di supplementazioni al latte materno durante il ricovero
in ospedale e alla dimissione (P \textless{} .001)
\textsuperscript{(60)}. Lo studio di Wiklund del 2007 mostra che un
minor numero di bambini nati da madri sottoposte a partoanalgesia si
attaccavano al seno nelle prime 4 ore dopo il parto e questo era poi
correlato al fatto che ricevevano supplementazioni al latte materno più
frequentemente durante la permanenza in ospedale e che presentavano un
minor tasso di allattamento al seno al momento della dimissione
dall'ospedale. \textsuperscript{(59)}

Un altro studio ha valutato i comportamento neonatali utilizzando la
PIBBS scale ad 1h e a 24h dalla nascita di 56 neonati e non ha trovato
differenze tra i bambini nati da madri che hanno ricevuto analgesia
epidurale e quelle che non l'hanno ricevuta, e differenze per quanto
riguarda la supplementazione al latte materno tra i due gruppi
\textsuperscript{(22)}. Anche lo studio di Wilson del 2010 non ha messo
in evidenza differenze significative trai gruppi di donne sottoposte a
Pa con diversi dosaggi di Fentanyl e quelle che non avevano ricevuto PA,
sia per quanto riguarda i tassi di inizio che di durata
dell'allattamento. Un ritardo nell'inizio dell'allattamento era
riscontrabile solo nel gruppo non sottoposto ad analgesia peridurale e
che aveva ricevuto Petidina per via endovenosa \textsuperscript{(14)}.
In uno studio cinese con solo l'abstract in inglese, 124 donne che
avevano partorito per via vaginale erano state divise nel gruppo PA
(n=75) e nel gruppo nPA (n=49) e non erano state trovate differenze
significative per quanto riguarda il tempo di insorgenza della montata
lattea, la quantità di latte prodotto o il calo ponderale del neonato
nei due gruppi \textsuperscript{(31)}. In un altro studio Cinese con
abstract in inglese, sono state osservate 170 donne sane che hanno
partorito vaginalmente senza complicanze ostetriche, di queste 96 hanno
avuto PA e 74 no. Il gruppo Epidurale aveva avuto un accorciamento nel
tempo di inizio della montata lattea, una maggiore quantità di latte
prodotto e più alti livelli di prolattina a 48h dal parto. Le donne del
gruppo PA inoltre riportavano una migliore analgesia nel post partum e
uno stato mentale migliore rispetto al gruppo di donne nPA. Ma
l'abstract non riporta quali farmaci sono stati utilizzati per la PA e a
quale dosaggio, se è stata utilizzata o meno ossitocina e la
disponibilità di servizi per il supporto all'allattamento nell'ospedale
dove si è svolto lo studio \textsuperscript{(32)}. Nel nostro studio il
primo attacco al seno si è comunque potuto realizzare entro la prima ora
dal parto in entrambi i gruppi (80\% nel gruppo PA e 70\% nel gruppo
nPA) senza differenze statisticamente significative trai due gruppi nei
tempi di insorgenza della montata lattea e sul tipo di allattamento alla
dimissione e a 7 giorni dal parto. Questi dati dovranno poi essere
ulteriormente analizzati, alla luce dei dati sui tassi di successo
dell'allattamento a 3 e 5 mesi dal parto, per vedere se lo skin-to-skin
possa avere impatto o meno sulla durata dell'allattamento.

Anche per quanto riguarda il peso neonatale alla dimissione risultava
una differenza statisticamente significativa (p=0.01 con test T-student)
tra i due gruppi. Il peso medio alla dimissione del gruppo PA (3.262 Kg)
risultava più elevato del peso medio alla dimissione del gruppo nPA
(2.904 Kg) nonostante la tendenza in questo gruppo alla maggiore
integrazione al latte materno con altri alimenti. Il calo ponderale
medio è risultato più elevato nei bambini del gruppo nPA (6.14\%)
rispetto al gruppo PA (6\%) (p=0.01 con il test T-student) anche qui
nonostante la maggiore tendenza nel primo gruppo all'integrazione al
latte materno. Questo dato può essere spiegato dal notevole calo
ponderale avvenuto in 3 neonati di questo gruppo, di cui però non è
possibile bapire le motivazioni.

Non si sono dimostrate differenze statisticamente significative per
quanto riguarda i tempi di insorgenza della montata lattea tra i due
gruppi PA e nPA. Cosi come sul tipo di allattamento alla dimissione, con
una percentuale di allattamento esclusivo al seno del 55\% nel gruppo PA
e del 40\% nel gruppo nPA. Il peso neonatale al controllo così come il
guadagno ponderale medio giornaliero sono stati simili per i due gruppi,
mettendo in evidenza l'assenza, a distanza di 7 giorni dal parto di
differenze statisticamente significative sia per quanto riguarda sia la
crescita ponderale del bambino che tipo di allattamento: in entrambi i
gruppi le madri allattavano al seno (PA 100\%) (nPA 95\%).

\hypertarget{limiti-dello-studio}{%
\subsection{Limiti dello studio}\label{limiti-dello-studio}}

Il campione per il momento è esiguo per trarre delle conclusioni che
possano essere generalizzate alla popolazione totale. Non è stato
utilizzato nessuno score per oggettivare il dolore durante travaglio, ma
ci si è basati soltanto sul riferimento del sollievo dal dolore da parte
della madre per considerare adeguata la somministrazione dei farmaci.
Non sappiamo se a dosaggi superiori a quelli somministrati in peridurale
sinora nel nostro studio, si verifichino, come in letteratura, una
riduzione degli indici di vitalità neonatale (APGAR) e dello score PIBBs
e se questi portino poi, sia nel breve termine che nel lungo termine ad
interferire con l'allattamento, determinandone una sua sospensione. Non
tutti i neonati sono stati valutati precisamente a 12 ore o a 48 ore ma
si è verificata una certa elasticità, per cui alcuni PIBBs sono stati
presi più precocemente, altri appena dopo 12 ore, quindi in momenti in
cui le concentrazione plasmatiche dei farmaci nei neonati possono essere
state diverse e aver influito in maniera diversa sul comportamento
rilevato. Non è stato possibile dosare le concentrazioni dei farmaci
impiegati per l'analgesia del travaglio di parto, somministrati per via
peridurale, né nel plasma della madre né nel sangue cordonale del
neonato, quindi non possiamo dedurre in maniera certa che un determinato
punteggio neonatale del PIBBS score sia stato influenzato da una
determinata concentrazione di farmaco nel neonato, perché questa non è
stato possibile misurarla direttamente. Così come non è stato possibile
dosare livelli di ossitocina plasmatica endogena, quindi non sappiamo
con certezza se la somministrazione di ossitocina sintetica esogena
alteri in qualche modo la secrezione pulsatile dell'ossitocina endogena
ed i suoi livelli plasmatici.

\hypertarget{conclusioni}{%
\subsection{Conclusioni}\label{conclusioni}}

In conclusione, l'allattamento è un fenomeno complesso, multifattoriale
e la partoanalgesia rappresenta soltanto uno dei possibili fattori
influenzanti.

Nel nostro studio la partoanalgesia ha dimostrato di non avere effetti
negativi sull'allattamento al seno dei neonati poiché quasi tutte le
donne, sia quelle sottoposte ad analgesia che le altre, allattavano al
seno sia al momento della dimissione dall'ospedale, che al controllo a 7
giorni, con una percentuale di allattamento materno esclusivo al seno al
momento della dimissione, più alto (55\% vs 40\%) tra le donne che hanno
ricevuto analgesia. Inoltre sia gli score neurocomportamentali legati
all'allattamento (il PIBBS score nel nostro studio), che 1'APGAR score
(in particolare quello rilevato a 5 minuti), che in alcuni studi
risultano essere ridotti dalla somministrazione peridurale di oppioidi,
sono risultati essere addirittura più alti, nel nostro studio, nei
bambini nati da madri sottoposte a partoanalgesia. Possiamo ipotizzare
che questo si sia verificato per la riduzione, per mezzo dell'analgesia,
dello stress materno legato al travaglio e al parto, con il risultato di
un migliore adattamento alla vita extrauterina per i bambini nati da
queste madri. Soprattutto il dosaggio totale del Fentanyl, come
dimostrato in altri studi, potrebbe avere un ruolo significativo nel
ridurre le capacità del neonato di attaccarsi correttamente al seno e
dare inizio ad un corretto processo di lattogenesi da parte della madre.
Questo effetto è stato però dimostrato con l'utilizzo di alti dosaggi
(\textgreater{}150-200 mcg di dose totale somministrata in peridurale
durante il travaglio di parto), che nel nostro studio non sono mai stati
raggiunti, con rilievo nonostante questo di una analgesia efficace,
facendo probabilmente prevalere gli effetti benefici di riduzione dello
stress materno-fetale ad opera dell'analgesia di parto rispetto ai suoi
possibili effetti collaterali di sedazione sul neonato.

Nel nostro studio, inoltre, è emerso che la PA è risultata essere
associata ad un maggiore utilizzo di ossitocina, sia in termini di
frequenza che di dose totale somministrata, ma l'utilizzo
dell'ossitocina non ha poi avuto effetti negativi sull'allattamento, sia
nell'immediato postpartum per quanto riguarda il tempo intercorso tra il
parto e il primo attacco al seno, sia sull'insorgenza della montata
lattea, avvenuta con tempi simili nei due gruppi, sia sul successo
dell'allattamento al seno, visto che quasi tutte le madri del gruppo PA
allattavano al seno al momento della dimissione e i loro neonati sono
stati quelli che durante la degenza in ospedale hanno avuto minori
necessità di integrazione con altri alimenti.

La differenza di calo ponderale medio nei due gruppi, e dipeso neonatale
medio alla dimissione, maggiori nel gruppo nPA, risultata
statisticamente significativa, nonostante le integrazioni al latte
materno, potrebbe essere spiegata dalla notevole perdita di peso
avvenuta in 3 neonati di questo gruppo.

I dati sulla montata lattea e sul tipo di allattamento alla dimissione e
a 7 giorni non mostrano differenze significative nei due gruppi PA e
nPA, sopportando quindi l'ipotesi che la partoanalgesia non interferisca
con l'instaurarsi dell'allattamento e la sua prosecuzione nel primo
periodo post partum.

Anche il peso neonatale al controllo e il guadagno ponderale medio
giornaliero non hanno mostrato differenze statisticamente significative
tra i due gruppi PA e nPA avvalorando ulteriormente l'ipotesi che la
partoanalgesia non interferisca con lo sviluppo nei normali
comportamenti neonatali necessari alla nutrizione e con la normale
produzione di latte da parte della madre.

Dovremo poi confrontare questi dati con quelli derivanti dalle
interviste telefoniche a 3 e 5 mesi (che attualmente non abbiamo) che
valuteranno il tipo di allattamento (esclusivo al seno, misto, solo
latte artificiale) e in caso di sospensione dell'allattamento materno
quali sono state le cause che hanno portato alla sua sospensione, per
cercare di capire se la partoanalgesia possa essere un fattore di
disturbo anche a lungo termine sull'allattamento al seno.

\emph{\textbf{Bibliografia:}}

\begin{enumerate}
\def\labelenumi{\arabic{enumi}.}
\item
  National Collaborating Centre for Women's and Children's Health,
  commissioned by National Institute for Healt and Clinica' Excellence
  (NICE). Intrapartum care; Care of Healty Women and Their Babies During
  Childbirth: Clinical Guideline. London: RCOG Press 2007.
\item
  Jordan (2009) Associations of drugs routenely given in labour with
  breastfeeding at 48 hours: analysis of the Cardiff Births Survey.
  BJOG. 2009 Nov;116(12):1622-9; discussion 1630-2
\item
  Keverne (1994) Maternal behaviour in sheep and its neuroendocrine
  regulation. Acta Paediatr Suppl 397:47-56
\item
  Osterman MJ, Martin JA: Epidural and spinai anesthesia use during
  labor: 27-state reporting area, 2008. Natl Vital Stat Rep 2011;
  59:1-13, 16
\item
  Ngan Kee WD, Khaw KS, Ng FF, Ng KK, So R, Lee A: Synergistic
  interaction between fentanyl and bupivacaine given intrathecally for
  labor analgesia. ANESTHESIOLOGY 2014; 120:1126-36
\item
  Sultan P, Murphy C, Halpern S, Carvalho B: The effect of low
  concentrations versus high concentrations of local anesthetics for
  labour analgesia on obstetric and anesthetic outcomes: A
  meta-analysis. Can J Anaesth 2013; 60:840-54
\item
  (Intrapartum Epidural Fentanyl and Breastfeeding in the Immediate
  Postpartum Period: a Randomized, Controlled, Double-blinded Study
  ClinicalTrials.gov Identifier: NCT01497509 Ashley Szabo, University
  Hospitals Cleveland Medical Center )
\item
  (Editorial Views October 2017) (Labor Epidural Analgesia and
  Breastfeeding David H. Chestnut, M.D. Anesthesiology 10 2017, Vol.127,
  593-595. doi:l0.1097/ALN.0000000000001794)
\item
  (Righard L, Alade MO. Effect of delivery room routines on success of
  first breast-feed. Lancet 1990; 336: 1105-7.)
\item
  (The neonatal neurologic and adaptive capacity score (NACS)
  Amiel-Tison C, Barrier G, Shnider SM, Levinson G, Hughes SC, Stefani
  SJ. Anesthesiology. 1982 Jun;56(6):492-3.)
\item
  French CA, Cong X, Chung KS: Labor Epidural Analgesia and
  Breastfeeding: A Systematic Review. J Hum Lact 2016; 32:507-20
\item
  Szabo AL: Review article: Intrapartum neuraxial analgesia and
  breastfeeding outcomes: limitations of current knowledge. Anesth Analg
  2013; 116:399-405
\item
  Beilin Y, Bodian CA, Weiser J, Hossain S, Arnold I, Feierman DE,
  MartinG, Holzman I: Effect of labor epidural analgesia with and
  without fentanyl on infant breast-feeding: A prospective, randomized,
  double-blind study. ANESTHESIOLOGY 2005; 103:1211-7
\item
  Wilson MJ, MacArthur C, Cooper GM, Bick D, Moore PA, Shennan A; COMET
  Study Group UK: Epidural analgesia and breastfeeding: randomised
  controlled trial of epidural techniques with and without fentanyl and
  a non-epidural comparison group. Anaesthesia 2010; 65:145-53
\item
  (Epidural Labor Analgesia-Fentanyl Dose and Breastfeeding Success: A
  Randomized Clinical Trial. Lee An, McCarthy RJ, Toledo P, Jones MJ,
  White N, Wong CA. Anesthesiology. 2017 Oct;127(4):614-624.
  doi:10.1097/ALN.0000000000001793)
\item
  (Epidural Analgesia and Lactation Mert Akbas1 , A. Baris Akcan2 1
  Department of Anesthesiology, Division of Algology, Faculty of
  Medicine, Akdeniz University, Antalya,Turkey 2 Department of
  Pediatrics, Faculty of Medicine, Akdeniz University, Antalya, Turkey
  EAJM 2011; 43: 45-9 )
\item
  (Righard L, Alade MO. Effect of delivery room routines on success of
  first breast-feed. Lancet 1990; 336: 1105-7. 39.)
\item
  (Perez-Escamilla R. Pollit E, Lönnerdal B, Dewey KG. Infant feeding
  policies in maternity wards and their effect on breast-feeding
  success: an analytical overview. Am J Public Health 1994; 84: 89-97.)
\item
  (Chung M,Raman G,Trikalinos T,Lau J, Ip S. Interventions in primary
  care to promote breastfeeding: an evidence review for the U.S.
  Preventive Services Task Force. Ann Intern Med 2008; 149: 565-82.)
\item
  (Montgomery A, Hale TW; Academy of Breastfeeding Medicine. ABM
  clinical protocol \#15: analgesia and anesthesia for the breastfeeding
  mother, revised 2012. Breastfeed Med. 2012;7(6):547-553)
\item
  Chen et al (Stress during labor and delivery and early lactation
  performance. Am J Clin Nutr. 1998;68:335-344)
\item
  (Radzyminski S. The effect of ultra low dose epidural analgesia on
  newborn breastfeeding behaviors. J Obstet Gynecol Neonatal Nurs.
  2003;32(3):322-331.)
\item
  Early Hum Dev. 1999 Jul;55(3):247-64. The development of preterm
  infants' breastfeeding behavior. Nyqvist KH1, Sjödén PO, Ewald U.)
\item
  (Acta Paediatr. 2008 Jun;97(6):776-81.
  doi:10.1111/j.1651-2227.2008.00810.x. Early attainment of
  breastfeeding competence in very preterm infants. Nyqvist KH1.)
\item
  The effect of epidural analgesia on rates of episiotomy use and
  episiotomy extension in an inner-city hospital. Newman MG1, Lindsay
  MK, Graves W. J Matern Fetal Med. 2001 Apr;10(2):97-101.
\item
  (Epidural analgesia and severe\_perineal tears: a literature review
  and large cohort study. Loewenberg-Weisband Y, Grisaru-Granovsky S,
  Ioscovich A, Samueloff A, Calderon-Margalit R. J Matern Fetal Neonatal
  Med. 2014 Dec;27(18):1864-9. doi:10.3109/14767058.2014.889113. Epub
  2014 Mar 3. Review.)
\item
  (The effect of epidural analgesia on the occurrence of obstetric
  lacerations and on the neonatal outcome during spontaneous vaginal
  delivery. Bodner-Adler B, Bodner K, Keiomberger O, Wagenbichler P,
  Kaider A, Husslein P, Mayerhofer K.Arch Gynecol Obstet. 2002
  Dec;267(2):81-4.)
\item
  (Effects of epidural analgesia on labor length, instrumental delivery,
  and neonatal short-term outcome. Hasegawa j, Farina A, Turchi G,
  Hasegawa Y, Zanello M, Baroncini S. J Anesth. 2013 Feb;27(1):43-7.
  doi:10.1007/s00540-012-1480-9. Epub 2012 Sep 11.)
\item
  (Factors influencing the likelihood of instrumental delivery success.
  Aiken CE, Aiken AR, Brockelsby JC, Scott JG. Obstet Gynecol. 2014
  Apr;123(4):796-803. doi:10.1097/A0G.0000000000000188.)
\item
  (The Apgar Score. AMERICAN ACADEMY OF PEDIATRICS COMMITTEE ON FETUS
  AND NEWBORN; AMERICAN COLLEGE OF OBSTETRICIANS AND GYNECOLOGISTS
  COMMITTEE ON OBSTETRIC PRACTICE. Pediatrics. 2015 Oct;136(4):819-22.
  doi:10.1542/peds.2015-2651.)
\item
  (Chen YM, Li Z, Wang AJ, Wang JM. Effect of labor analgesia with
  ropivacaine on the lactation of paturients {[}in Chinese{]}. Zhonghua
  Fu Chan Ke Za Zhi. 2008;43(7):502-505.)
\item
  (Wang BP, Li QL, Hu YF. Impact of epidural anesthesia during delivery
  on breast feeding {[}in Chinese{]}. Di Yi Jun Yi Da Xue Xue Bao.
  2005;25(1):114-115, 118.)
\item
  (Early skin-to-skin contact for mothers and their healthy newborn
  infants Elizabeth R Moorel, Gene C Anderson2, Nils Bergman3, and
  Therese Dowswell4 Cochrane Database Syst Rev.; 5: CD003519.
  doi:10.1002/14651858.CD003519.pub3.)
\item
  (De Carvalho, M, Robertson S, Friedman A, Klaus M.l 983; Effect of
  frequent breast-feeding on early milk production and infant weight
  gain. Pediatrics. 1983 Sep;72(3):307-11. 35.
\item
  Risk factors for suboptimal infant breastfeeding behavior, delayed
  onset of lactation, and excess neonatal weight loss. Dewey KG,
  Nommsen-Rivers LA, Heinig MJ, Cohen RJ. Pediatrics. 2003 Sep;112(3 Pt
  1):607-19.
\item
  Krehbiel (1987) Peridural anesthesia disturbs maternal behavior in
  primiparous and multiparous parturient ewes. Physiol Behav 40:463-472
  37.
\item
  Levy (1992) Intracerebral oxytocin is important for the onset of
  maternal behavior in inexperienced ewes delivered under peridural
  anesthesia.
\item
  Williams (2001) Physiological regulation of maternal behavior in
  heifers: roles of genital stimulation, intracerebral oxytocin release,
  and ovarian steroids. Biol Reprod 65:295-300
\item
  Goodfellow (1983) Oxytocin deficiency at delivery with epidural
  analgesia. Br J Obstet Gynaecol 90:214-219
\item
  P.D.Fortea (201 4) Oxytocin administered during labor and
  breast-feeding: a retrospective cohort study. J Matern Fetal Neonatal
  Med, 2014;27(15): 1598-1603
\item
  Olza (2012) Newborn feeding behaviour depressed by intrapartum
  oxytocin: a pilot study. Acta Paediatr 2012;101:749-54
\item
  Bell (2013) Fetal exposure to synthetic oxytocin and the relationship
  with prefeeding cues within one hour postbirth. Early Hum Dev
  2013;89:137-43
\item
  Odent (2012) The role of the shy hormone in breastfeeding. Midwifery
  Today Int Midwife 2012;101:14
\item
  RU1S (1981) Oxytocin enhances onset of lactation amaong mothers
  delivering prematurely. Br Med J (Clin Res Ed) 1981;283:340-2
\item
  Fewtrell (2006) Randomised, double blind trial of oxitoci nasal spray
  in mothers expressing breast milk for preterm infants. Arch Dis Child
  Fetal Neonatal Ed 2006;91:F169-74
\item
  Anim-Somuah (2005) Epidural versus non-epidural or no analgesia in
  labour. Cochrane Database of Systematic Reviews 2005 4, 1-51
\item
  Alehagen (2005) Fear, pain and stress hormones during childbirth. J
  Psychosom Obstet Gynaecol 26:153-165
\item
  Loewenberg-Weisband Y, Grisaru-Granovsky S, loscovich A, Samueloff A,
  Calderon-Margalit R. Epidural analgesia and severe perineal tears: a
  literature review and large cohort study. J Matern Fetal Neonatal Med.
  2014 Dec;27(18):1864-9. doi:10.3109/14767058.2014.889113. Epub 2014
  Mar 3. Review.
\item
  Newman MG, Lindsay MK, Graves W. The effect of epidural analgesia on
  rates of episiotomy use and episiotomyextension in an inner-city
  hospital. J Matern Fetal Med. 2001 Apr;10(2):97-101.
\item
  Bodner-Adler B, Bodner K, Kimberger O, Wagenbichler P, Kaider A,
  Husslein P, Mayerhofer K. The effect of epidural analgesia on
  obstetric lacerations and neonatal outcome during spontaneous vaginal
  delivery. Arch Gynecol Obstet. 2003 Jan;267(3):130-3.
\item
  Hasegawa J, Farina A, Turchi G, Hasegawa Y, Zanello M, Baroncini S.
  Effects of epidural analgesia on labor length, instrumental delivery,
  and neonatal short-term outcome. J Anesth. 2013 Feb;27(1):43-7.
  doi:10.1007/s00540-012-1480-9. Epub 2012 Sep 11.
\item
  Anwar S, Anwar MW, Ahmad S. EFFECT OF EPIDURAL ANALGESIA ON LABOR AND
  ITS OUTCOMES. J Ayub Med Coll Abbottabad. 2015 Jan-Mar;27(1):146-50.
\item
  Gizzo S, Di Gangi S, Saccardi C, Patrelli TS, Paccagnella G, Sansone
  L, Barbara F, D'Antona D, Nardelli GB. Epidural analgesia during
  labor: impact on delivery outcome, neonatal well-being, and early
  breastfeeding. Breastfeed Med. 2012 Aug;7:262-8.
  doi:10.1089/bfm.2011.0099. Epub 2011 Dec 13
\item
  Chen et al (Stress during labor and delivery and early lactation
  performance. Am J Clin Nutr. 1998;68:335-344)
\item
  Zhang G, Feng Y. {[}Effect of epidural analgesia on the duration of
  labor stages and delivery outcome{]}. Nan Fang Yi Ke Da Xue Xue Bao.
  2012 Aug;32(8):1218-20.
\item
  Bilié N, Djaković I, Kličvan-Jaić K. Rudman SS, Ivanec Ž. EPIDURAL
  ANALGESIA IN LABOR- CONTROVERSIES. Acta Clin Croat. 2015
  Sep;54(3):330-6.
\item
  (De Chateau 1977, Long-term effect on mother-infant behavior of extra
  contact during the first hours postpartum. II. A follow-up at three
  months. Acta Paediatrica Scandinavica 66,145-151.)
\item
  Widstrom (1987) Gastric suctioning in healthy newborn infants. Effects
  on circulation and developing feeding behaviour. Acta Paediatrica
  Scandinavica 76, 56-72.
\item
  Wiklund (2007) Epidural analgesia: Breast-feeding success and related
  factors. Midwifery. 2009 Apr;25(2):e31-8. Epub 2007 Nov 5.
\item
  Baumgarder (2003) Effects of labour epidural anaesthesia on
  breast-feeding of healty full-term newborns delivered vaginally.
  Journal of the American Board of Family Practice 16, 7-13
\item
  Ransjo-Arvidson (2001) Maternal analgesia during labor disturbs
  newborn behaviour: effects on breastfeeding, temperature and crying.
  Birth 28, 5-12
\item
  Bystrova (2008) Effect of closeness versus separation after birth and
  influence of swaddling on mother-infant interaction one year later: a
  study in St. Petersburg Birth (in press)
\item
  De Chateau (1984) Long term effect on mother-infant behaviour of extra
  contact during the first hour post partum. III Folluw-up at one year.
  Scand J Soc Med 12:91-103
\end{enumerate}

\end{document}
